
\section{Sentences}\label{sec:SL-intro}


In this chapter we will be looking at logical connectives in a natural language,
 in particular English.  Just exactly what logical connectives are will be 
 explained as we proceed.


\subsection{Sentences}

Our natural languages have components that can be combined to form larger 
linguistic units. For example, we can combine words to form sentences.  
Sentences to form paragraphs. In this part of our class, our interest is in a 
certain type of sentences. Which type? In English there is an easy way of 
isolating the type of sentence we are interested in: prefixing them with `It is 
true that' results in a grammatically correct sentence. Let me explain:

In English, the following are all sentences:

\begin{itemize}

 \item Grass is always greener on the other side.

 \item Is your neighbor's lawn greener than yours?

 \item Let's go check out our neighbor's lawn.

\end{itemize}

If you prefix each of these with `It is true that', you get:

\begin{itemize}

 \item It is true that grass is always greener on the other side.

 \item It is true that is your neighbor's lawn greener than yours?

 \item It is true that let's go check out our neighbor's lawn.

\end{itemize}

Of the latter group, only the first is a grammatically correct sentence of 
English so the only of interest to us in the former group is the first sentence.  
It is the only one of the three that grammatically speaking could be true or 
false in the sense that English grammar allows us to say that it is true by 
simply prefixing `it is true that' to it.  From now on, when we speak of 
sentences, we only mean such sentences.  The following are all sentences:

\begin{itemize}

 \item One plus one is two.

 \item Vanilla ice cream is delicious.

 \item I don't like steak to be well done.

 \item The Earth is flat.

 \item If the Mongol Empire had not expanded into Eastern Europe, Russia as we 
  know it would not have existed.

 \item Free will is an illusion.

 \item The Middle Ages ended with the fall of Constantinople.

 \item It is morally impermissible to tell a lie.

\end{itemize}

Each of these results in a grammatically correct sentence of English when 
prefixed with `it is true that':

\begin{itemize}

 \item It is true that one plus one is two.

 \item It is true that vanilla ice cream is delicious.

 \item It is true that I don't like steak to be well done.

 \item It is true that the Earth is flat.

 \item It is true that if the Mongol Empire had not expanded into Eastern Europe, 
  Russia as we know it would not have existed.

 \item It is true that free will is an illusion.

 \item It is true that the Middle Ages ended with the fall of Constantinople.

 \item It is true that it is morally impermissible to tell a lie.

\end{itemize}

That we can prefix a sentence with `it is true that' and get a grammatically 
correct sentence means just that: the resulting sentence is grammatically 
correct. It does not mean that the sentence is true. For instance, in the list 
just given, some are true, others definitely false, some controversial, and in 
some cases it might even be unclear whether what's being asserted really is the 
sort of thing that can be true or false. But that does not matter for our 
purposes.\footnote{English has a rather simple grammar that allows just 
 prefixing a sentence with `it is true that' to yield a grammatically correct 
 sentence. Your native tongue might not allow anything this simple to isolate 
 the sentences we are interested in. When in doubt about which of the sentences 
in your native tongue are of interest, the simple solution is to translate them 
into English and see.} 

\subsection{Connectives}

The prefix `it is true that' is a device that we can use to form a new sentence 
out of another sentence. There are other such devices. Some of them can take two 
sentences and form a new sentence out of them. We call devices that can be used 
to form new sentences out of other sentences \emph{connectives}. For example, 
`it is true that' is a connective. There are also others.

For example, take the following two sentences:

\begin{slist}
 \item Henry VIII was surrounded by men named Thomas.\footnote{Here are some of 
  the people who were at one point or another of great influence under Henry 
 VIII: Thomas Wolsey, Thomas More, Thomas Cromwell, Thomas Boleyn, Thomas Howard, 
Thomas Seymour, Thomas Cranmer, Thomas Wriothesley.} 

 \item Henry VIII was too lazy to remember names.
\end{slist}

These can be combined in various ways to form more complex sentences:

\begin{slist}

\item Henry VIII was surrounded by men named Thomas \emph{and} Henry VIII was 
 too lazy to remember names.


\item Henry VIII was surrounded by men named Thomas \emph{because} Henry VIII 
 was too lazy to remember names.

\end{slist}

We can also take a single sentence and turn it into another sentence by, for 
instance, adding a prefix:

\begin{slist}
\item \emph{It is not true that} Henry VIII was surrounded by men named Thomas.

\item \emph{Anne believes that} Henry VIII was surrounded by men named Thomas.
\end{slist}

Notice that the results are all sentences in our sense since they can all be 
prefixed with `it is true that'. For example:

\begin{slist}

\item It is true that Henry VIII was surrounded by men named Thomas \emph{and} 
 Henry VIII was too lazy to remember names.

\item It is true that Henry VIII was surrounded by men named Thomas 
 \emph{because} Henry VIII was too lazy to remember names.

\item It is true that \emph{it is not true that} Henry VIII was surrounded by 
 men named Thomas.

\item It is true that \emph{Anne believes that} Henry VIII was surrounded by men 
 named Thomas.

\end{slist}

So `and', `because', `it is not true that', and `Anne believes that' are all 
connectives. There are many more connectives in English. Sentences that are 
formed from other sentences through the use of connectives are called 
\emph{compound} sentences. Sentences 3 through 10 are all compound sentences. 

We will only be interested in some connectives. Of the ones we have seen so far, 
we will be interested in `and' and `it is not the case that' because these  
qualify as logical connectives. But we will not be interested in `because' and 
`Anne believes that' because these do not qualify as logical connectives. Just 
why that is so will become clearer as we move along. 

Here is some terminology that we will be using:
\begin{itemize}

 \item We will be speaking of \emph{atomic} sentences and \emph{compound} 
  sentences. Atomic sentences are the sentences that we use as basic building 
  blocks for forming compound sentences. Compound sentences are formed by 
  manipulating atomic sentences using connectives (like combining them with the 
  word `and', or prefixing a sentence with `it is not the case that').  

\item We will speak of \emph{sentences} to cover both atomic and compound 
 sentences.

 \item If a sentence is true, we will say that its \emph{truth value} is true.  
  If the sentence is not true, we will say that its truth value is false.

\end{itemize}

Just like atoms that chemistry talks about are not at all simple, atomic 
sentences need not be simple sentences.  But when we designate a sentence as 
atomic, we will disregard any internal structures that the atomic sentence might 
have---in particular, we disregard whether they might themselves have parts that 
are sentences. So calling a sentence `atomic' does not say anything about the 
sentence except that we are treating it as a basic building block out of which 
we form other sentences. What the sentence says might be highly complex.






\section{Necessary and Sufficient Conditions}

We will often be talking about necessary and sufficient conditions for this and 
that to be the case.

A \emph{necessary condition} is a condition that must be met for something to be 
the case.  For instance, in order to vote in congressional elections in the U.S.  
you must be 18 years or older. So being 18 years or older is a necessary 
condition for it to  be the case that you are eligible to vote in congressional 
elections.  The standard way to express a necessary condition in this course 
will be:

\begin{center}
 \blank only if \blank

\end{center}

The blanks need to be filled in so that the result is a grammatically correct 
sentence of English. For instance:

\begin{center}
 You are eligible to vote in U.S. congressional elections \bemph{only if} you 
 are 18 years or older.
\end{center}

The bold face is just so you see the `only if'. It will usually not be bolded or 
otherwise emphasized.


A \emph{sufficient condition} for something to be the case is a condition that 
suffices to make it the case but is not necessary. For instance, suppose that 
satisfactorily passing a third semester French course suffices to meet your 
school's foreign language requirements for graduation but that it is not 
necessary to pass a third semester French course.  You could take Chinese, 
Arabic, etc.  Maybe there are even ways of avoiding a language course.  The 
standard way of expressing a sufficient condition in this course will be:

\begin{center}
 \blank if \blank
\end{center}

For instance:

\begin{center}
 You meet the foreign language requirement \bemph{if} you satisfactorily pass a 
 third semester French course.
\end{center}

Some conditions are both necessary and sufficient. The standard way of 
expressing that is:

\begin{center}
 \blank if, and only if, \blank
\end{center}

For instance, 

\begin{center}
 You meet the mathematical and formal reasoning requirement at your school 
\bemph{if, and only if,} you take a course in mathematics, computer science, 
formal logic, or statistics.  \end{center}

`if, and only if,' is often abbreviated as `iff.', like this:

\begin{center}
 You meet the mathematical and formal reasoning requirement at your school 
\bemph{iff.} you take a course in mathematics, computer science, formal logic, 
or statistics.  \end{center}



\section{Simple Symbolization}\label{sec:SL-compound}

Let us use the following sentences as our atomic sentences:

\begin{slist}

\item Snow is white.

\item The moon is made of cheese.

\item Marco Polo visited China.

\item Marco Polo met people who had visited China.

\item Marco Polo completely made up his story.

\end{slist}

Until further notice, these are the only atomic sentences there are. 

The following are all compound sentences that can be formed from these atomic 
sentences.  I am putting brackets around the sentences making up the compound 
sentence. The brackets can also be nested when compound sentences are combined 
to form further compound sentences.

\begin{slist}
 
\item \pp{Snow is white} and \pp{the moon is made of cheese}.

 \item It is not the case that \pp{the moon is made of cheese}.

 \item Either \pp{Marco Polo visited China} or \pp{Marco Polo met people who had 
  visited China}.

\end{slist}


Spelling out the sentences can get tedious quickly so let's use upper case Roman 
letters (A, B, C, ...) to represent sentences. For example,

\begin{lkey*}

\item[A] Snow is white.

\item[B] The moon is made of cheese.

\item[C] Marco Polo visited China.

\item[D] Marco Polo met people who visited China.

\item[E] Marco Polo completely made up his story.

\end{lkey*}

With these symbolization keys, we can rewrite the compound sentences above as:

\begin{slist}

 \item A and B.

 \item It is not the case that B.

 \item Either C or D.

\end{slist}

We can also produce more complex sentences:

\begin{slist}

\item \pp{A and \pp{C or D}} or \pp{B and \pp{\pp{it is not the case that C} and 
 E}}
 .

\end{slist}

The brackets are used to make clearer what goes with what. Notice that if we had 
stipulated that only A through D are atomic sentences, the last sentence would 
not be one of the compound sentences that can be formed: it needs E and that 
would not be available as an atomic sentence. So changing which sentences are 
the atomic sentences makes a difference to what compound sentences can be 
formed.  That is why it is important to be explicit about what the atomic 
sentences are. 



\section{Logical Connectives}

Let's take a closer look at the connective `and'. It takes two sentences and 
produces a third one. Let's use the following symbolization keys:

\begin{lkey*}

\item[A]  Henry VIII was surrounded by men named Thomas. 

\item[B]  Henry VIII was too lazy to remember names.

\end{lkey*}

A and B are both sentences. That means either can be true or false. They could 
be both true, both false, or one of them true and the other false. That gives us 
four possibilities. For each possibility, we can say whether \pp{A and B} is 
true or not:

\begin{itemize}

 \item A and B both true: \pp{A and B} is true.

 \item A false and B true: \pp{A and B} is false.

 \item A true and B false:  \pp{A and B} is false.

 \item A and B both false: \pp{A and B} is false.

\end{itemize}


This means the truth value of \pp{A and B} is determined by the truth values of 
A and B. This is an important feature of `and'. Not all connectives are like 
this. For instance, consider the connective `because'. Even if both A and B are 
true, that does not determine whether \pp{A because B} is true: even if both A 
and B are true, that does not mean that the reason Henry was surrounded by 
Thomases is that he was too lazy to remember names---maybe Thomas happened to be 
an extremely popular name at the time.

When the truth value of a compound sentence using a connective is determined by 
the truth values of its component sentences, the connective is said to be  
truth-functional. 

For the remainder of this course, a \emph{logical connective} is a 
truth-functional connective.\footnote{In more advanced areas of logic, there are 
logical connectives that are not truth-functional but we will not be discussing 
them.} This means that `and' is a logical connective, `because' is not. Moreover, 
when we speak of \emph{compound sentences}, we will mean compound sentences 
constructed using only logical connectives unless otherwise noted. 

We can represent how the connective `and' works by the following table:

\begin{center}

 \begin{center}
\begin{tabular}{c c||c}
 A  & B & \pp{A and B}\\
\hline
\emph{T} & \emph{T} & \emph{T} \\
\emph{F} & \emph{T} & \emph{F} \\
\emph{T} & \emph{F} & \emph{F}  \\
\emph{F} & \emph{F} & \emph{F} \\
\end{tabular}
\end{center}


\end{center}
 
I am putting the compound sentence inside brackets like \pp{A and B} to make 
clear that I am talking about the whole compound sentence.  Each row of the  
table shows a possible combination of the truth values of A and B and the truth 
value of \pp{A and B} for that combination of truth values for the atomic 
sentences.  Let's call each possible combination of the truth values for the 
atomic sentences an \emph{interpretation}.  Note that an interpretation 
determines the truth value of each and every atomic sentence. Let us say that 
the table gives us the \emph{truth condition} of the compound sentence \pp{A and 
B}.

The difference between `and' and `because' is that you cannot produce a complete 
table like this for the latter. If you tried, you would get:

\begin{center}

 
\begin{tabular}{c c||c}
 A  & B & \pp{A because B}\\
\hline
\emph{T} & \emph{T} & \emph{??} \\
\emph{F} & \emph{T} & \emph{F} \\
\emph{T} & \emph{F} & \emph{F}  \\
\emph{F} & \emph{F} & \emph{F} \\
\end{tabular}


\end{center}

It is plausible that if at least one of A and B is false, \pp{A because B} is 
also false. But if both A and B are true, that does not determine whether \pp{A 
because B} is true. A logical connective must be such that what goes into each 
cell of the truth table is determinate.

Are there logical connectives beside `and'? There are. For instance, `it is not 
the case that' is a logical connective. The truth condition for `it is not the 
case that A' is given by:

\begin{center}


\begin{tabular}{c c||c}
 A  & B & \pp{it is not the case that A}\\
\hline
\emph{T} & \emph{T} & \emph{F} \\
\emph{F} & \emph{T} & \emph{T} \\
\emph{T} & \emph{F} & \emph{F}  \\
\emph{F} & \emph{F} & \emph{T} \\
\end{tabular}

\end{center}

Notice that \pp{it is not the case that A} is true if and only if A is false 
(second and fourth interpretation). The truth value of B does not matter. So we can use a 
smaller table:

\begin{center}


\begin{tabular}{c ||c}
 A  & \pp{it is not the case that A}\\
\hline
\emph{T}  & \emph{F} \\
\emph{F}  & \emph{T} \\
\end{tabular}

\end{center}


Another logical connective is `or'. The truth condition for \pp{A or B} is given 
by:

\begin{center}

 
\begin{tabular}{c c||c}
 A  & B & \pp{A or B}\\
\hline
\emph{T} & \emph{T} & \emph{T} \\
\emph{F} & \emph{T} & \emph{T} \\
\emph{T} & \emph{F} & \emph{T}  \\
\emph{F} & \emph{F} & \emph{F} \\
\end{tabular}


\end{center}

\pp{A or B} is true if at least one of A and B is true.

We will be making heavy use of tables like this. In order to facilitate checking 
the correctness of tables and comparing results, let us stick to a particular 
way of listing the interpretations. For tables with two atomic sentences, the 
interpretations will always be listed as above. We will see a more general 
procedure for listing all the interpretations for arbitrary number of atomic 
sentences in the next chapter.  For now, get used to the above order of listing 
interpretations in the case of two atomic sentences.

What about more complex compound sentences? Let's take a look at \pp{it is not 
the case that \pp{A and B}}. Can we state its truth condition? We can:

\begin{center}

 \begin{tabular}{cc|c||c}
A & B & \pp{A and B}& it is not the case that \pp{A and B}\\
\hline
\emph{T} & \emph{T} & \emph{T} & \emph{F}\\
\emph{F} & \emph{T} & \emph{F} & \emph{T}\\
\emph{T} & \emph{F} & \emph{F} & \emph{T}\\
\emph{F} & \emph{F} & \emph{F} & \emph{T}\\
\end{tabular}


\end{center}

The third column tells us the truth value of \pp{A and B} for each interpretation, 
and the last column the truth value of \pp{it is not the case that \pp{A and B}} 
for each interpretation. As you would expect, it is false iff. both A and B are true, 
and that happens only on the first interpretation.

Because of the way logical connectives work, any compound sentence constructed 
using only logical connectives is going to be such that its truth value is 
determined by the truth values of the atomic sentences involved. You see an 
example of that in the above example of \pp{it is not the case that \pp{A and B}}
. And this has important implications which we will discuss in the next section.


\section{All the Possible Truth Conditions}
\label{sec:NaturalLanguagePossibleTruthConditions}

\pp{A and B} and \pp{A or B} have different truth conditions. And because they 
have different truth conditions, when you assert \pp{A and B} you assert 
something different than when you assert \pp{A or B}. What do you assert when 
you assert \pp{A and B}? Well, you assert the world is such that both A and B 
are true. Of course, you might be mistaken about that, but it is what you 
assert.  When you assert \pp{A or B}, you assert that at least one of A and B is 
true but you are not going as far as asserting that both are true---you are not 
excluding that possibility but you are not committed to that---and that is why 
asserting \pp{A or B} is different from asserting \pp{A and B}. 

Thus, when you assert one sentence and then another and you actually are 
asserting two things, the two sentences must have different truth-conditions.  
For in asserting a sentence, you are asserting that the world is a certain way, 
viz., one of the ways in which the sentence can be true. If you assert a second 
sentence, you are asserting that the world is such that one of the ways in which 
the second sentence can be true obtains. If the two sentences have the same 
truth condition, then in asserting the two sentences, you are not asserting 
anything different about the world: the way the world is according to your first 
assertion is identical to the way the world is according to your second 
assertion.  You are saying the same thing twice over, in perhaps different ways.  
Two sentences with the same truth condition `say the same thing.' 

Here is a question: given two atomic sentences, how many different things could 
you say using compound sentences?

Consider an arbitrary compound sentence and its truth condition. As noted 
earlier, the truth value of the compound sentence is determined by the truth 
values of the atomic sentences. That is, for each interpretation the compound 
sentence is either true or false. And that means given two atomic sentences, the 
following
are the possible ways the truth-condition of the compound sentence could be:

\begin{itemize}
 
 \item The sentence is false in any of the four interpretations.
  
 \item The sentence is true in only one of the four interpretation.  
  
 \item The sentence is true in two of the four interpretations.  

 \item The sentence is true in three of the four interpretations.  

 \item The sentence is true in any of the four interpretations. 

\end{itemize}

Here are some examples. Let us write \pp{A\comp B} to stand for some compound 
sentence. \pp{A\comp B} could stand for a very simple compound sentence like 
\pp{A or B} but could be any compound sentence that can be formed using logical 
connectives. Also, it could stand for a compound sentence that does not use both 
atomic sentences as components like \pp{it is not the case that A}.  That is to 
say \pp{A\comp B} is simply any compound sentence that can be formed given the 
availability of A and B as atomic sentences.
 
Suppose the sentence \pp{A\comp B} is true in exactly one of the possible 
interpretations. In that case, its truth condition must be given by one of the following 
tables:

\begin{center}
 \showtt{tc-c1}
 \showtt{tc-c3}
 \showtt{tc-c2}
 \showtt{tc-c4}
\end{center}

Here is an example of  a truth condition that make a sentence come out true in 
two possible interpretations:

\begin{center}
 \showtt{tc-c7}
\end{center}

Given this table, the sentence is true if A and B are both true or both false.  
There are altogether six ways in which a compound sentence could be true in 
exactly two interpretations. I will leave figuring them out to the exercises.  The 
following list exhausts the possible truth conditions of any compound sentence 
\pp{A\comp B}:

\begin{itemize}
 \item 1 way in which the compound sentence is true in \emph{no} interpretation.
 \item 4 ways in which the compound sentence is true in exactly one interpretation.
 \item 6 ways in which the compound sentence is true in exactly two interpretations.
 \item 4 ways in which the compound sentence is true in exactly three 
  interpretations.
 \item 1 way in which a compound sentence is true in all four interpretations.
\end{itemize}

 Altogether, we have 16 ways the truth condition of a compound sentence 
 \pp{A\comp B} can be.


So, given just A and B as atomic sentences,   there are only sixteen different 
things you could say by compound sentences.

Notice that the number of compound sentences must be much larger than 16 as can 
be seen from the following simple sequence of compound sentences:

\begin{enumerate}

 \item \pp{A and B}.

 \item \pp{B and \pp{A and B}}.

 \item \pp{A and \pp{B and \pp{A and B}}}.

\item \pp{B and \pp{A and \pp{B and \pp{A and B}}}}.

\item \pp{A and \pp{B and \pp{A and \pp{B and \pp{A and B}}}}}.
 
\item[\vdots]

\end{enumerate}

You can continue the sequence ad infinitum which means that even with just two 
atomic sentences, there are infinitely many compound sentences. But there are 
only 16 different things you can say using those infinitely many compound 
sentences---all we've got is infinitely many ways of saying 16 things.


Of course, there is nothing special about the particular atomic sentences A and 
B: given two atomic sentences, whatever they are, there are only 16 different 
things that can be said by compound sentences constructed out of them.






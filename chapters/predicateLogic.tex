

\section{Introduction}\label{secAxioms}
\newcommand{\model}{\p{\mathfrak I}}

In the previous chapters we have seen sentential logic. It is called
`sentential' because it focuses on sentences and logical relations that hold
between sentences. We have seen that the proof system we constructed is 
complete. Any tautology can be proven and that means that whenever the truth of 
\p{s} guarantees the truth of \p{t}---i.e. \p{s\lentails t}---we can also prove 
that \p{\seq{}{s\limplies t}}. And this means that any valid argument can be 
turned into a successful derivation. You might therefore hope that we are done 
with formalizing reasoning.  But, alas, things are more complicated.

Consider the following piece of reasoning:

\begin{Example}\label{ex:ch4-1}
 All whales are mammals. Ed is a whale. So Ed is a mammal.
\end{Example}

This seems like a fine piece of reasoning. On the assumption that all whales are 
mammals and that Ed is a whale, it surely follows that Ed is a mammal. But how 
would we formalize this?

Let's use the following keys:

\begin{lkey*}
\item[A] All whales are mammals.
\item[B] Ed is a whale.
\item[C] Ed is a mammal.
\end{lkey*}

Then the above reasoning looks as follows:

\begin{argument}
 \aitem \sqA{\Gamma}{A}{premise}
 \aitem \sqA{\Delta}{B}{premise}
 \aitem \sqA{\Gamma,\Delta}{C}{1,2,??}
\end{argument}


There is no inference rule that would license the move to line 3 from lines 1 
and 2: It cannot be that given some arbitrary three sentences \p{s_1,s_2} and 
\p{s_3}, we can infer from \p{\seq{\Gamma}{s_1}} and \p{\seq{\Delta}{s_2}} to        
\p{\seq{\Gamma,\Delta}{s_3}}. But the argument about Ed in the example above is 
surely cogent. So we have an example of an argument that can be stated in plain 
English without difficulties, but cannot be formalized in sentential logic. 

Let's think about the example a bit more. Compare:

\begin{itemize}

 \item All whales are mammals.

 \item If anything is a whale, then it is a mammal.

\end{itemize}

These two sentences say the same thing: the first is true iff. the second is  
true. Let's use this and rewrite the example:

\begin{Example}\label{ex:edTheWhale}
 All whales are mammals. That is to say, if anything is a whale, then it is a 
 mammal. So if Ed is a whale, then Ed is a mammal. Ed is a whale. It follows 
 that Ed is a mammal.
\end{Example}

Breaking this up into lines, we get:

\begin{argument}

 \aitem If anything is a whale, then it is a mammal.
 \aergo If Ed is a whale, then Ed is a mammal.
 \aitem Ed is a whale.
 \aergo Ed is a mammal.

\end{argument}

Lines 5 through 7 pose no difficulties for sentential logic. We can partially 
formalize the argument as (using the same keys as before):

\begin{argumentN}[8]
\aitem ???
 \aitem \sqA{\Delta}{B\limplies C}{8,?}
 \aitem \sqA{\Theta}{B}{premise}
 \aitem \sqA{\Delta,\Theta}{C}{9,10,\condE}

\end{argumentN}

The trouble is with formalizing the move from 4 to 5. How do we get to line 9 
(which corresponds to line 5 of the version in English)?

You might be tempted by the thought that 4 is a conditional connecting the 
sentences `anything is a whale' with `it is a mammal'. So let's try the 
following keys:

\begin{lkey*}

\item[D] Anything is a whale.
\item[E] It is a mammal.

\end{lkey*}

And then we get:

\begin{argumentN}[8]

 \aitem \sqA{\Gamma}{D\limplies E}{premise}
 \aitem \sqA{\Delta}{B\limplies C}{8,?}
 \aitem \sqA{\Theta}{B}{premise}
 \aitem \sqA{\Delta,\Theta}{C}{9,10,\condE}

\end{argumentN}

But this will not do. First of all, there is no way to derive 9 from 8. Secondly, 
and much more importantly for our purposes, 4 (if anything is a whale, then it 
is a mammal) is \emph{not} a conditional. If it were, then both `anything is a 
whale' and `it is a mammal' must be sentences. Take the former. That might be 
interpreted to mean that everything is a whale. That's false. Given the truth 
conditions of a conditional, the whole conditional would therefore be true no 
matter what the consequent. For instance, it would be true that if anything is a 
whale, it is a reptile. And that can't be right. Moreover, take `it is a 
mammal'.  What does the `it' in `it is a mammal' refer to? There is nothing in 
particular the `it' refers to.  And that means that `it is a mammal' cannot be 
true or false: there is nothing of which it is claimed that \emph{it} is a 
mammal.  So it is not a sentence, and 8 misconstrues 4.

Let's think again. There should be no doubt that 5 follows from 4. When you 
compare the two, you will notice that they have more in common than the `if, 
then' construction.  They are both instances of:

\begin{center}
 if \blank{} is a whale, then \blank{} is a whale.

\end{center}

The difference between 4 and 5 consists in how the blanks are filled in. This is 
a similarity that cannot be captured with sentential logic because the language 
of sentential logic is incapable of revealing structures smaller than whole 
sentences. But it seems that the cogency of the move from 4 to 5 has something 
to do with these sub-sentential structures.

In the following we will introduce a formal language---the language of predicate 
logic---that is capable of representing sub-sentential structures and formulate 
rules of inferences that can exploit them. Taken together, we will have a system 
of \emph{first-order predicate logic}.

\section{Quantifiers and Variables}

Take a look again at 

\begin{center}

 if \blank is a whale, then \blank is a mammal

\end{center}


This is not a sentence. It is more like a \emph{template} of a sentence. You can 
generate a sentence by filling in the blanks. Let us use letters like \p{x} and 
\p{y} to represent the blanks. Using letters allows us to distinguish between 
templates like: 

 \begin{itemize}
  \item if \p{x} is a whale, then \p{x} is a mammal
  \item if \p{x} is a whale, then \p{y} is a mammal
 \end{itemize}

The rule is that blanks represented by the same letter within a template must be 
filled in the same way. So in the first template, if you plug `Ed' into the 
first \p{x}, you must plug `Ed' into the second as well. But blanks represented 
by different letters can be filled in differently. For instance, in the second 
template you can fill `Ed' into the \p{x} and `Gina' into \p{y} and get the 
sentence `if Ed is a whale, then Gina is a mammal' which is a sentence you 
cannot get by filling in the blanks of the first template. We will call letters 
that represent blanks \emph{variables}.

`If Ed is a whale, then Ed is a mammal' and `if Ed is a whale, then Gina is a 
mammal' are both sentences. Why? Because each of them makes a claim about the 
world and can be true or false. A template does not yet make any claims about 
the world and that is why the templates above are not sentences. We can get 
sentences by plugging into the variables.

But there is another way of getting a sentence out of the templates. Consider  
prefixing a template with `for any x':
\begin{center}

 for any x, if x is a whale, then x is a mammal

\end{center}

In more ordinary English, we might put this as: everything is such that if it is 
a whale, then it is a mammal. Or more colloquially: if anything is a whale,
 it is a mammal. Even more colloquially: all whales are mammals.  This, too, is 
 a sentence since it is the sort of thing that is either true or false.  And 
 this is not a complex sentence formed by connecting two sentences.  Rather, it 
 is a sentence whose parts are not themselves sentences.

There is one more way of getting a sentence out of a template that we will be 
discussing in this chapter. We can prefix the template above with `there is an x 
such that' and get a sentence as well:

\begin{center}

 there is an x such that if x is a whale, then x is a mammal

\end{center}


The language of predicate logic formalizes these features of sentence templates.

Before we proceed, let me make some more observations. The template like `if 
\p{x} is whale, then \p{x} is a mammal' is itself made out of two templates:
\begin{center}

 x is a whale

 x is a mammal

\end{center}

Connecting these templates with the `if ... then ...' construction gives us the 
template `if \p{x} is a whale, then \p{x} is a mammal'. But we could easily 
connect these two with a different connective. For example, `\p{x} is a whale 
and \p{x} is a mammal', or `\p{x} is a whale or \p{x} is a mammal', etc. So 
templates can be turned into compound templates using the familiar logical 
connectives.

While some templates are compound templates that connect two or more templates, 
others like `\p{x} is a whale' are not. These atomic (i.e., non-compound)
templates take subject-verb form in English. We will have something similar in 
our formal language. 

Not all sentences of English take a simple subject-verb form. In fact, most 
don't. For instance, `Hermione is married to Ron' involves Hermione and Ron, not 
just Hermione. And `Harry mediates between Hermione and Ron' involves Harry, 
Hermione and Ron. A template for generating such a sentence would be `\p{x} 
mediates between \p{y} and \p{z}'.  We want our formal language to have such 
templates as well.





\section{Formalization}

Let us introduce a formal language \lL[Q]{}. The Q stands for quantifier. Just 
as with \lL, our aim is to construct a bare-bones language that is simple and 
does the job of representing what we want to represent but nothing else. 

Let us start. \lL[Q]{} has \emph{formulas}. The intuitive characterization of a 
formula is that  both sentences and templates are formulas. The formulas are 
such that:

\begin{itemize}

 \item If \p{\phi} is a formula, so is \p{\lnot\phi}.

 \item If \p{\phi} and \p{\psi} are formulas, so are \p{(\phi \land \psi)}, 
  \p{(\phi \lor \psi)}, \p{(\phi \limplies \psi)}.

\end{itemize}

(\p{\phi} is Greek lowercase phi, and \p{\psi} is Greek lowercase psi). Notice 
that if we restrict \p{\phi} and \p{\psi} to sentences, this is the same we had 
for \lL.  So we can expect our new language \lL[Q] to carry over many of the 
features of \lL.

What does a formula look like? We need a general characterization that can cover 
both sentences and templates. Compare the sentence `Ed is a whale' with the 
template `\p{x} is a whale'. They have something in common, namely the `is a 
whale' part. But they differ in that `Ed' is a proper name whereas \p{x} is a 
variable.  

\begin{itemize}
 
 \item \lL[Q] has \emph{predicates} corresponding to expressions like `is a 
  whale' in English.  

\item \lL[Q] has \emph{variables} like the \p{x} and \p{y} that we have been 
 using.

\item \lL[Q] has \emph{constants} which correspond to proper names likes `Ed' in 
 English. 

 \end{itemize}

 If we combine a predicate with a variable, we get a template. If we combine a 
 predicate with a constant, we get a sentence. Let us introduce the word 
 \emph{term} to cover both variables and constants. Then a formula is something 
 that combines a predicate with a term.

 Some predicates must be combined with multiple terms to generate a formula.  
 For instance, the predicate corresponding to `... mediates between ... and ...' 
 combines with three terms to yield a formula; e.g., `\p{x} mediates between 
 \p{y} and \p{z}'. A predicate that combines with \emph{n} terms to yield a 
 formula is called an \emph{n}-place predicate. So  `... mediates between ...  
 and ...' is a 3-place predicate.  We will say the following about formulas in 
 \lL[Q]:

\begin{itemize}

 \item If \p{\mathcal{F}} is an n-place predicate, \p{\tau_1}, \p{\tau_2}, 
  \ldots, \p{\tau_n} are terms, then \p{\mathcal{F}\tau_1\tau_2\ldots\tau_n} is 
  a formula.

\end{itemize}

(\p{\tau} is the Greek lowercase tau). So we give \lL[Q] a bunch of predicates, 
a bunch of constants, and a bunch variables, and that gives us  a bunch of 
formulas (in fact, infinitely many).
You can read, for example, \p{Fx} as `\p{x} is F' and \p{Fc} as `c is F'. In 
this course, we will mostly be dealing with 1-place, or \emph{monadic}, 
predicates.

This looks scarier than it is. Notice that once we have formulas, they combine 
to form more complex formulas according to  the same principles as the compound 
sentences of \lL{} we saw earlier. So what is new is a way of representing the 
internal structure of sentences and templates. Think of \lL[Q]{} as a language 
with a very primitive grammar: any atomic formula starts with a predicate and 
has one or more terms attached.  Because the grammar is extremely primitive, the 
order of the terms is important (just as in English, word order matters).

Some examples:  

\begin{itemize}
 
 \item Let \p{W} be a 1-place predicate that can be translated into `is a whale' 
  in English, and let \p{x} be a variable.  Then \p{Wx} is a formula of \lL[Q]{} 
  which can be translated as `\p{x} is a whale' (so it's a template, rather than 
  a sentence). Let \p{e} be the name for Ed in \lL[Q]{}
---i.e., \p{e} is a constant of \lL[Q]{}. Then \p{We} is a formula of \lL[Q]{} 
that we can translate as `Ed is a whale' (so it's a sentence). 

\item Let \p{M} be a 3-place predicate \p{M} such that \p{Mxyz} translates into 
 `\p{x} mediates between \p{y} and \p{z}' (it's often easier to explain what a 
 predicate in \lL[Q]{} means by providing a full template for a sentence). if 
 \p{h_1} is a constant that names Harry, \p{h_2} a constant that names Herminone, 
 and \p{r} a constant that names Ron, then \p{Mh_1h_2r} can be translated as 
 `Harry mediates between Herminone and Ron'.

\item Let \p{T} be a 2-place predicate such that \p{Txy} translates into `\p{x} 
 is taller than \p{y}'. Then if \p{s} is a constant naming Sherlock and \p{j} is 
 a constant naming John, \p{Tsj} can be translated as `Sherlock is taller than 
 John' while \p{Tjs} can be translated into `John is taller than Sherlock'. Word 
 order matters.

 \end{itemize}
 
 
 There is one more important feature of \lL[Q].  Recall that we can take a 
 template like `\p{x} is a whale', prefix it with `for any x' or `there is an 
 \p{x}  such that' and get a sentence out of it. To accomplish the same in 
 \lL[Q], we will say:

\begin{itemize}

 \item if \p{\phi} is a formula and \p{\upsilon} a variable,  
  \p{\lforall\upsilon\phi} is a formula.

\end{itemize}

(\p{\upsilon} is Greek lowercase upsilon). You can read \p{\lforall\upsilon} as 
`for all \p{\upsilon}' and it is known as the
\emph{universal quantifier}. For instance , \p{\forall x Wx} means that for all 
\p{x}, \p{x} is a whale (assuming \p{Wx} means that \p{x} is a whale)---or more 
colloquially, everything is a whale.  And if \p{Mx} means that \p{x} is a mammal, 
then the sentence 'for all \p{x}, if \p{x} is a whale, then \p{x} is a mammal' 
can be put as \p{\lforall x (Wx\limplies Mx)}.

We will also have another quantifier called the existential quantifier:

\begin{itemize}

 \item if \p{\phi} is a formula and \p{\upsilon} a variable, 
  \p{\lthereis\upsilon\phi} is a formula.

\end{itemize}

You can read \p{\lthereis\upsilon} as `there is an \p{\upsilon} such that'. For 
instance, \p{\lthereis x Wx} means `there is an \p{x} such that \p{x} is a 
whale' (again keeping the same meaning for \p{W} as above)---or, more 
colloquially, `there are whales' (the plural `whales' is not meant to commit us 
to the existence of multiple whales; there might be just one).


Altogether, we can characterize  a language \lL[Q]---a language of first-order 
predicate logic, as it is known---in the following way:

\begin{enumerate}

 \item There are infinitely but countably many constants.

 \item There are infinitely but countably many variables.

 \item There are infinitely but countably many predicates.

 \item If \p{\mathcal{F}} is an \emph{n}-place predicate, \p{\tau_1}, \p{\tau_2}, 
  \ldots, \p{\tau_n} are terms (constants and variables), then \p{\mathcal{F}
   \tau_1\tau_2\ldots
  \tau_n} is a formula.

 \item If \p{\phi} is a formula, then \p{\lnot\phi} is a formula.

 \item If \p{\phi} and \p{\psi} are formulas, then \p{(\phi \land \psi)}, 
  \p{(\phi \lor \psi)}, \p{(\phi \limplies \psi)} are all formulas.

  \item If \p{\phi} is a formula and \p{\upsilon} a variable, then \p{\lforall 
	\upsilon
   \phi} is a formula.

 \item if \p{\phi} is a formula and \p{\upsilon} a variable,  
  \p{\lthereis\upsilon\phi} is a formula.
 
 \item Nothing else is a formula.

\end{enumerate}

It is customary (but not required) to use lower case letters from the end of the 
Roman alphabet like \p{x}, \p{y}, \p{z} as variables.    And it is customary to 
use upper case Roman alphabet letters like \p{F}, \p{G}, \p{H} for predicates, 
and lower case letters from the start of the Roman alphabet like \p{a}, \p{b}, 
\p{c} for constants.  If we need more, we will generally resort to subscripts.  
And for ease of exposition, we will call the quantifiers \emph{connectives} so 
we can say that the main connective of \p{\lforall x Fx} is \p{\lforall x}. 

It is good practice to always explicitly state which letters are used as 
variables, which as predicates, and which as constants. And no letter should be 
used as both a variable and a constant in the same context.

Notice that a formula of \lL[Q]{} need not be a sentence. We need to do a little 
bit more work to be able to clearly specify what a sentence of \lL[Q]{} is.



\section{Scope, Bound and Free Variables}


Let \p{Lxy} stand for `\p{x} loves \p{y}'. We can prefix this with a universal 
quantifier:

\begin{argument}
 \aitem \plshp{UxLxy}
\end{argument}

This means `For any \p{x}, \p{x} loves \p{y}'. The universal quantifier does not 
affect \p{y}. \p{y} is still a simple blank so that 1 is still a template.  We 
say that the \p{x} in \p{Lxy} is \emph{bound} by the quantifier \p{\forall x}
.  \p{y} is not bound by any quantifier. We say that \p{y} is a \emph{free} 
variable---\p{y} is still just a blank and you are free to plug into it a 
constant to get a sentence, but you cannot plug a constant into \p{x} given that 
it is bound.  

We can prefix another quantifier to bind \p{y}. E.g.:

\begin{argument}
 \aitem \plshp{UyUxLxy}
\end{argument}

For any \p{y} and any \p{x}, \p{x} loves \p{y}. That is, we can plug anything 
into \p{y} and plug anything into \p{x} and the result will be true. And this 
means that in this case, the order of the quantifiers does not matter:

\begin{argument}
 \aitem \plshp{UxUyLxy}
\end{argument}

3 means the same as 2. But this is a special case. Usually, the order of 
quantifiers matters greatly. Start with 1 again but let's think of a different 
way of binding \p{y}:

\begin{argument}
 \aitem \plshp{XyUxLxy}
\end{argument}

There is a \p{y}, such that any \p{x} loves \p{y}. So if everybody loves Raymond, 
4 is true. But if there isn't anyone who is loved by everyone, 4 would be false.  
Now consider the following:

\begin{argument}
 \aitem \plshp{UxXyLxy}
\end{argument}

For any \p{x}, there is a \p{y} such that \p{x} loves \p{y}. Notice that this 
does not require that there be someone whom everyone loves. Maybe no one loves 
Raymond. But that does not conflict with everyone's having someone they love 
(it's just that that someone is never Raymond).  5 would still be true in that 
case, but not 4. 4 and 5 mean different things.

To capture the difference between 4 and 5, logicians speak of the \emph{scope} 
of a quantifier. The scope of a quantifier is simply the formula to which it is 
attached. The scope of the existential quantifier \p{\lthereis y} in 4 is the 
formula \p{\forall xLxy}. But in 5 it is \p{Lxy}. The scope of the universal 
quantifier \p{\lforall x} in 4 is \p{Lxy} but in 5 it is \p{\lthereis yLxy}.  
Differences in scope usually matter a great deal.

A quantifier always names the variable it binds. The quantifier 
\p{\lforall\upsilon} binds all instances of \p{\upsilon} that would be free 
without the quantifier. For instance, in 2 above, the quantifier \p{\lforall y} 
binds \p{y} in \p{\lforall xLxy} because the \p{y} in that formula is still 
free. On the other hand, consider:
\begin{itemize}
 \aitem \p{\lforall x\lforall xLxy}
\end{itemize}
In 6, the left-most quantifier \p{\lforall x} does not bind anything because it 
names the variable to be bound as \p{x} but there is no free \p{x} within its 
scope.  And the \p{y} remains free in 6. It is in general bad practice to attach 
a quantifier that does not bind anything even though the rules for \lL[Q]{} do 
not forbid it. 

There are some tricky issues to be aware of when dealing with scope. Take the 
following two templates:
\begin{argument}
 \aitem \p{Wx}
 \aitem \p{Mx} 
\end{argument}

(\p{Wx} means \p{x} is a whale, \p{Mx} means \p{x} is a mammal).
We can take 7 and prefix it with a quantifier:
\begin{argument}
 \aitem \p{\lthereis xWx}
\end{argument}

We can take 9 and 8 and connect them with the conditional to get a new formula:

\begin{argument}
\aitem \p{\lthereis xWx \limplies Mx}
\end{argument}

Notice that the \p{x} in \p{Mx} is still free because it is not within the scope 
of the existential quantifier at the left end. This means we can bind the \p{x} 
in  \p{Mx}. For example:

\begin{argument}
 \aitem \p{\forall x(\lthereis xWx\limplies Mx)}
\end{argument}

Notice that the universal quantifier binds the \p{x} in \p{Mx} but does not bind 
the \p{x} in \p{Wx} because the latter \p{x} is already bound by the existential 
quantifier. While 11 is allowed by the rules of \lL[Q]{} it is confusing and it 
will make life really difficult when we move to inferences involving 
quantifiers.  The solution is to rename one of the \p{x}'s.  They are just 
blanks with labels to keep track of what gets plugged into what.  So there is no 
change in meaning if we rename one of the \p{x}'s. E.g.:

\begin{argument}
 \aitem \p{\forall y(\lthereis xWx\limplies My)}
\end{argument}

For any \p{y}, if there are whales, then \p{y} is a mammal (that's probably 
false).

\textbf{Going forward, we will require that all variables of a given name within 
the scope of a quantifier must be bound by the same quantifier.} 11 fails this 
requirement because not all  \p{x}'s within the scope of the universal 
quantifier are bound by it. But 12 meets this requirement. 

One more point of notation. Given the variable \p{\upsilon}, we will write 
\p{\phi(\upsilon)} to stand for a formula in which \p{\upsilon} occurs as a free 
variable.  \p{\phi} can be any formula so it may contain other variable that are 
already bound by quantifiers within \p{\phi}. But when we write 
\p{\phi(\upsilon)}, \p{\upsilon} is free within \p{\phi}. So when we write 
\p{\lforall x\phi(x)}, the universal quantifiers binds the \p{x} in \p{\phi} 
because \p{x} is free within \p{\phi}. When there are multiple free variables in 
\p{\phi}, we will indicate that by a comma-separated list; e.g., 
\p{\phi(\upsilon_1,\upsilon_2)} is a formula in which there are two free 
variables \p{\upsilon_1} and \p{\upsilon_2}.

We will also sometimes write \p{\phi(\kappa)} where \p{\kappa} is a constant to 
mean that some free variable in \p{\phi} was replaced by the constant   
\p{\kappa}. As I said earlier, it is very good practice to always be explicit 
about which letters are used for constants, and which for variables.


\subsection{Sentences of \lL[Q]{}}

We now can specify which formulas of \lL[Q]{} are sentences: a formula \lL[Q]{} 
is a sentence iff. it contains no free variables.






\section{Semantics for \lL[Q]}\label{sec:semanticsPL}

So far we have relied on an intuitive understanding of what sentences of \lL[Q]{} 
mean. It is time to introduce a more formal account of the semantics of \lL[Q].  
In particular, we want to have a more rigorous definition of when a sentence of 
\lL[Q]{} is true.

In the case of sentential logic, we used interpretations to figure out things 
like whether a given sentence is a tautology. An interpretation there assigned 
truth values to atomic sentences, and that enabled us to figure out the truth 
value of a compound sentence in that interpretation. We will do something 
similar for our language of predicate logic. However, since the smallest 
building blocks of the sentences of predicate logic are not entire sentences, 
but terms and predicates, our interpretations make assignments to those. Since 
the building blocks are not sentences, the interpretation cannot  assign truth 
values to them. So what does an interpretation do?  The following specifies what 
an interpretation does.


  
First, an interpretation of \lL[Q]{} specifies a \emph{domain of discourse}, or 
just \emph{domain}.  This is the set of things our language \lL[Q]{} can talk 
about.  `Thing' is understood in the broadest possible sense: anything that can 
be named can be a thing. There is no requirement of any kind of commonality 
among the members of the domain save that they all must be `things.' And a 
domain of discourse can have finitely many, countably many, or even uncountably 
many things as members. The set \{Julius Caesar, Beethoven's Ninth Symphony, the 
United Nations Security Council, Cantor's Diagonal Argument\} can be a  domain 
of discourse, and so can be the set of all rational numbers.  However, the 
domain of discourse cannot be empty.

Next, an interpretation assigns to each constant of \lL[Q]{} a member of the domain of 
discourse.  That is, it specifies for each constant a thing in the domain that 
is named---or \emph{referred to}---by that constant. Let's call the thing named 
by a constant its \emph{referent}.  Notice that different constants can name the 
same thing: some things have more than one name like the first emperor of Rome 
who is variously known as Octavius, Octavian, Augustus.  But there is no 
constant without a referent; i.e., every name is a name of something.  Finally,  
not everything in the domain need be assigned as the referent of a 
constant---like most streets in Japan, some things might have no names.

Thirdly, an interpretation assigns to each predicate of \lL[Q]{} an 
\emph{extension}. For a one-place predicate \p{\mathcal{F}}, the interpretation 
assigns a set of members of the domain. Intuitively, the extension of 
\p{\mathcal{F}} is the set of all the members of the domain that are 
\p{\mathcal{F}}. For example, if the domain is the set of all students at the 
Claremont Colleges, an interpretation could assign all the Pomona College 
students to \p{P}, which would enable us to translate \p{Px} as `x is a Pomona 
College student'. For a two-place predicate \p{R}, the extension is a set of 
ordered pairs. For instance, the interpretation might assign to \p{H} all the 
pairs \p{<\mathfrak{m},\mathfrak{n}>} such that \p{\mathfrak{m}} is taller than 
\p{\mathfrak{n}} (those are m and n in a font style known as Fraktur). That 
would enable us to translate \p{Hxy} as `x is taller than y'. Notice that the 
pairs must be ordered, i.e., the order in which the members of the pairs are 
listed matters.  More generally, for each n-place predicate \p{\mathcal F}, the 
interpretation assigns a set of ordered n-tuplets as the extension of 
\p{\mathcal F}.  Finally, the extension of a predicate may be empty: there might 
be nothing that is \p{\mathcal F}.

The above suffices to explain what it takes for an atomic sentence of \lL[Q]
{} that consists of a single 1-place predicate and a constant  to be true given 
an interpretation \p{\mathfrak I} (that's an I in Fraktur): a sentence 
\p{\mathcal{F}
\kappa} is \emph{true in an interpretation \p{\mathfrak I}}  iff.  given \p{\mathfrak I} 
the referent of \p{\kappa} is in the extension of \p{\mathcal F}.   We will also 
sometimes say that  an interpretation \p{\mathfrak{I}} \emph{satisfies}  
\p{\mathcal F\kappa} true to mean that \p{\mathcal{F}\kappa} is true in 
\p{\mathfrak I}.
 
More generally, given an n-place predicate \p{\mathcal R}, an interpretation 
\p{\mathfrak{I}} makes true \p{\mathcal R\kappa_1 \ldots\kappa_n} iff.  
\p{<ref(\kappa_1),\ldots, ref(\kappa_n)>} is in the extension of \p{\mathcal R} 
where \p{ref(\chi)} is the referent of \p{\chi}.
This takes care of the quantifier-free atomic sentences.

What about quantifier-free compound sentences? We can adapt the understanding 
from sentential logic:
\begin{itemize}

 \item \p{\lnot s} is true in \p{\mathfrak I}  iff.  \p{s} is not true in 
  \model.

 \item  \p{s_1 \land s_2} is true in \model{}  iff.  both \p{s_1} and \p{s_1} 
  are true in \model.

 \item \p{s_1 \lor s_2} is true in \model{}  iff.  at least one of \p{s_1} and 
  \p{s_2} is true in \model.

\item  \p{s_1 \limplies s_2} is true in \model{} iff.   at least one of \p{\lnot 
 s_1} or \p{s_2} is true in \model.

\end{itemize}

We say that a sentence is \emph{false} iff. it is not true.

Let's move to quantified sentences---those sentences that have a quantifier as 
the main connective. Consider \p{\lforall x Fx}. We want this to be a reasonable 
translation of `everything is \p{F}' (that's how we have been treating 
quantified sentences). The obvious thing to say is: \p{\lforall\upsilon \mathcal 
F\upsilon} is true in \model{}  iff. the extension of \p{\mathcal F} contains 
every member of the domain. While this is unobjectionable, it does not help us 
with figuring out whether \p{\mathfrak I} makes true \p{\lforall \upsilon 
\mathcal (F\upsilon \limplies \mathcal G\upsilon)} and other such more complex 
quantified sentences.  Recall that the complex sentence just mentioned is not a 
compound sentence since it has no parts that are themselves sentences so that we 
cannot figure out whether it is true in \model{} by building up from its 
component sentences.


\newcommand{\llqex}{\p{\mathcal L_Q^+}}

Consider \p{\lforall \upsilon \phi(\upsilon)}. Intuitively, this is true in 
\model, iff. everything in the domain is as described by \p{\phi}. So if 
everything in the domain is the referent of one constant or another, \p{\lforall 
\upsilon \phi(\upsilon)} is true in \model{} iff. \p{\phi(\kappa)} is true for 
all constants of \lL[Q]. But remember that not everything in the domain need be 
named by a constant. For instance, let the domain be people, and let \p{N} be a 
predicate that has all the nice people in the world in its extension but nobody 
else.  Let's make sure that all our constants refer to one or another person in 
the extension of \p{N}.  The not-nice people shall remain nameless. This  would 
make all sentences  of the form \p{N\kappa} true in the interpretation. Would 
that mean that everyone is nice? Hardly. What to do?

The solution is to pretend that there is a name for everything. Roughly,  
\p{\lforall \upsilon \phi(\upsilon)} is true in \model{} iff. any member 
\p{\mathfrak m} of the domain is such that if it had the name \p{\kappa}, 
\p{\phi(\kappa)} would be true.

Let's be more precise. There are two points to note. First, one way of getting a 
name for \p{\mathfrak m} is to take an existing name \p{c} and change its 
interpretation so that it refers to \p{\mathfrak m}. But that risks making a 
sentence that is true on the old interpretation false on the new one. For 
instance, if \p{c} originally refers to \p{\mathfrak t} and \p{\mathfrak t} is 
in the extension of \p{F}, then \p{Fc} is true in the original interpretation.  
But if we change the interpretation of \p{c} so that it refers to \p{\mathfrak s} 
and \p{\mathfrak s} is not in the extension of \p{F}, \p{Fc} would be false on 
the new interpretation. We do not want to introduce such changes in introducing 
a constant referring to \p{\mathfrak m}.  So what we do is introduce a \emph{new} 
constant \p{\kappa} and interpret this new constant as referring to \p{\mathfrak 
m} while keeping everything else unchanged. Even though \lL[Q]{} has infinitely 
many constants, we can add infinitely many more (e.g., we could start with 
having constants that have even numbers as subscripts, and then add constants 
with odd numbers as subscripts as the new ones).

Secondly, we cannot in general name everything all at once because there are 
domains that have more members than there are constants. E.g., the domain of 
real numbers has uncountably many members, whereas we have only countably many 
constants.\footnote{This is also why we can't solve the problem by modifying the 
 rules of the language so that every member of the domain has a name: that would 
rule out too many interesting collections of things from being the domain---like 
the real numbers.} So we are not asking what would happen if everything had a 
name.  Rather, we ask for each particular \p{\mathfrak m} what would happen if 
we introduced a new constant for \emph{it}.

So here is the more precise formulation for when a universally quantified 
sentence is true: \p{\lforall \upsilon \phi(\upsilon)} is true in \model{} iff.  
each member \p{\mathfrak m} of the domain is such that if \p{\kappa} is a new 
constant referring to \p{\mathfrak m}, \p{\phi(\kappa)} is true (alternatively,
there is no member \p{\mathfrak m} of the domain such that if \p{\kappa} is a 
new constant referring to \p{\mathfrak m}, \p{\phi(\kappa)} is false).

Similarly, \p{\lthereis \upsilon \phi(\upsilon)} is true in \model{} iff. there 
is some member \p{\mathfrak m} such that if \p{\kappa} is a new constant 
referring to \p{\mathfrak m}, then \p{\phi(\kappa)} is true.

You might be puzzled by the fact these specifications of when quantified 
sentences are true do not mention the `old' constants. Do they not matter? The 
point to notice is that if \p{\kappa_1} and \p{\kappa_2} are constants referring 
to the same thing in the domain, then \p{\phi(\kappa_1)} is true in \model{} 
iff. \p{\phi(\kappa_2)} is true in \model{}.

We now have everything we need to figure out whether or not  any sentence of 
\lL[Q]{} is true in a given interpretation \model. One thing worth noting is 
this: whether or not a sentence \p{s} is true in \model{} is solely a matter of 
the referents and extensions of the constants and predicates occurring in \p{s}.  
This will be of some use when discussing inferences in predicate logic.

 This all may  sound a bit  abstract and scary. Let's discuss some examples to 
 make things more concrete.

\subsection{Examples}

But first: we defined \lL[Q]{} to have infinitely many constants and predicates.  
However, we are usually interested in only a small number of them, namely those 
that appear in the sentences whose truths we are interested in. Since the 
semantic of \lL[Q]{} is such that the truth of a sentence in an interpretation 
only depends on the interpretation of the constants and predicates that occur in 
the sentence, we can safely ignore all the other constants and predicates. So in 
the following examples, only a small number of constants and predicates have 
their interpretations specified.

\subsubsection*{Example 1}

Let our language have 3 constants of interest: \p{j}, \p{t}, \p{h}. And let it 
have two one-place predicates of interest: \p{F}, \p{R}.

The following specifies our interpretation \model:

\begin{description}

 \item[Domain of Discourse:] Jerry the mouse, Tom the cat, Hobbes the tiger.

 \item[Referents of Constants:] \p{j} refers to Jerry, \p{t} refers to Tom, \p{h} 
  refers to Hobbes.

 \item[Extensions or Predicates:] The extension of \p{F} is \{Tom, Hobbes\}; the 
  extension of \p{R} is \{Jerry\}

\end{description}

Let's consider some sentences.

\begin{itemize}

 \item \p{Ft} is true in \model{} because the referent of \p{t}, Tom, is in the 
  extension of \p{F}.

 \item \p{Rt} is not true in \model{} because the referent of \p{t}, Tom, is not 
  in the extension of \p{R}.

 \item \p{Rj} is true in \model{}  because the referent of \p{j}, Jerry, is in 
  the extension of \p{R}.

 \item \p{Fj} is not true in \model{}  because the referent of \p{j}, Jerry, is 
  not in the extension of \p{F}.

 \item \p{ \lnot Fj} is true in \model{} because  \p{Fj} is not true in \model.  

 \item \p{Rj \limplies Rt} is false in \model{}  because neither \p{\lnot Rj} 
  nor \p{Rt} is true in \model.

 \item \p{ \lforall x Fx} is not true in \model{} because there is a member of 
  the domain, Jerry, such that if we introduce \p{c_{Jerry}} as a constant 
  referring to Jerry, \p{Fc_{Jerry}} is not true in the extended interpretation.

 \item \p{ \lforall x (Fx \lor Rx)} is true in model{}  because each member 
  \p{\mathfrak m} of the domain is such that if we introduce a new constant 
  \p{c_{new}} to refer to it,  \p{Fc_{new} \lor Rc_{new}} is true in the 
  extended interpretation. 

\end{itemize}


\subsubsection*{Example 2}

Same language as above but with the following interpretation \model:

\begin{description}

 \item[Domain of Discourse:] Jar Jar Binks, Harry.

 \item[Referents of Constants:] \p{j} refers to Jar Jar Binks, \p{t} refers to 
  Jar Jar Binks, \p{h} refers to Harry.

 \item[Extensions of Predicates:] The extension of \p{F} is \{Jar Jar Binks, 
  Harry\}. The extension of \p{R} is empty.

\end{description}

Let's consider some sentences:

\begin{itemize}

 \item \p{Ft} is true in \model{} because the referent of \p{t}, Jar Jar Binks, 
  is in the extension of F.

 \item \p{Rt} is not true in \model{}  because the referent of \p{t}, Jar Jar 
  Binks, is not in the extension of \p{R}.

 \item \p{Rj} is not true in \model{}  because the referent of \p{j}, Jar Jar 
  Binks,  is not in the extension of \p{R}.

 \item \p{Fj} is true in \model{}  because the referent of \p{j}, Jar Jar Binks, 
  is in the extension of \p{F}.

 \item \p{ \lnot Fj} is not true in \model{} because  \p{Fj} is true in 
  \model.
   

 \item \p{Rj \limplies Rt} is true in \model{} because \p{\lnot Rj} is true in 
  \model.

 \item \p{ \lforall x Fx} is true in \model{}  because each member \p{\mathfrak 
  m} of the domain is such that if we introduced a new constant \p{c_{\mathfrak 
  m}} to refer to it,  \p{Fc_{\mathfrak m}} is true in the extended interpretation (since 
  all members of the domain are in the extension of \p{F}).
  
 \item \p{ \lforall x (Fx \lor Rx)} is true in \model{}  because each member 
  \p{\mathfrak m} of the domain is such that if we introduce a new constant 
  \p{c_{\mathfrak m}} to refer to it, \p{Fc_{\mathfrak m} \lor Rc_{\mathfrak m}} 
  is true in the extended interpretation.

\end{itemize}


Notice how changing the interpretation changes which sentences are true in the interpretation and 
which not.



\subsubsection*{Example 3}

Same language as above but with the following interpretation:

\begin{description}

 \item[Domain of Discourse:] Jar Jar Binks, Harry, JFK.

 \item[Referents of Constants:] \p{j} refers to Jar Jar Binks, \p{t} refers to 
  Jar Jar Binks, \p{h} refers to Harry.

 \item[Extensions of Predicates:] The extension of \p{F} is \{Jar Jar Binks, 
  Harry\}. The extension of \p{R} is \{JFK\}.

\end{description}

for quantifier-free sentences, this interpretation behaves like the one in the previous 
example (check it). But:

\begin{itemize}

 \item \p{\lforall x Fx} is not true in \model{}  because introducing a new 
  constant \p{c_{J\!F\!K}} to refer to JFK results in an extended interpretation 
  in which \p{Fc_{J\!F\!K}} is not true  because JFK is not in the extension of 
  \p{F}
  .

\end{itemize}



\section{Entailment, Logical Truth, Contradiction}

Suppose \p{\lforall x (Fx\land Gx)} is true in \model. In that case, \p{\lforall 
x Fx} is also true in \model. Suppose \p{\lforall x (Fx \lor Gx)} and 
\p{\lforall x \lnot Gx} are both true in \model. In that case, \p{\lforall x Fx} 
is true in \model. So the truth in \model{} of some sentences can guarantee 
truth of some sentence \p{\phi}. We will write

\begin{center}

 \p{\Gamma \lentails \phi}

\end{center}

to mean that every interpretation is such if all the sentences in \p{\Gamma} are 
true, then \p{\phi} is also true. More colloquially, the truth of the sentences 
in \p{\Gamma} guarantees the truth of \p{\phi}.  We will read it as `\p{\Gamma} 
entails \p{\phi}', just as we did in sentential logic.

Some sentences are true in every possible interpretation. E.g.,\p{\lforall x (Fx 
\lor \lnot Fx)} is true in every possible interpretation. A sentence that is 
true in every interpretation is called a \emph{logical truth}. Any tautology of 
sentential logic is a logical truth, but not all logical truths are 
tautologies---just look at the one mentioned in this paragraph. We will write

\begin{center}

 \p{ \lentails \phi} 

\end{center}

to mean that \p{\phi} is a logical truth. 

A logical falsehood is a sentence that is false in every possible 
interpretation.

We say that  a set of sentences is \emph{satisfiable} iff. if there is an 
interpretation \model{} such that all sentences in the set are true in \model.  
Derivatively, we will say that an individual sentence is satisfiable iff.  the 
set containing it as its unique member is satisfiable.




\section{Reasoning With Quantifiers: Two Simple Rules}\label{sec:
Quantifier-Reasoning}

Let's go back to the argument in Example~\ref{ex:edTheWhale} (the one about Ed 
the whale). Our difficulty was that sentential logic could not account for the 
obvious cogency of that argument, and that seemed to be due to the limitations 
of \lL{}. Let's see how \lL[Q]{} can help.

Let \p{e} be a constant naming Ed, \p{x} a variable, and \p{Wx} and \p{Mx} mean 
`\p{x} is a whale' and `\p{x} is a mammal' respectively. We can formalize the 
argument as follows:

\begin{argument}

 \aitem \sqA{\Gamma}{\forall x(Wx\limplies Mx)}{premise}
 \aitem \sqA{\Gamma}{We\limplies Me}{1,?}
 \aitem \sqA{\Delta}{We}{premise}
 \aitem \sqA{\Gamma,\Delta}{Me}{2,3,\condE}

\end{argument}

The move from 1 to 2 is obviously cogent. If every \p{x} is such that if it is 
\p{W}, then it is \p{M}, it follows that any particular thing \p{e} is such that 
if it is \p{W}, it is \p{M}. We can turn this point into an inference rule:

\begin{infrule}

 \item[Universal Quantifier Elimination (\p{\lforall}E)] From \p{\seq{\Lambda}
  {\lforall\upsilon\phi(\upsilon)}}, infer \p{\seq{\Lambda}{\phi(\kappa)}}, for 
  any constant \p{\kappa}. (All instances of \p{\upsilon} must be replaced by 
  \p{\kappa}.)

\end{infrule}

This is also often called Universal Instantiation because it formalizes the 
activity of giving an instance of a universal claim.  We can now fill in the 
inference rule in the above argument:

\begin{argumentN}[1]

 \aitem \sqA{\Gamma}{\forall x(Wx\limplies Mx)}{premise}
 \aitem \sqA{\Gamma}{We\limplies Me}{1,\uniE}
 \aitem \sqA{\Delta}{We}{premise}
 \aitem \sqA{\Gamma,\Delta}{Me}{2,3,\condE}

\end{argumentN}

Here is another obvious inference rule:


\begin{infrule}

 \item[Existential Quantifier Introduction (\p{\lthereis}I)] Given a constant 
  \p{\kappa}, infer from \p{\seq{\Lambda}
 {\phi(\kappa)}} to \p{\seq{\Lambda}\lthereis\upsilon\phi(\upsilon)}. (At least 
 one---but not necessarily all---instances of \p{\kappa} must be replaced by 
 \p{\upsilon}.)

\end{infrule}

This also is obvious: if \p{\kappa} is \p{\phi}, then there is something that is 
\p{\phi}, viz. \p{\kappa}. Here is an example illustrating the use of \exI.  

\begin{argumentN}[1]

 \aitem \sqA{\Gamma}{\forall x(Wx\limplies Mx)}{premise}
 \aitem \sqA{\Gamma}{We\limplies Me}{1,\uniE}
 \aitem \sqA{\Delta}{We}{premise}
 \aitem \sqA{\Gamma,\Delta}{Me}{2,3,\condE}
 \aitem \sqA{\Gamma,\Delta}{ \lthereis x Mx}{4,\exI}

\end{argumentN}

If \p{\Gamma,\Delta} prove that Ed is a whale, then \p{\Gamma,\Delta} prove that 
there is a whale. 

Notice that \exI{} allows the following inference:

\begin{argument*}

 \ai{\Gamma}{Ldd}{premise}
 \ai{\Gamma}{\lthereis x Lxd}{1,\exI}

\end{argument*}

If \p{Ldd} means that Donald loves Donald, then it follows that there is someone 
(i.e., Donald) who loves Donald.


For the logical connectives of \lL{} we have a pair of rules for each of them.  
The same is true for the quantifiers.  What we have so far is one half of the 
inference rules for quantifiers. Let's discuss the other two in the following 
sections.






\section{Universal Quantifier Introduction}


Take any constant \p{c} and a predicate \p{F}. We can easily prove \p{\seq{}
{Fc\lor\lnot Fc}}. All we have to do is take our earlier proof of EM and replace 
\p{p} with \p{Fc}:


\begin{argument*}

\ai[0.25]{\lnot (Fc \lor \lnot Fc)}{\lnot (Fc \lor \lnot Fc)}{A}

\ai[0.25]{Fc}{Fc}{A}

\ai[0.25]{Fc}{Fc \lor \lnot Fc}{2,\disjI}

\ai[0.25]{Fc,\lnot (Fc \lor \lnot Fc)}{Fc \lor \lnot Fc}{3}

\ai[0.25]{\lnot (Fc \lor \lnot Fc),Fc}{\lnot (Fc \lor \lnot Fc)}{1}

\ai[0.25]{\lnot (Fc \lor \lnot Fc)}{\lnot Fc}{4,5,\negI}

\ai[0.25]{\lnot (Fc \lor \lnot Fc)}{Fc \lor \lnot Fc}{6,\disjI}

\ai[0.25]{}{\lnot \lnot (Fc \lor \lnot Fc)}{1,7,\negI}

\ai[0.25]{}{Fc \lor \lnot Fc}{8,\negE}

\end{argument*}

Now, when you inspect the above argument, it is easy to see that we could run 
the same argument with any other constant. So we should be allowed to conclude 
\p{\seq{}{\lforall x (Fx\lor\lnot Fx)}}.


What we just did is universalize from a particular case. That is usually a very 
bad idea (e.g., ``Donald is narcissist, therefore everyone is a narcissist'' is 
an obvious \emph{non sequitur}\footnote{`non sequitur' is Latin for `it does not 
follow.'}).  So why does the above work?  It is crucial that there is 
\emph{nothing} special about \p{c} in the proof of \p{\seq{}{Fc\lor \lnot Fc}}.  
What does that mean?  Different things differ from each other in various ways.  
But they also have things in common.  To say that there was nothing special 
about \p{c} in the above proof is to say that nothing that makes \p{c} different 
from other things played a role in the proof.  This is why we can be confident 
that the universalization to all \p{x} works.

So the thought is that if we can show that \p{\phi(c)} is true without depending 
on anything that distinguishes \p{c} from other things, then we can universalize 
to the claim that \p{\lforall\upsilon\phi(\upsilon)}. Let's put this as an 
inference rule:

\begin{infrule}

 \item[Universal Quantifier Introduction (\p{\lforall}I)] Given a constant 
  \p{\kappa}, from \p{\seq{\Lambda}
   {\phi(\kappa)}} infer \mbox{\p{\seq{\Lambda}{\lforall\upsilon
  \phi(\upsilon)}}}, provided \p{\kappa} does not appear in any of the sentences 
  in \p{\Lambda}. (Must replace all instances of \p{\kappa} in \p{\phi(\kappa)}  
  with \p{\upsilon}.) 

 \end{infrule}


The proviso ensures that no claims about \p{\kappa} are needed to support 
\p{\phi(\kappa)} which means that nothing that might distinguish \p{\kappa} from 
other things plays a role in the support for \p{\phi(\kappa)}. 

We can now formalize the universalization of a tautology:


\begin{argument*}

\ai{}{Fc \lor \lnot Fc}{EM}

\ai{}{\plshpn{UxAFxNFx}}{1,\uniI}

\end{argument*}

So there are theorems involving quantifiers, too.


Notice that this introduces a sequent whose succedent is \p{Fc \lor \lnot Fc} 
without first declaring that \p{c} is a constant.  If \p{c} is understood as a 
variable, this would not be a sentence and the sequent makes no sense. We will 
be charitable and interpret everything in such a way that any unbound term is 
understood as a constant with the exception that a term that is previously used 
as a variable somewhere in the derivation  cannot be understood as a constant.  
And once a letter is used for a constant, you cannot use it as a name for a 
variable later.
For example, take

\begin{argument*}

 \ai{\Gamma}{\lforall x Fx}{premise}
 
 \ai{\Gamma}{Fy}{1, \uniE}

\end{argument*}

In this case, \p{y} on line 2 is understood as a constant. Because \p{y} does 
not appear as a variable previously. But the following is \emph{not} ok:

\begin{argument*}

 \ai{\Gamma}{\lforall y Fy}{premise}
 
 \ai{\Gamma}{Fy}{1, \uniE}

\end{argument*}

\p{y} is used as a variable on line 1, and that means that \p{y} must be 
understood as an unbound variable on line 2 which renders the sequent there 
senseless. Similarly, you cannot do

\begin{argument*}

 \ai{\Gamma}{Fx}{premise}

 \ai{\Gamma}{\lthereis x Fx}{1, \exI}

\end{argument*}

\p{x} on the first line must be understood as a constant, but then you cannot 
use it as a variable name on line 2.

Finally, we will adopt the convention that when a constant appears for the first 
time in a derivation, it is a \emph{new} constant that is being added to \lL[Q] 
unless it is explicitly specified otherwise in the context.
Remember our discussion of the semantics of quantified sentences: it is possible 
to add new constants without affecting what is true without that new constant.  
All sentences that appear prior to the introduction of the constant should be 
understood as sentences in the language without the new constant. This makes 
life a lot easier when it comes to applying \uniI.

Here is an example: all whales are mammals, and all mammals are warm-blooded; it 
follows that all whales are warm-blooded. We can formalize this using the same 
interpretation of the predicate letters as as in the previous section:

\begin{argumentN}[1]
%generated by gentzen

\ai{\Gamma}{\lforall x (Wx\limplies Mx)}{premise}

\ai{\Delta}{\lforall x (Mx\limplies Bx)}{premise}

\ai{Wn}{Wn}{A}

\ai{\Gamma}{Wn\limplies Mn}{1,\uniE}

\ai{\Delta}{Mn\limplies Bn}{2,\uniE}

\ai{\Gamma, Wn}{Mn}{3,4,\condE}

\ai{\Gamma, \Delta, Wn}{Bn}{5,6,\condE}

\ai{\Gamma, \Delta}{Wn\limplies Bn}{7,\condI}

\ai{\Gamma, \Delta}{\lforall x (Wx\limplies Bx)}{8,\uniI}

\end{argumentN}



Here is an informal version of what this does: all whales are mammals, and all 
mammals are warm-blooded. Pick anything that's a whale. Let's call it Nina.  
Since Nina is a whale, Nina is a mammal, and therefore warm-blooded. Nothing 
depended on our choice of Nina. So all whales are warm-blooded.

In the derivation, we introduce a new constant \p{n} on line 3 with the 
assumption \p{\seq{Wn}{Wn}}. We can tell it's a constant since if it were a 
variable the sequent would not make sense, and \p{n} does not occur in the 
previous lines as a variable. And it's a \emph{new} constant because \p{n} 
appears there for the first time. Because \p{n} is a new constant, it cannot 
appear in \p{\Gamma,\Delta}: those must be sets of sentences of the original 
language because they are introduced into the derivation before the new constant 
is introduced.  So we know the proviso for \uniI{} is met for the inference to 
line 9.

There is one more rule of inference. It is, you guessed it, the Existential 
Quantifier Elimination rule. We will discuss that in the next section. 


\section{Existential Quantifier Elimination}


Here is something obvious: if all whales are mammals, and there are whales, it 
follows that there are mammals. This means that we should be able to infer from 
\p{\seq{\Gamma}{\lforall x}{(Wx \limplies Mx)}} and \p{\seq{\Delta}
{\lthereis x Wx}} to \p{\seq{\Gamma,\Delta}{\lthereis x Mx}}. 


The inference rules we have so far do not enable us to make the inference, 
though. So we need a new inference rule. What should that look like? Here is a 
thought.  Consider the following argument:

\begin{Example}

All whales are mammals, and there are whales.  Let's suppose Cindy is one of the 
whales. On this supposition, Cindy is a mammal because of the first premise.  So, 
on the supposition that Cindy is a whale, there are mammals.  Now, we don't know 
that Cindy is a whale. But there are whales and that means that what we did is 
like picking a whale and naming it Cindy. So whether or not the real Cindy is a 
whale, there are mammals. 

\end{Example}

We will call the reasoning pattern exhibited here Existential Quantifier 
Elimination. Given a general existential claim, we pretend that the claim is 
true of a particular thing, \p{\kappa}, show that something follows on that 
pretense, and conclude that it follows even without the pretense.

We do need to be careful. Consider:

\begin{Example}

 There are scientist. So let's suppose Donald is a scientist. Donald believes 
 that injecting bleach can cure infectious diseases. So on the supposition that 
 Donald is a scientist, there are scientists who believe that injecting bleach 
 can cure infectious diseases.   Therefore, there are scientists who believe 
 that injecting bleach can cure infectious diseases.

\end{Example}

This is a terrible piece of reasoning. How does it differ from the Cindy case?  
The problem is that believing that injecting bleach can cure infectious diseases 
is a special fact about Donald. Had you picked somebody else, say Deborah, you 
wouldn't have been able to say that Deborah believes that injecting bleach can 
cure infectious diseases. In the Cindy case, the reasoning does not depend on 
anything special about Cindy. That's crucial.

Here is a rule of inference that captures the Cindy case while ruling out the 
Donald case:

\begin{infrule}

 \item[Existential Quantifier Elimination (\exE)] From \p{\seq{\Lambda_1}
   {\lthereis\upsilon\phi(\upsilon)}} and \p{\seq{\Lambda_2,\phi(\kappa)}
  {\psi}}, infer \p{\seq{\Lambda_1,\Lambda_2}{\psi}}, provided \p{\kappa} does not 
  appear in any of \p{\Lambda_1}, \p{\Lambda_2}, and \p{\psi}.

 \end{infrule}

A note on the  proviso that \p{\kappa} must not appear in any of   \p{\Lambda_1}, 
\p{\Lambda_2}, and \p{\psi}. Requiring that \p{\kappa} not appear in \p{\Lambda_1} or 
\p{\Lambda_2} ensures we do not illicitly depend on some special features of 
\p{\kappa}.  By requiring that \p{\kappa} not appear in \p{\psi}, we prevent 
concluding something about \p{\kappa} even though \p{\Lambda_1} and \p{\Lambda_2} say 
nothing about \p{\kappa}.


We can now formalize the argument in the above Example involving Cindy:


%Derive from \p{\seq{\Gamma}{\lforall x (Wx\limplies Mx)}} to \p{\seq{\Gamma, \Delta}{\lthereis z (Mz)}}

\begin{argumentN}[1]
%generated by gentzen

\ai{\Gamma}{\lforall x (Wx\limplies Mx)}{premise}

\ai{\Delta}{\lthereis y Wy}{premise}

\ai{Wc}{Wc}{A}

\ai{\Gamma}{Wc\limplies Mc}{1,\uniE}

\ai{\Gamma, Wc}{Mc}{3,4,\condE}

\ai{\Gamma, Wc}{\lthereis z Mz}{5,\exI}

\ai{\Gamma, \Delta}{\lthereis z Mz}{2,6,\exE}

\end{argumentN}



Here is another example of \exE{} in action. Suppose there is something that is 
F. It follows that not everything is not-F. E.g., if there are reasonable people, 
not everyone is unreasonable. That means we should be able to infer from 
\p{\seq{\Gamma}{\lthereis x Fx}} to \p{\seq{\Gamma}{\lnot\lforall x\lnot Fx}}.  
Here is one way: Assume that something is F, and also assume that everything is 
not-F. Let Alec be F. But given the second assumption, Alec is not-F. But then 
Alec is both F and not-F. That's a contradiction. Something has to give. Since 
we are going to keep the first assumption, we must give up the second 
assumption.  Thus, not everything is not-F. 

We can formalize this:

%Derive from \p{\seq{\Gamma}{\lthereis x (Fx)}} to \p{\seq{\Gamma}{\lnot \lforall x (\lnot Fx)}}

\begin{argumentN}[1]
%generated by gentzen

\ai{\Gamma}{\lthereis xFx}{premise}

\ai{\lforall x \lnot Fx}{\lforall x \lnot Fx}{A}

\ai{Fa}{Fa}{A}

\ai{\lforall x \lnot Fx}{\lnot Fa}{2,\uniE}

\ai{Fa, \lforall x \lnot Fx}{\lnot Fa}{4}

\ai{Fa, \lforall x \lnot Fx}{Fa}{3}

\ai{Fa}{\lnot \lforall x \lnot Fx}{5,6,\negI}

\ai{\Gamma}{\lnot \lforall x \lnot Fx}{1,7,\exE}

\end{argumentN}






\section{Proof System for Predicate Logic}

Here is our proof system for predicate logic in one place. We now use Greek 
lower case letters for sentence variables. You may: 

\begin{infrule}

 \item[Assumption Introduction (A)] Infer   \p{\seq{\phi}
  {\phi}}.

 \item[Conjunction Introduction (\conjI)] From \p{\seq{\Lambda_1}{\phi_1}} and 
  \p{\seq{\Lambda_2}{\phi_2}}, infer \p{\seq{\Lambda_1, \Lambda_2}{\phi_1 \land \phi_2}}.


 \item[Conjunction Elimination (\conjE)] From \p{\seq{\Lambda}{\phi_1 \land \phi_2}}, 
  infer \p{\seq{\Lambda}{\phi_1}} as well as \p{\seq{\Lambda}{\phi_2}}.

 \item[Disjunction Introduction (\disjI)] From \p{\seq{\Lambda}{\phi_1}}, infer 
  \p{\seq{\Lambda}{\phi_1 \lor \phi_2}} as well as \p{\seq{\Lambda}{\phi_2 \lor \phi_1}}, for 
  any \p{\phi_2}.

 \item[Disjunction Elimination (\disjE)] From \p{\seq{\Lambda_1}{\phi_1\lor \phi_2}} and
  \p{\seq{\phi_1, \Lambda_2}{\phi_3}} and \p{\seq{\phi_2, \Lambda_3}{\phi_3}}, infer 
  \p{\seq{\Lambda_1, \Lambda_2, \Lambda_3}{\phi_3}}.

 \item[Negation Introduction (\negI)] From \p{\seq{\Lambda_1, \phi_1}
  {\phi_2}} and \p{\seq{\Lambda_2, \phi_1}{\lnot \phi_2}}, infer \p{\seq{\Lambda_1, \Lambda_2}
 {\lnot \phi_1}}.

 \item[Negation Elimination (\negE)] From \p{\seq{\Lambda}{\lnot\lnot \phi}}, 
  infer \p{\seq{\Lambda}{\phi}}.

 \item[Conditional Introduction (\condI)] From \p{\seq{\Lambda, \phi_1}
  {\phi_2}}, infer \p{\seq{\Lambda}{\phi_1 \limplies \phi_2}}.
 
 \item[Conditional Elimination (\condE)] From \p{\seq{\Lambda_1}
  {\phi_1\limplies \phi_2}} and \p{\seq{\Lambda_2}{\phi_1}}, infer \p{\seq{\Lambda_1, \Lambda_2}
 {\phi_2}}.


 \item[Universal Quantifier Introduction (\p{\lforall}I)] Given a constant 
  \p{\kappa}, from \p{\seq{\Lambda}
   {\phi(\kappa)}} infer \mbox{\p{\seq{\Lambda}{\lforall\upsilon
  \phi(\upsilon)}}}, provided \p{\kappa} does not appear in any of the sentences 
  in \p{\Lambda}. (Must replace all instances of \p{\kappa} in \p{\phi(\kappa)
  } with \p{\upsilon}.) 

 \item[Universal Quantifier Elimination (\p{\lforall}E)] From \p{\seq{\Lambda}
  {\lforall\upsilon\phi(\upsilon)}}, infer \p{\seq{\Lambda}{\phi(\kappa)}}, for 
  any constant \p{\kappa}. (All instances of \p{\upsilon} must be replaced by 
  \p{\kappa}.)
 
 \item[Existential Quantifier Elimination (\exE)] From \p{\seq{\Lambda_1}
   {\lthereis\upsilon\phi(\upsilon)}} and \p{\seq{\Lambda_2,\phi(\kappa)}
  {\psi}}, infer \p{\seq{\Lambda_1,\Lambda_2}{\psi}}, provided \p{\kappa} does not 
  appear in any of \p{\Lambda_1}, \p{\Lambda_2}, and \p{\psi}.

 \item[Existential Quantifier Introduction (\p{\lthereis}I)] Given a constant 
  \p{\kappa}, infer from \p{\seq{\Lambda}
 {\phi(\kappa)}} to \p{\seq{\Lambda}\lthereis\upsilon\phi(\upsilon)}. (At least 
 one---but not necessarily all---instances of \p{\kappa} must be replaced by 
 \p{\upsilon}.)

\end{infrule}


Also, you may rewrite the datum side of sequents in the following ways:

\begin{enumerate}
 \renewcommand{\labelenumi}{\alph{enumi}.}
 
 \item You may reorder items within the datum as you see fit.

 \item You may delete duplicate items within the datum.

 \item You may add arbitrary items to the datum of a sequent.

\end{enumerate}

Finally, you may introduce new constants into a derivation using Assumption 
Introduction or Universal Quantifier Elimination. The new constant must not 
appear on any previous line nor be declared a constant in the surrounding 
context.



\section{Theorems}\label{sec:theoremsPL}

As we have already seen, predicate logic has theorems, too.  A proof proves a 
theorem if it proves a sequent with an empty datum by using only the rules of 
our proof system. It is exactly like what it was in the case of sentential 
logic. And we can prove theorems in just the same sort of ways as before. E.g., 

%Prove \p{\seq{}{\lthereis x (Fx)\limplies \lnot \lforall x (\lnot Fx)}}

\begin{argumentN}[1]
%generated by gentzen

\ai{\lthereis x Fx}{\lthereis x Fx}{A}

\ai{\lforall x \lnot Fx}{\lforall x \lnot Fx}{A}

\ai{Fa}{Fa}{A}

\ai{Fa, \lforall x \lnot Fx}{Fa}{3}

\ai{\lforall x \lnot Fx}{\lnot Fa}{2,\uniE}

\ai{Fa}{\lnot \lforall x \lnot Fx}{4,5,\negI}

\ai{\lthereis x Fx}{\lnot \lforall x \lnot Fx}{1,6,\exE}

\ai{}{\lthereis x Fx\limplies \lnot \lforall x \lnot Fx}{7,\condI}

\end{argumentN}




\subsection{More Theorems}

Here are some useful theorems:



\begin{theorems}

 \thrm[Quantifier Exchange (QE)]{ \plshpn{CXxFxNUxNFx}}

 \thrm[Quantifier Exchange (QE)]{ \plshpn{CNUxNFxXxFx}}

 \thrm[Quantifier Exchange (QE)]{ \plshpn{CUxFxNXxNFx}}

 \thrm[Quantifier Exchange (QE)]{ \plshpn{CNXxNFxUxFx}}

 \thrm[Quantifier Exchange (QE)]{ \plshpn{CXxNFxNUxFx}}
 
 \thrm[Quantifier Exchange (QE)]{ \plshpn{CNUxFxXxNFx}}

 \thrm[Quantifier Exchange (QE)]{ \plshpn{CUxNFxNXxFx}}

 \thrm[Quantifier Exchange (QE)]{ \plshpn{CNXxFxUxNFx}}

 \thrm[Confinement (CF)]{\plshpn{CKUxFxUxGxUxKFxGx}}

 \thrm[Confinement (CF)]{\plshpn{CUxKFxGxKUxFxUxGx}}

 \thrm[Confinement (CF)]{\plshpn{CAUxFxUxGxUxAFxGx}}


\thrm[Confinement (CF)]{\plshpn{CKXxFxXxGxXxXyKFxGy}}

\thrm[Confinement (CF)]{\plshpn{CXxXyKFxGyKXxFxXxGx}}

\thrm[Confinement (CF)]{\plshpn{CAXxFxXxGxXxAFxGx}}

\thrm[Confinement (CF)]{\plshpn{CXxAFxGxAXxFxXxGx}}

\end{theorems}



\section{Generalizing the Theorems}

Take a look at the theorem we proved in the last section. The proof takes a 
particular predicate \p{F} but it is obvious that you could replace it with any 
other predicate and construct a parallel proof. So just like the proofs we saw 
of theorems of sentential logic, the proof of that Theorem (and the other 
theorems) is a proof template. You can replace any 1-place predicate with any 
other 1-place  predicate. 

You will notice that the sentences involved in that proof of the last section 
are very simple.  For instance, \plshp{XxFx} has an existential quantifier 
attached to the simplest formula possible. But quantifiers can also be attached 
to complex formulas. We can easily generalize the proof in the previous section 
to cover cases in which the existential quantifier is attached to a complex 
formula by replacing \p{Fx} with \p{\phi(x)} and \p{Fa} with \p{\phi(a)} (\p{a} 
is a constant). \p{\phi(x)} is a formula in which \p{x} is free and \p{\phi(a)} 
is the formula obtained by substituting \p{a} into \p{x}:

%Prove \p{\seq{}{\lthereis x φ(x)\limplies \lnot \lforall x [\lnot φ(x)]}}

\begin{argumentN}[1]
%generated by gentzen

\ai{\lthereis x φ(x)}{\lthereis x φ(x)}{A}

\ai{\lforall x \lnot φ(x)}{\lforall x \lnot φ(x)}{A}

\ai{φ(a)}{φ(a)}{A}

\ai{φ(a), \lforall x φ(x)}{φ(a)}{3}

\ai{\lforall x \lnot φ(x)}{\lnot φ(a)}{2,\uniE}

\ai{φ(a)}{\lnot \lforall x \lnot φ(x)}{4,5,\negI}

\ai{\lthereis x φ(x)}{\lnot \lforall x \lnot φ(x)}{1,6,\exE}

\ai{}{\lthereis x φ(x)\limplies \lnot \lforall x \lnot φ(x)}{7,\condI}

\end{argumentN}



The proofs for the other theorems can be generalized in the same fashion. For 
the rest of the course, you may assume that we have proven the theorems in this 
generalized form.


\section{Soundness of the System of Predicate Logic}

Given the explanation of the inference rules, it is plausible that our proof 
system is sound.  We will use the same terminology as in Section
\ref{sec:soundnessSL}: \p{ \Gamma \lproves \phi} is correct iff. \p{ \Gamma 
\lentails \phi}.  What we want to show is that if a derivation consists of only 
correct sequents, then extending the proof by one more line via one of our four 
new rules will result in another correct sequent. 


\subsection{Existential Quantifier Introduction}

This is easy to see given the semantics of the existential quantifier. Any 
interpretation in which \p{\phi(c)} is true is also such that \p{\lthereis x 
\phi(x)} is true.

\subsection{Universal Quantifier Elimination}

This, too is easy to see given the semantics of the universal quantifier.

\subsection{Universal Quantifier Introduction}

Suppose \p{\Gamma\lentails \phi(c)} where \p{c} does not appear in any of the 
sentences of \p{\Gamma}. Let \p{\mathfrak I} be an interpretation in which all 
the sentences in \p{\Gamma} are true. We are supposing that \p{\phi(c)} is also 
true in \p{\mathfrak I}. Consider an interpretation \p{\mathfrak I'} which is 
just like \p{\mathfrak I} except that it assigns a different referent to \p{c}.  
All the sentences in \p{\Gamma} must still be true in \p{\mathfrak I'} because 
\p{c} does not appear in any sentence in \p{\Gamma} so that the referent of \p{c} 
makes no difference to the truth of those sentences. So \p{\phi(c)} must also be 
true in \p{\mathfrak I'} because \p{\Gamma \lentails \phi(c)}. But this means 
that any interpretation that makes all sentences in \p{\Gamma} true also make 
\p{\phi(c)} true independently of the referent of \p{c}. Thus, if \p{\Gamma 
\lentails \phi(c)}, then \p{\Gamma \lentails \lforall x \phi(x)} provided \p{c} 
does not appear in any sentence in \p{\Gamma}. This shows that \uniI{} is a 
valid rule of inference.


\subsection{Existential Quantifier Elimination}

Suppose we can prove \p{\seq{\Delta, \phi(c)}{\psi}} where \p{c} does not appear 
in \p{\Delta} or \p{\psi} and that the sequent is correct. That is, \p{\Delta, 
\phi(c) \lentails \psi}. The point to notice is that this is a claim about 
truth.  What it says is that if \p{\phi(c)} is true, then \p{\phi(\kappa)} is 
true for any constant \p{\kappa} that refers to the same thing as \p{c}. I.e., 
if an interpretation is such that the sentences in \p{\Delta} are true and there 
is something that is \p{\phi} (if \p{\kappa} refers to it,  then \p{\phi(\kappa)} 
is true), then \p{\psi} is also true in that interpretation.

With this in mind, suppose we also have \p{\Gamma \lentails \lthereis x \phi(x)
}, and consider an interpretation \p{\mathfrak I} that makes all sentences in 
\p{\Gamma} and \p{\Delta} true. Given the semantics of the existential 
quantifier, \p{\mathfrak I} is such that there is something in the domain that 
is \p{\phi}. Thus, \p{\mathfrak I} also makes \p{\psi} true: \p{\Gamma,
\Delta \lentails \psi}. This shows that \exE{} is a valid rule of inference.


\subsection{Completeness}


What about completeness? Can our system prove all truths of logic?

As stated, our system is not complete. Here is a truth of logic that cannot be 
proved in our system:

\begin{center}

 \p{Fc \limplies \lforall x Fc}

\end{center}

Notice that the universal quantifier does not bind anything in its scope. We 
have been discouraging such formulas but it is allowed. Given the semantics of 
the universal quantifier, \p{\lforall x Fc} is equivalent to \p{Fc} so the above 
is indeed a truth of logic.

We can modify Universal Quantifier Introduction a bit to allow inferring from 
\p{\seq{\Lambda}{\phi}} to \p{\seq{\Lambda}{\lforall \upsilon \phi}} (notice 
that \p{\phi} is unmodified).  This modified system is complete but we will not 
be able to prove it here.  I have to ask you to take my word for it.
 
If you are interested, Kurt \citet{Godel1929} was the first person to prove the 
completeness of predicate logic. The proof was his doctoral dissertation.  
Nowadays the preferred proof of completeness is one that was given by Leon 
\citet{Henkin1949}.  




\section{Counting and Identity}\label{secIdentity}

The sentence \p{\lthereis x Fx} tells us that there is at least one thing that  
is \p{F}. But it does not tell us how many. Maybe there is only one, but maybe 
there are gazillions. \p{\lthereis xFx} does not distinguish between those 
cases. How could we say there is exactly one thing which is \p{F}?


Let's start with an easier question. How could we say that there are at least 
two things that are \p{F}? To say that there are two things that are \p{F} is to 
say that there are two distinct things that are \p{F}. We might try:


\begin{argument}
 \aitem \plshp{KXxFxXyFy}
\end{argument}

There is an \p{x} which is \p{F}, and  there is a \p{y} which is \p{F}. But this 
does not work. \p{\lthereis x Fx} is true just in case there is at least one 
thing that is \p{F}. And the same goes for \p{\lthereis y Fy}. They are two ways 
of saying the same thing. So even if there is only one thing that is \p{F}, 1 
would be true. We need to ensure that \p{x} and \p{y} are different things.  Let 
us write \p{\nident{x}{y}} to mean that \p{x} is not identical to \p{y}.  And we 
will write \p{\ident{x}{y}} to mean that \p{x} is identical to \p{y}. That is, 
we will use the identity sign the way it is normally used. We could then try:
\begin{argument}
 \aitem \plshp{KKXxFxXyFyN=xy}

\end{argument}

But this still does not work because the formula \p{\nident{x}{y}} is outside 
the scope of either quantifier. What we want to say is something like: we can 
find an \p{x} which is \p{F}, and find a \p{y} which is \p{F}, and when we 
compare \p{x} and \p{y} we find that they are two distinct things. 2 doesn't say 
that.  To be able to compare the two things we find, we have to get them within 
the scope of the same quantifier. The following will do:

\begin{argument}
\aitem \plshp{XxKFxXyKFyN=xy}
\end{argument}

This says   that there is an \p{x} which is \p{F} and also a \p{y} which is \p{F}
but is different from \p{x}. That means there are at least two things that are  
\p{F}. It's the existential claim starting with \p{\lthereis y} inside the scope 
of the \p{\lthereis x} that guarantees the existence of a second thing that is 
\p{F}. 

As an example, suppose our domain contains only mom, dad, grandma, and Fido.  
Fido is a dog, the others are humans; dad is male, the other humans in the 
domain are women. As you can see, there are at least two women in our domain: 
mom and grandma. And if we let \p{Fx} mean `\p{x} is a woman' we can see that 
the above sentence is true given our domain of discourse. There is someone, e.g.  
mom, who is a woman and there is someone else, e.g., grandma, who is a woman but 
is not mom.

Let's get back to the original question, How do we say that there is exactly one 
thing which is \p{F}? Consider Fido.  Fido is the only dog in our domain so 
there is exactly one dog in the domain.  Since we now know how we could say that 
there are at least two dogs in the domain, we can also say that there is exactly 
one dog: there are dogs, but it is not true that there are at least two dogs.  
And that means there is a thing which is a dog but there is no second thing that 
is a dog. More generally, there is exactly one \p{F} is to say that there is at 
least one things that is \p{F} but there is no second thing that is  \p{F}.  We 
can say this using the following sentence:
\begin{argument}
\aitem   \plshp{XxKFxNXyKFyN=xy}

\end{argument}

Notice the negation sign in front of the \p{\lthereis y}. That rules out that 
there is a second thing that is \p{F}.  Alternatively, we could also say:

\begin{argument}
 
 \aitem  \p{\lthereis x\big[Fx \land\lforall y(Fy \limplies\ident{x}{y})
 \big]}

\end{argument}


Continuing with our example domain of discourse, we can see that there are 
exactly two women: mom and grandma. How could we say that in our formal language?  
That is not hard. We start with the sentence that says there are at least two 
and add something that rules out the existence of a third. Like this:
\begin{argument}

\aitem \plshp{XxKFxXyKKFyN=xyNXzKKFzN=xzN=yz}

\end{argument}

There is an \p{x} which is \p{F}, another thing \p{y} which is also \p{F}, but 
there is no further thing distinct from both \p{x} and \p{y} that is also \p{F}.

We can say that there are at least three by removing the negation sign in the 
sentence above:

\begin{argument}

\aitem \plshp{XxKFxXyKKFyN=xyXzKKFzN=xzN=yz}

\end{argument}

We can then say that there are exactly three by adding a clause that rules out 
the existence of any further thing that is \p{F}. You get the drift. We could 
continue that forever. So for every natural number \emph{n}, we can say that 
there are exactly \emph{n} things that are \p{F} as well as that there are at 
least \emph{n} things that are \p{F}. Of course, we will run out of letters of 
the Roman alphabet quickly, so will have to resort to subscripts. The genius of 
the Arabic numerals guarantees we will never run out of them. 




\section{Axioms}\label{sec:axioms}

When I introduced the identity sign in the previous section, I cheated a bit by 
depending on your previous understanding of the \p{=} symbol. Let's a look at it 
more closely.

A formula like \p{\ident{x}{y}} expresses a relation between \p{x} and \p{y}.  
So the identity sign is just a 2-place predicate. We could have kept our 
notation more consistent by writing \p{\setlength{\thickmuskip}{0.5mu}=xy} but 
that looks very unfamiliar, so we (like everybody else) will use the more 
familiar style.  Since \p{=} is just a 2-place predicate, if we treat it like 
any other 2-place predicate, there is no reason why it should mean identity.  In 
declaring that the symbol is to be used in familiar ways, I  imposed a certain 
interpretation of it. In other words, I restricted the permissible interpretations of our 
language to those that take \p{=} to mean identity.  What are the features of a 
interpretation that take \p{=} to mean identity?

We can specify the requirements in terms of the sentences that must come out to 
be true. Any interpretation that takes \p{=} to mean identity must make all sentences 
that fit the following templates true:
\begin{argument}

 \aitem \plshp{=xx}

 \aitem \plshp{CK=xy/fx/fy}

\end{argument}

That is, everything is identical with itself, and if \p{x} and \p{y} are 
identical, whatever is true of \p{x} is true of \p{y} (the latter is known as 
the indiscernibility of identicals, or Leibniz's Law).

Sentences that are declared to be true are called \emph{axioms}. So the above 
templates give us an infinite number of axioms. We want to make sure that our 
system can make use of these axioms. We can accomplish that by adding the 
following inference rule to our system:

\begin{infrule}
\item[Axiom] If \p{\alpha} is an axiom, infer \p{\seq{}{\alpha}}.
\end{infrule}


Notice that the sequents that this rule allows us to infer have empty datums.  
However, the succedents of these sequents are not truths of logic. For instance, 
we can infer \p{\seq{}{\ident{\kappa}{\kappa}}} but \p{\ident{\kappa}{\kappa}} 
is true only under a particular interpretation of the \p{=} sign. 

The characterization of an axiom as a sentence declared to be true without proof 
may make it sound like an assumption. But the role of an axiom is importantly 
different from an assumption.  In our proof system, an assumption is introduced 
as \p{\seq{s}{s}} which is a correct sequent under any and all interpretations.  
So in introducing an assumption, we say nothing about which interpretation we 
want to use. An axiom is different. Making an axiom true---or, equivalently, 
ensuring that Axiom does not lead us astray by enabling us to infer a 
falsehood---requires that we use some interpretations but not others as can be 
seen for the need to interpret \p{=} as identity given the axioms governing
\p{=}. In the case identity, it is trivial to ensure \p{=} means identity no 
matter what the domain of discourse is: the extension of \p{=} is simply the set 
of all ordered pairs \p{<\mathfrak m, \mathfrak m>} where \p{\mathfrak m} is a 
member of the domain.  Because of this, the axioms governing \p{=} are typically 
added to the proof system to yield what is known as predicate logic \emph{with 
identity}.

But we can also add axioms that impose much more serious constraints in the 
permissible interpretations.  Here is an intuitive example: we might declare 
that the following as axioms:

\begin{enumerate}
 \item \plshp{UxXyKTyxN=xy}
 \item \plshp{UxUyUzCKTxyTyzTxz}
 \item \plshp{UxUyCTxyNTyx}
 \end{enumerate}
 
Tthese three combined require an infinite domain of discourse---e.g., if \p{Tab} 
means that \p{a} is taller than \p{b}, then those axioms require the existence 
of ever taller items in the domain. 

Axioms like the ones just discussed play an important role in mathematics.  Such 
axioms are used to define the subject matter by imposing requirements that can 
only be met by some domains but not others.  For instance,
Euclidean geometry (the kind you learned pre-college) is built on five axioms 
that together determine the kinds of geometric figures that Euclidean geometry 
deals with. Given those axioms, Euclidean geometry is not talking about figures 
drawn on the surface of a sphere. If you change an axiom you get a different 
subject matter. In particular, changing the so-called parallel postulate results 
in various interesting non-Euclidean geometries.
An axiom therefore imposes restrictions on the interpretations in ways that 
assumptions cannot.

When formalizing arguments as derivations, I resisted treating ordinary premises 
as assumptions. Here is one way of thinking about premises. A premise takes the 
form \p{\seq{\Gamma}{p}} and, if \p{\Gamma} is non-empty, that's the same as 
\p{\seq{}{\widehat{\Gamma}\limplies p}} where \p{\widehat{\Gamma}} is the 
conjunction of all the sentences of \p{\Gamma}. So we can think of the 
introduction of a premise as the introduction of an axiom. Unlike an assumption, 
a premise makes a substantive claim about the world and we are to choose an 
interpretation for our language that makes that claim true.


















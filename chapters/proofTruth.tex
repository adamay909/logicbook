



\section{Sentential Logic and Its Theorems}\label{sec:theorems}

So far, the derivations we have seen had premises.  Let us move to derivations 
that have no premises. You might wonder how there could be derivations without 
premises. Don't we need some starting points and aren't they the premises? But 
in our proof system it's easy to construct a derivation without premises.  
Consider:

\begin{argument*}

\ai{p}{p}{A}

\ai{}{p \limplies p}{1,\condI}

\end{argument*}



There are two points to note about this derivation. First, it has no 
premises---just check the annotations. Secondly, the concluding sequent has an 
empty datum. Let me discuss them in order.

The derivation has no premises. But it does have an assumption.  A premise-less 
derivation is a derivation that has no premises.  But that does not mean it 
cannot have assumptions. The above is an example of that.  And notice that we do 
have an inference rule for getting to our assumptions: the Assumption 
Introduction rule.  So a premise-less derivation can be constructed using 
nothing more than our rules of inference.  The typical premises of the 
derivations in the previous sections require some work to get hold of---often 
that work is empirical research as well as some philosophical reflection.  But 
premise-less derivations require no appeals whatever to knowledge generated 
outside of our proof system.  Such derivations are of particular interest to the 
study of logic because it means that there are things we can figure out via 
logic independently of any other sources of knowledge. We will call our proof 
system combined with our characterization of a formal language \lL{} a 
\emph{system of sentential logic}, or just \emph{sentential logic} for short.  
In this chapter, we will be exploring important features of sentential logic.

We will call a premise-less derivation a \emph{proof}, and when we can derive a 
sequent through a proof, we say that we have proved the sequent. So a multi-line 
proof proves each sequent that appears in the proof. Since a derivation can only 
be of finite length, the same applies to proof. 

Now to the second point of interest. The concluding sequent, line 4, has an 
empty datum.  When we can prove a sequent that has an empty datum, we call the 
succedent of the sequent a \emph{theorem} of sentential logic. And in such cases, 
we will also say that we have proved the theorem. Notice that what's called the 
theorem is the succedent of the relevant sequent. That is, a theorem is a 
sentence of \lL{} and not a sequent.  What does such a theorem tell us? 

A sequent \p{\seq{\Gamma}
{s}} means that \p{\Gamma} conclusively supports \p{s}. So if we accept all the 
claims in \p{\Gamma}, we must accept the claim that \p{s}. In the case of a 
theorem \p{t}, we can prove \p{\seq{\Gamma}{t}} there \p{\Gamma} is empty. This 
means means that we must accept \p{s} regardless of whatever else we accept.  
That is, we must accept \p{s} regardless of what we believe and know about the 
world. 

Some theorems are more interesting than others.  The more interesting ones often 
have commonly used names. The above theorem tells us that it's a theorem that a 
sentence implies itself. It is known as \textbf{Identity (ID)}.



Conditional Introduction is an easy way of inferring to a theorem. The reason it 
works is that when you apply the inference rule, the number of items in the 
datum decreases by one. Conditional Introduction is not the only inference rule 
with this feature. Negation Introduction also has this feature. Recall:

\begin{description}
 \item[Negation Introduction (\negI)] From \p{\seq{\Gamma, s_1}
  {s_2}} and \p{\seq{\Delta, s_1}{\lnot s_2}}, infer \p{\seq{\Gamma, \Delta}
 {\lnot s_1}}.
\end{description}

We start with three items, \p{\Gamma}, \p{\Delta}, and \p{s_1} on the datum side 
and end with only two: \p{\Gamma}, \p{\Delta}. This means that \p{\negI} can 
also be used to prove a theorem. Here is an example:

\begin{argument*}

\ai{p \land \lnot p}{p \land \lnot p}{A}

\ai{p \land \lnot p}{p}{1,\conjE}

\ai{p \land \lnot p}{\lnot p}{1,\conjE}

\ai{}{\lnot (p \land \lnot p)}{2,3,\negI}

\end{argument*}



Given the contradiction \p{p \land \lnot p}, its negation is a theorem of 
sentential logic.  This is known as \textbf{Non-Contradiction (NC)}.

Here is another important theorem closely related to Non-Contradiction. 
Non-Contradiction says that we can prove that \p{p} is not both true and 
not-true. We can also prove that \p{p} is either true or not-true---that is, it 
cannot be that \p{p} is neither true nor false.  This is known as 
\textbf{Excluded Middle (EM)}. Here is a proof:
\newpage

\begin{argument*}

\ai{\lnot (p \lor \lnot p)}{\lnot (p \lor \lnot p)}{A}

\ai{p}{p}{A}

\ai{p}{p \lor \lnot p}{2,\disjI}

\ai{p,\lnot (p \lor \lnot p)}{p \lor \lnot p}{3}

\ai{\lnot (p \lor \lnot p),p}{\lnot (p \lor \lnot p)}{1}

\ai{\lnot (p \lor \lnot p)}{\lnot p}{4,5,\negI}

\ai{\lnot (p \lor \lnot p)}{p \lor \lnot p}{6,\disjI}

\ai{}{\lnot \lnot (p \lor \lnot p)}{1,7,\negI}

\ai{}{p \lor \lnot p}{8,\negE}

\end{argument*}




Notice that \p{p} in the  proofs of the theorems above is a sentential variable.  
So strictly speaking the proofs are templates of proofs. And a theorem is not a 
particular sentence of \lL{} but a sentence template.  You can plug any actual 
sentence of \lL{} into \p{p} and get an actual proof of a sentence. A theorem of 
sentential logic, therefore, tells us that any sentence of a particular form can 
be derived using the derivation rules of our proof system without needing any 
extra knowledge. From now on, we will mainly be discussing such proof 
templates---but we will call them proofs for brevity.

In order to avoid confusions, we will make the following stipulation: you are 
allowed to derive a sequent with an empty datum only if you can do so without 
depending on a premise. This ensures that if we can derive \p{\lproves s}, \p{s} 
is a theorem.

\section{Working with Proof Templates}


Given a proof template, you can replace a variable with any other variable and 
that will make no difference so long as you make sure that you replace 
consistently throughout.  For instance, take the first proof from the previous 
section:

\begin{argument}

 \aitem \sqA{p}{p}{A}
 \aitem \sqA{}{p\limplies p}{1,\condI}

\end{argument}

You can replace \p{p} with \p{s} throughout to get:

\begin{argument}

 \aitem \sqA{s}{s}{A}
 \aitem \sqA{}{s\limplies s}{3,\condI}

\end{argument}

You can also plug complex sentences into \p{p}. E.g., you could plug \p{\lnot A} 
into \p{p} and get:

\begin{argument}

 \aitem \sqA{\lnot A}{\lnot A}{A}
 \aitem \sqA{}{\lnot A\limplies \lnot A}{5,\condI}

\end{argument}

You can plug the negation of any sentence into \p{p} and get a fine proof.  That 
means you can replace \p{p} with \p{\lnot q} and get another fine proof template:

\begin{argument}

 \aitem \sqA{\lnot q}{\lnot q}{A}
 \aitem \sqA{}{\lnot q\limplies \lnot q}{7,\condI}

\end{argument}

In fact, you can replace \p{p} with any complex sentence and that means that you 
can replace \p{p} with any complex sentence template. E.g., the following is 
fine:
\begin{argument}

 \aitem \sqA{q\land s}{q\land s}{A}
 \aitem \sqA{}{(q\land s)\limplies (q\land s)}{9,\condI}

\end{argument}

Such systematic replacements of variables make no difference to the proof---just 
think about how the inference rules work.  So you may rewrite theorems by 
replacing sentence variables with other sentence variables or with sentence 
templates.  This will make life easier in what follows.

Be careful that you only make replacements and not engage in illicit 
derivations. For instance, you can do:

\begin{argument}
 \aitem \sqA{\lnot\lnot r}{\lnot\lnot r}{A}
 \aitem \sqA{}{\lnot\lnot r \limplies \lnot\lnot r}{11,\condI}
\end{argument}

But you may \emph{\textbf{not}} delete the double negations in line 12  unless 
you can prove that that is ok (it's not that difficult to prove, but you 
absolutely must prove it). 





\section{More Theorems of Sentential Logic}\label{sec:theoremList}

Here is a theorem known as \textbf{DeMorgan's Law (DM)} (there are a few other 
theorems known by the same name):

\begin{argument*}
%generated by gentzen

\ai{\lnot (p\lor q)}{\lnot (p\lor q)}{A}

\ai{\lnot (p\lor q), p}{\lnot (p\lor q)}{1}

\ai{p}{p}{A}

\ai{p}{p\lor q}{3,\disjI}

\ai{\lnot (p\lor q)}{\lnot p}{2,4,\negI}

\ai{\lnot (p\lor q), q}{\lnot (p\lor q)}{1}

\ai{q}{q}{A}

\ai{q}{p\lor q}{7,\disjI}

\ai{\lnot (p\lor q)}{\lnot q}{6,8,\negI}

\ai{\lnot (p\lor q)}{\lnot p\land \lnot q}{5,9,\conjI}

\ai{}{\lnot (p\lor q) \limplies (\lnot p\land \lnot q)}{10,\condI}

\end{argument*}




Theorems can be proven from assumptions only and that means we could insert the 
proof of a theorem into any other proof without disturbing it.  And that can be 
useful. In order to save space, time, and energy, you do not have to restate the 
proof of a theorem provided you have already proven it somewhere. We name 
theorems that we have proven, and in the annotation we just state the name of 
the theorem.  The proof of the following theorem, \textbf{Implication (IM)}
is an example:

%Prove \p{\seq{}{(p\limplies q)\limplies (\lnot p\lor q)}}

\begin{argumentN}[12]
%generated by gentzen

\ai{p\limplies q}{p\limplies q}{A}

\ai{\lnot (\lnot p\lor q)}{\lnot (\lnot p\lor q)}{A}

\ai{}{\lnot (\lnot p\lor q)\limplies (\lnot \lnot p\land \lnot q)}{DM}

\ai{\lnot (\lnot p\lor q)}{\lnot \lnot p\land \lnot q}{13,14,\condE}

\ai{\lnot (\lnot p\lor q)}{\lnot \lnot p}{15,\conjE}

\ai{\lnot (\lnot p\lor q)}{p}{16,\negE}

\ai{p\limplies q, \lnot (\lnot p\lor q)}{q}{12,17,\condE}

\ai{\lnot (\lnot p\lor q)}{\lnot q}{15,\conjE}

\ai{p\limplies q}{\lnot \lnot (\lnot p\lor q)}{18,19,\negI}

\ai{p\limplies q}{\lnot p\lor q}{20,\negE}

\ai{}{(p\limplies q)\limplies (\lnot p\lor q)}{21,\condI}

\end{argumentN}


On line 14 we just state the theorem without giving its proof---we could, but it 
would be a lot of work. The annotation tells us where to look for the proof.  
Note that we replaced \p{p} in our proof of DeMorgan with \p{\lnot p}.


Given Implication, we can give a shorter looking  proof of Excluded Middle which 
may be easier to see through given the definition of \p{\limplies}:
%Prove \p{\seq{}{\lnot p\lor p}}

\begin{argumentN}[23]
%generated by gentzen

\ai{}{p\limplies p}{ID}

\ai{}{(p\limplies p)\limplies (\lnot p\lor p)}{IM}

\ai{}{\lnot p\lor p}{23,24,\condE}

\end{argumentN}




This proof looks short because instead of giving an explicit proof of 
Implication, we merely state the theorem. Appealing to theorems proved elsewhere 
is a real time and space saver.

\subsection{Some useful theorems}

Here are some theorems that are often useful. Most of the names are commonly 
used ones. Notice that several related but distinct theorems have the same name 
(e.g., DeMorgan).

If you find yourself appealing to these theorems, use the name of the theorem in 
your annotations: 

\newpage


\begin{theorems}

 \thrm[Identity (ID)]{\plshn{Cpp}}

\thrm[ Non-Contradiction (NC)]{\plshn{NKpNp}}

\thrm[ Excluded Middle (EM)]{\plshn{ApNp}}

\thrm[ DeMorgan  (DM)]{\plshn{CNApqKNpNq}}

\thrm[ DeMorgan  (DM)]{\plshn{CKNpNqNApq}}

\thrm[DeMorgan  (DM)]{\plshn{CNKpqANpNq}}

\thrm[DeMorgan  (DM)]{\plshn{CANpNqNKpq}}

\thrm[ Implication (IM)]{\plshn{CCpqANpq}}

\thrm[ Elimination (EL)]{\plshn{CApqCNpq}}

\thrm[ Contraposition (CP)]{\plshn{CCpqCNqNp}}

\thrm[ Commutativity of Conjunction (CC)]{\plshn{CKpqKqp}}

\thrm[ Commutativity of Disjunction (CD)]{\plshn{CApqAqp}}

\thrm[ Associativity of Conjunction (AC)]{\plshn{CKKpqrKpKqr}}

\thrm[ Associativity of Conjunction (AC)]{\plshn{CKpKqrKKpqr}}

\thrm[ Associativity of Disjunction (AD)]{\plshn{CAApqrApAqr}}

\thrm[ Associativity of Disjunction (AD)]{\plshn{CApAqrAApqr}}

\thrm[ Double Negation Introduction (DN)]{\plshn{CpNNp}}

\end{theorems}


\section{Using Theorems in Derivations}

Theorems can be used to make proofs of other theorems easier. But there is 
nothing special about proofs of theorems in this regard. We can also use 
theorems in more general derivations that we discussed in the previous chapter.  
Appealing to theorems can make certain derivations much shorter and intuitive.  
Consider, for instance, the following formalization of an argument that in the 
plain English version uses Modus Tollens:

\begin{argumentN}[1]
%generated by gentzen

\ai{\Gamma}{P\limplies Q}{premise}

\ai{\Delta}{\lnot Q}{premise}

\ai{}{(P\limplies Q)\limplies (\lnot Q\limplies \lnot P)}{CP}

\ai{\Gamma}{\lnot Q\limplies \lnot P}{1,3,\condE}

\ai{\Gamma, \Delta}{\lnot P}{2,4,\condE}

\end{argumentN}


This is a little shorter than the derivation we gave in Section          
\ref{sec:negationRules} and it might feel like a more natural representation of 
how we reason using Modus Tollens. 

Our proof system is designed to use only a very small number of inference rules 
and that often results in surprisingly long formal derivations of matters that 
seem obvious. Appealing to theorems can make things a lot shorter and easier to 
see through by hiding some of the complexity. Here is another example, deriving 
from \p{\seq{\Gamma}
{P\lor Q}} and \p{\seq{\Delta}{\lnot P}} to \p{\seq{\Gamma,\Delta}{\lnot Q}}:

\begin{argumentN}[1]
%generated by gentzen

\ai{\Gamma}{P\lor Q}{premise}

\ai{\Delta}{\lnot P}{premise}

\ai{}{(P\lor Q)\limplies (\lnot P\limplies Q)}{EL}

\ai{\Gamma}{\lnot P\limplies Q}{1,3,\condE}

\ai{\Gamma, \Delta}{Q}{2,4,\condE}

\end{argumentN}


This should be much easier to comprehend than the derivation we saw earlier in 
one of our exercises. 




\section{Valid Arguments}

A very important idea in discussions of arguments is that of \emph{valid} 
arguments that we mentioned earlier. An argument is said to be valid iff. it is 
impossible for the conclusion to be false while all the premises are true. In 
other words, in a valid argument, the truth of all the premises guarantees the 
truth of the conclusion. Valid arguments are of great interest because given 
reasons to accept the premises of a valid argument, we will thereby have reason 
to accept its conclusion: you cannot rationally accept the premises while 
rejecting the conclusion of a valid argument---they leave no such wiggle room. 

Consider a very simple derivation:

\begin{argument*}
 
 \ai{\Gamma}{P\limplies Q}{premise}

 \ai{\Delta}{P}{premise}

 \ai{\Gamma,\Delta}{Q}{1,2,\condE}

\end{argument*}

Given our interpretation of a sequent, what this derivation tells us is that 
given conclusive support for \plsh{CPQ} and \p{P}, there is also conclusive 
support for \p{Q}. That is, the derivation tells us what would be true if an 
argument from \plsh{CPQ} and \p{P} to \p{Q} were valid. But would such an 
argument be valid? Of course, it would be.  We can see that by looking at the 
truth table of the conditional.  Just look at the first row (both the 
conditional and its antecedent are true) of:

\begin{center}

 
\begin{center}
\begin{tabular}{c c||c}
$s_1$  & $s_2$ & $(s_1 \limplies s_2)$\\
\hline
\emph{T} & \emph{T} & \emph{T} \\
\emph{F} & \emph{T} & \emph{T} \\
\emph{T} & \emph{F} & \emph{F}  \\
\emph{F} & \emph{F} & \emph{T} \\
\end{tabular}
\end{center}



\end{center}


Here is a question: given a successful derivation from \p{\seq{\Gamma_1}
{P_1}} through \p{\seq{\Gamma_n}{P_n}} to \p{\seq{\Gamma_1,\ldots,\Gamma_n}{Q}},
 is there a valid argument from \p{P_1}, \ldots, \p{P_n} to \p{Q}?  That is, 
 does the truth of all of \p{P_1} through \p{P_n} guarantee the truth of \p{Q} 
 given the existence of the derivation?

Here is a related question. Suppose the truth all of \p{P_1} through \p{P_n} 
guarantees the truth of \p{Q}. Is there a derivation from \p{\seq{\Gamma_1}
{P_1}} through \p{\seq{\Gamma_n}{P_n}} to \p{\seq{\Gamma_1,\ldots,\Gamma_n}{Q}}?

We want the answers to both of these questions to be Yes. Because in that case, 
we can say that: a) any argument that can be formalized as a derivation in our 
proof system is valid; and b) any valid argument can be captured by a derivation 
in our proof system. 

We will see in the next few sections that the answer to both these questions is 
indeed Yes.

Some preliminary exercises:

\begin{enumerate}

 \item Suppose there is a derivation from \p{\seq{\Gamma}{s}} to \p{\seq{\Gamma}
  {t}}.  Explain why it follows that \p{\seq{}{s\limplies t}}. (Hint: you can 
  plug anything you want into \p{\Gamma}.)

 \item Suppose \plsh{Cst} is a tautology. Explain why in that case the truth of 
  \p{s} guarantees the truth of \p{t}.

  \item Suppose we can show that if \p{\seq{}{s\limplies t}}, then \plsh{Cst} is 
   a tautology. Explain why this would show that if there is a derivation from 
   \p{\seq{\Gamma}{s}} to \p{\seq{\Gamma}{t}}, then the truth of \p{s} 
   guarantees the truth of \p{t}.

  \item Suppose we can show that if \plsh{Cst} is a tautology, then \p{\seq{}
   {s\limplies t}}. Explain why this would show that if \plsh{Cst} is a 
   tautology, then there is a derivation from \p{\seq{\Gamma}{s}} to 
   \p{\seq{\Gamma}{t}}.

\end{enumerate}



\section{Theorems and Tautologies}\label{sec:theoremsAndTautologies}

Let us introduce some terminology and symbols to make our life easier in 
answering the questions raised in the previous section.


 When the truth of one sentence \p{p} guarantees the truth of another sentence 
 \p{q}, we say that \p{p} \emph{entails} \p{q}.  We write 

\begin{center}
 \p{p\lentails q}
\end{center}

to express that \p{p} entails \p{q}. The entailment symbol, \p{\lentails} 
(double turnstile), is \emph{not} part of \lL{}. It is a symbol we use in our 
metalanguage to concisely express that the truth of one sentence guarantees the 
truth of another. Truth tables tell us that the truths of several sentences 
together guarantee the truth of another. For instance, the truth table for 
disjunction tells that if \p{p\lor q} and \p{\lnot p} are both true, that 
guarantees that \p{q} is true. In such a case, we use a comma separated list on 
the left side of the entailment sign to indicate the truths of all those 
sentences on the list guarantees the truth of the sentence on the right of the 
entailment sign:

\begin{center}
 \p{p\lor q,\lnot p\lentails q}
\end{center}

The order in which we list the sentences on the left is obviously irrelevant.
Notice that if \p{p\lentails q}, then \p{p}'s truth alone suffices to guarantee 
the truth of \p{q} which means that it does not matter what else is, or isn't, 
true.  Thus, if \p{p\lentails q}, then \p{p,\Gamma\lentails q} must also hold 
for any list of sentences \p{\Gamma}. 

Suppose \p{t} is a tautology. What is the list of sentences whose truths 
guarantees the truth of \p{t}? Since \p{t} is a tautology, any list of sentences 
is such that the truth of all the sentences on the list guarantees the truth of 
\p{t}. To indicate that it does not matter what list goes on the left side of 
the entailment symbol, we leave things blank there and write:

\begin{center}
 \p{\lentails t}
\end{center}

to mean that \p{t} is a tautology.

You will have noticed some obvious similarities between the uses of \p{\lproves} 
and \p{\lentails} symbols. But the two symbols mean different things.  
\p{\lproves} concerns what we can reason our way to, and \p{\lentails} is about 
what is true. For instance, inspecting the truth table of a sentence \p{t} might 
tell you that it is a tautology, but so far you have no reason to think that you 
will be able to construct a  derivation of \p{\seq{}{t}} (take a look again at 
the tautologies in Section \ref{sec-taut-contra} and see if you can prove them 
in our proof system).

With this notation in place, let's consider generalized versions of the 
preliminary exercises of the last section:


\begin{enumerate}

 \item Explain why there is a derivation from \p{\seq{\Gamma_1}{s_1}},  
  \p{\seq{\Gamma_2}
   {s_2}}, \ldots , \p{\seq{\Gamma_n}{s_n}} to \p{\seq{\Gamma_1, \Gamma_2,\ldots,
   \Gamma_n}
  {t}} iff. \p{\seq{}{[s_1\land (s_2 \ldots \land s_n)
 ]\limplies t}}. (imagine all the brackets in place to make things well-formed.)

 \item Explain why \p{\lentails (s_1\land s_2\ldots \land s_n)\limplies t} iff.  
  \p{s_1, s_2,\ldots, s_n \lentails t}.

  \item Suppose we can show that if \p{\seq{}{[s_1\land (s_2 \ldots \land s_n)
 ]\limplies t}}, then \p{\lentails ( s_1\land s_2\ldots \land s_n)
   \limplies t}.  Explain why this would show that: if there is a derivation 
   from \p{\seq{\Gamma_1}
  {s_1}},  \p{\seq{\Gamma_2}{s_2}}, \ldots , \p{\seq{\Gamma_n}{s_n}} to 
  \p{\seq{\Gamma}{t}}, then \p{s_1, s_2,\ldots, s_n \lentails t}.

  \item Suppose we can show that if \p{\lentails ( s_1\land s_2\ldots \land s_n)
   \limplies t}, then \p{\seq{}{[s_1\land (s_2 \ldots \land s_n)
 ]\limplies t}}.  Explain why this would show that: if \p{\lentails (s_1\land 
  s_2\ldots \land s_n)
   \limplies t}, then there is a derivation from \p{\seq{\Gamma_1}{s_1}},  
   \p{\seq{\Gamma_2}{s_2}}, \ldots , \p{\seq{\Gamma_n}{s_n}} to \p{\seq{\Gamma_1, 
	\Gamma_2,\ldots,\Gamma_n}
  {t}}.


\end{enumerate}

If we can show our suppositions in  3. and 4. above, we can show that an 
argument can be captured by a derivation if and only if it is valid. And we can 
show those two things, if we can show the following two more general claims:

\begin{description}

 \item[Soundness.] If \p{\seq{}{t}}, then  \p{\lentails t}.

 \item[Completeness.] If \p{\lentails t}, then \p{\seq{}{t}}.

\end{description}

We will prove both of these claims in what follows.

\section{Soundness}\label{sec:soundnessSL}



Let's start with soundness. When a proof system is such that all its theorems 
are tautologies,  the proof system is called \emph{sound}. We want to show that 
our proof system is sound.

To make stating our argument easier, let's say say that a sequent \p{\seq{\Gamma}
{s}} is \emph{correct} iff. \p{\Gamma\lentails s} (in particular, \p{\seq{}
{s}} is correct iff. \p{\lentails s}).



Now the argument showing the soundness of our proof system. Recall that a proof 
is constructed by extending a series of sequents in accordance with the 
inference rules.  We will show the following two points about such a series of 
sequents:

\begin{enumerate}\renewcommand{\labelenumi}{(\roman{enumi})}

 \item If the series has only one sequent, that sequent is correct.

 \item If all the sequents of a series of \emph{n} sequents are correct, 
  then extending the series by one sequent in accordance with the inference 
  rules results in a series of \emph{n+1} correct sequents.

\end{enumerate}

\emph{If} both of these points hold, then any series of length 1 consists of a 
correct sequent (because of (i)). So, by (ii), any series of length 2 consists 
of only correct sequents; ditto for any series of lengths 3, 4, etc. for any 
natural number n. So any series of sequents constructed in accordance with the 
inference rules consists of only correct sequents. In particular, if any sequent 
appearing in such a series has an empty datum, its succedent is a tautology. So 
any theorem is a tautology.

Let's deal with the two points one by one. 

\subsection{Series has only one sequent}

(i) is trivial. If a series of sequents constructed in accordance with the 
inference rules has length 1, that single sequent must have been inferred using 
the Assumption Introduction rule because that is the only rule that does not 
require any preceding sequents.  And the one sequent must take the form 
\p{\seq{s}{s}}. Moreover, \p{s\lentails s}. So (i) is true.

\subsection{Series has only correct sequents up to n-th line}

What about (ii)? This is more work, but we have in fact done most of the grunt 
work already. Most of the task is putting together the various pieces. 

We will say that an inference rule is \emph{valid} just in case it is such that: 
if all previous sequents are correct, then applying the rule results in 
another correct sequent. 

What we want to show is that all of our rules of inference are valid. If they 
are, then if all the sequents in a series of \m{n} sequents are correct, so is a 
sequent inferred from that series.

We will work through the rules one by one.
\begin{enumerate} \renewcommand{\labelenumi}{\arabic{enumi}.}
 \setlength{\labelwidth}{1.5em}
 \setlength{\leftmargin}{0em}

\item \textbf{Assumption Introduction}

The Assumption Introduction rule is obviously valid since it infers to \p{\seq{s}
{s}} and that is correct. In particular, it is correct if all the 
previous sequents are correct.

\item \textbf{Conditional Introduction}

Applying Conditional Introduction requires the presence of a sequent of the form 

\p{\seq{\Gamma,s_1}{s_2}} 

and allows inferring from this the sequent

\p{\seq{\Gamma}{s_1\limplies s_2}}. 

We are assuming that all sequents prior to applying this rule are correct.  
That is, we assume that \p{\Gamma,
 s_1\lentails s_2}. We need to show that given this assumption \p{\seq{\Gamma}
 {s_1\limplies s_2}} is also correct; i.e., \p{\Gamma\lentails s_1\limplies 
s_2}. 

Now, suppose the truth of \p{\Gamma} and the truth of \p{s_1} together  
guarantee the truth of \p{s_2} (`truth of \p{\Gamma}' is short for `truth of all 
the sentences of \p{\Gamma}).  Further suppose that \p{\Gamma} on its own does 
not guarantee the truth of \p{s_1\limplies s_2}.  That means that \p{\Gamma} 
could be true, and \p{s_1} true, and \p{s_2} false.  But this contradicts the 
first supposition that the truth of \p{\Gamma} and the truth \p{s_1} together 
guarantee the truth of \p{s_2}. Thus, the truth of \p{\Gamma} guarantees the 
truth of \p{s_1\limplies s_2}:
\begin{center}
If \p{\Gamma,s_1\lentails s_2}, then \p{\Gamma\lentails s_1\limplies s_2}.
\end{center}
Comparing with the Conditional Introduction rule, we can see that:

\begin{center}
 if \p{\seq{\Gamma,s_1}{s_2}} is correct then so is \p{\seq{\Gamma}
{s_1\limplies s_2}}.  \end{center}
Thus, Conditional Introduction is valid.

\item \textbf{Conditional Elimination}

 According to the truth table of the conditional, given a true conditional, if 
 the antecedent of a conditional is true, its consequent is also true. So if the 
 antecedent of a true conditional is guaranteed to be true, then its consequent 
 is guaranteed to be true. Thus, if \p{\Gamma} guarantees the truth of 
 \p{s_1\limplies s_2} and \p{\Delta} guarantees the truth of \p{s_1}, then 
 \p{\Gamma,\Delta} must guarantee the truth of \p{s_2}: 

 \begin{center}
  If \p{\Gamma\lentails s_1\limplies s_2} and \p{\Delta\lentails s_1}, then 
  \p{\Gamma,\Delta\lentails s_2}.  \end{center}

 So Conditional Elimination is a valid rule of inference.

There is a special case of the point we just made that we will be exploiting:

\begin{center}
If \p{\lentails s_1\limplies s_2} and \p{\Gamma\lentails s_1}, then 
\p{\Gamma\lentails s_2}. 

\end{center}
Or, more usefully: 

\begin{center}
 If \p{\lentails s_1\limplies s_2}, then \pp{if \p{\Gamma\lentails s_1} then 
 \p{\Gamma\lentails s_2}}. 

\end{center}

What this means is: if \p{\lentails s_1\limplies s_2}, then any rule that tells 
us that we may infer from \p{\seq{\Gamma}{s_1}} to \p{\seq{\Gamma}{s_2}} is 
valid.


\item \textbf{Negation Elimination}

We have already shown that \p{\lentails \lnot\lnot s\limplies s}. Thus, by what 
we have just pointed out above, Negation Elimination is valid.

\item \textbf{Disjunction Introduction}

We also have shown that \p{\lentails s_1 \limplies (s_1\lor s_2)} as well as 
\p{\lentails s_1\limplies(s_2\lor s_1)}. Thus Disjunction Introduction is valid.

\item \textbf{Conjunction Elimination}

Similarly, we have seen that \p{\lentails (s_1\land s_2) \limplies s_1} as well 
as \p{\lentails (s_1 \land s_2) \limplies s_2}. So Conjunction Elimination is 
valid.


\item \textbf{Conjunction Introduction}

This is a bit more complicated but not by much. We know that

\begin{center}
 
 \p{\lentails \plshn{Cs_1Cs_2Ks_1s_2}}.
   

\end{center}

 So
 
 \begin{center}
  
  If \p{\Gamma\lentails s_1}, then \p{\Gamma\lentails s_2\limplies (s_1\land 
  s_2)}.

  \end{center}


   And this means that if additionally \p{\Delta\lentails s_2}, then \p{\Gamma,
 \Delta\lentails s_1\land s_2}.  Put together, we have:
 
 \begin{center}

  If \p{\Gamma\lentails s_1} and \p{\Delta\lentails s_2}, then \p{\Gamma,
  \Delta\lentails s_1\land s_2}. 

 \end{center}

  I.e., if \p{\seq{\Gamma}{s_1}
 } and \p{\seq{\Delta}{s_2}} are correct, then \p{\seq{\Gamma,\Delta}
{s_1\land s_2}} is correct. Thus, Conjunction Introduction is valid.


\item \textbf{Disjunction Elimination}

 Let's leave this as an exercise.
 

\item \textbf{Negation Introduction}

 Let's leave this as an exercise, too.


\end{enumerate}

\subsection{Drawing the conclusion}

Since all the inference rules are valid, given  any series of \m{n} correct 
sequents, inferring in accordance with our proof system will result in a series 
of \m{n+1} correct sequents. Since we know that a series of 1 sequent is a 
series with only correct sequents---remember, only Assumption Introduction can 
get that single line in place---we now know that extending such a series to 2 
sequents will leave us with a series of 2 correct sequents, and ditto if we 
continue to infer a third sequent, fourth sequent, etc. It does not matter how 
long the series of sequents gets, if each sequent was inferred using one or the 
other of our rules of inference, then the series of sequents contains only 
correct sequents. Therefore, any sentence proven to be a theorem is indeed a 
tautology.  Our system of sentential logic is sound.




\section{Exercise: Completeness}\label{sec:propCompleteness}
\newcommand{\setM}{\ensuremath{\mathcal{M}}}

This section walks you through an argument showing Completeness: any tautology 
of \lL{} is also a theorem (if \p{\lentails t}, then \p{\seq{}{t}}). 

The questions give an outline of the proof of Completeness (`proof' in the 
colloquial sense---it's not a proof in the technical sense we introduced) in the 
style of the proof of Completeness given by László \citet{Kalmar1935}. Your task 
is to fill in  the details.  Just as in the fill-in-the-blanks exercises, it 
makes sense to look at the whole thing first---it might be that some later parts 
give you clues about how to do earlier parts. And if you don't know how to do a 
question, you can proceed under the pretense that you managed to answer the 
question.


The easiest way to understand how our argument works is by thinking of what the 
truth table of a sentence \p{s} tells us. A truth tables lists all interpretations and it 
tells us for each interpretation whether or not \p{s} is true in that interpretation.  Let us 
think of a interpretation not as a series of \emph{T}s and   \emph{F}s but as a set of 
sentences: for each atomic sentence \p{a_i}, if the table says \emph{T}, \p{a_i} 
is in the interpretation, and if the table says \emph{F}, \p{\lnot a_1} is in the interpretation; 
and no other sentences are in the interpretation.  For instance, for \lL{} with two 
atomic sentences \p{a_1} and \p{a_2}, the interpretations are:

\begin{enumerate}

 \item \p{a_1, a_2}\texpl{corresponds to \emph{TT}}

 \item \p{\lnot a_1, a_2}\texpl{corresponds to \emph{FT}}

 \item \p{a_1, \lnot a_2}\texpl{corresponds to \emph{TF}}

 \item \p{\lnot a_1, \lnot a_2}\texpl{corresponds to \emph{FF}}

\end{enumerate}

In the following, we will be using this understanding of interpretations.


\newpage


\subsection*{The Proof}



\begin{enumerate}
 \newcommand{\Seq}[1]{\p{\seq{\setM}{#1}}}

 \item Let's first take some sentence \p{t} and see what happens if 
  \p{\seq{\setM}{t}} for any \setM.  We will show that in that case \p{\seq{}
  {t}}. (Note:  \setM{} is a placeholder for a set of sentence. So it must not 
  appear on the succedent side of a sequent.)

 Suppose \p{t} is a sentence constructed using $n$ atomic sentences \p{a_1} 
 through \p{a_n}, and further suppose that \p{\seq{\setM}{t}} for any interpretation 
 \setM. That is, we are supposing that the following $2^n$ sequents can all be 
 proven (we list them in the order we list interpretations in a truth table):


 \begin{tabular}{l r r r r r c c}
  %
  1. & \p{a_1}, & \p{a_2}, &  \p{a_3}, & \ldots, & \p{a_n} &\lproves & \p{t}\\
 %
  2. & \p{\lnot a_1}, & \p{a_2}, &  \p{a_3}, & \ldots, & \p{a_n} &\lproves & \p{t}\\
  %
  3. & \p{a_1}, & \p{\lnot a_2}, &  \p{a_3}, & \ldots, & \p{a_n} &\lproves & \p{t}\\
  %
  4. & \p{\lnot a_1}, & \p{\lnot a_2}, &  \p{a_3}, & \ldots, & \p{a_n} &\lproves 
	 & \p{t}\\
	 %
  5. & \p{a_1}, & \p{a_2}, &  \p{\lnot a_3}, & \ldots, & \p{a_n} &\lproves & \p{t}\\
  %
  6. & \p{\lnot a_1}, & \p{a_2}, &  \p{\lnot a_3}, & \ldots, & \p{a_n} &\lproves 
	 & \p{t}\\
	 %
  7. & \p{a_1}, & \p{\lnot a_2}, &  \p{\lnot a_3}, & \ldots, & \p{a_n} &\lproves 
	 & \p{t}\\
	 %
  8. & \p{\lnot a_1}, & \p{\lnot a_2}, &  \p{\lnot a_3}, & \ldots, & \p{a_n} 
	 &\lproves & \p{t}\\
	 %
  \vdots &  &  &   &  & &\\
  %
  $2^n$. & \p{\lnot a_1}, & \p{\lnot a_2}, &  \p{\lnot a_3}, & \ldots, & 
  \p{\lnot a_n} &\lproves & \p{t}\\
  %
 \end{tabular}

 \begin{enumerate}

  \item From the first pair of sequents in the list above, derive: 

 \begin{tabular}{r r r r c c}
  %
  \p{a_2}, &  \p{a_3}, & \ldots, & \p{a_n} &\lproves & \p{t}\\
 \end{tabular}

 You may appeal to exactly one theorem that we have named earlier, and you may 
 appeal to this one theorem in the subsequent problems; but you may not appeal 
 to any other theorem.  

 
\opts{
 \dotline

\dotline

\dotline

\dotline

\dotline
}
{
 \answerk{

  \begin{argument*}

   \ai[0.3]{a_1,a_2,a_3,\ldots,a_n}{t}{premise}
   \ai[0.3]{\lnot a_1,a_2,a_3\ldots,a_n}{t}{premise}
   \ai[0.3]{}{a_1\lor\lnot a_1}{EM}
   \ai[0.3]{a_2,a_3,\ldots,a_n}{t}{1,2,3,\disjE}

 \end{argument*}

}
}


\item From the second pair of sequents, derive:

 \begin{tabular}{r r r r c c}
  %
  \p{\lnot a_2}, &  \p{a_3}, & \ldots, & \p{a_n} &\lproves & \p{t}\\
 \end{tabular}

\opts{

\dotline

\dotline

\dotline

\dotline

\dotline
}
{
 \answerk{

  \begin{argument*}

   \ai[0.3]{a_1,\lnot a_2,a_3,\ldots,a_n}{t}{premise}
   \ai[0.3]{\lnot a_1,\lnot a_2,a_3\ldots,a_n}{t}{premise}
   \ai[0.3]{}{a_1\lor\lnot a_1}{EM}
   \ai[0.3]{\lnot a_2,a_3,\ldots,a_n}{t}{1,2,3,\disjE}
 \end{argument*}

 NB: I renumber the derivations to start from line numbered 1, but that's not 
 necessary. The numbers are there so we can unambiguously refer to them in the 
 annotations. If chose other line numbers, that's fine.
 }
}


\item From the results of (a) and (b), derive:

 \begin{tabular}{r r r c c}
  %
  \p{a_3}, & \ldots, & \p{a_n} &\lproves & \p{t}\\
 \end{tabular}

\opts{

\dotline

\dotline

\dotline

\dotline

\dotline

\dotline
}
{
\answerk{

 \begin{argument*}

  \ai[0.3]{a_2,a_3,\ldots,a_n}{t}{premise}
  \ai[0.3]{\lnot a_2,a_3,\ldots,a_n}{t}{premise}
  \ai[0.3]{}{a_2\lor\lnot a_2}{EM}
  \ai[0.3]{a_3,\ldots,a_n}{t}{1,2,3,\disjE}

 \end{argument*}
}
}


\item From the sequents number 5 through 8 in the above list, derive:

 \begin{tabular}{r r r c c}
  %
  \p{\lnot a_3}, & \ldots, & \p{a_n} &\lproves & \p{t}\\
 \end{tabular}

\opts{

\dotline

\dotline

\dotline

\dotline

\dotline

\dotline

\dotline

\dotline

\dotline

\dotline
}
{
 \answerk{

  \begin{argument*}

   \ai[0.3]{a_1,a_2,a_3,\ldots,a_n}{t}{premise}
   \ai[0.3]{\lnot a_1,a_2,a_3,\ldots,a_n}{t}{premise}
   \ai[0.3]{a_1,\lnot a_2,a_3,\ldots,a_n}{t}{premise}
   \ai[0.3]{\lnot a_1,\lnot a_2,a_3,\ldots,a_n}{t}{premise}
   \ai[0.3]{}{a_1\lor \lnot a_1}{EM}
   \ai[0.3]{a_2,a_3,\ldots,a_n}{t}{1,2,5,\disjE}
   \ai[0.3]{\lnot a_2,a_3,\ldots,a_n}{t}{3,4,5,\disjE}
   \ai[0.3]{}{a_2\lor\lnot a_2}{EM}
   \ai[0.3]{a_3,\ldots,a_n}{t}{6,7,8,\disjE}

  \end{argument*}

 }
}


\item Notice how the datum side of the sequents get shorter? Explain why you 
 eventually will be able to derive the following two sequents:

 \begin{tabular}{r c c}
  %
  \p{a_n} &\lproves & \p{t}\\
  \p{\lnot a_n} &\lproves & \p{t}\\

 \end{tabular}


\opts{
\dotline

\dotline

\dotline

\dotline

\dotline

\dotline

\dotline

\dotline

\dotline

\dotline
}
{\answerk{

The initial list of sequents lists them pairwise from the top: each pair has 
identical datums after the first item in the datum, and the first datum items 
are negations of each other.  We can use EM to halve the number of sequents and 
we will end with a list that again lists sequents in pairs where each pair has 
identical datums after the first item in the datum and the first datum items are 
negations of each other.  We can use EM again to again halve the number of 
sequents.  If we repeat the procedure often enough, we will end with the two 
sequents shown above.
}
}



 \item Derive \p{\seq{}{t}} from the result of (e).

\opts{

 \dotline

\dotline

\dotline

\dotline

\dotline

}
{
 \answerk{

  \begin{argument*}

   \ai{a_n}{t}{premise}
   \ai{\lnot a_n}{t}{premise}
   \ai{}{a_n\lor\lnot a_n}{EM}
   \ai{}{t}{1,2,3,EM}

  \end{argument*}
 }
}

\end{enumerate}



This results show that if \p{\seq{\setM}{t}} for every interpretation \setM, 
then \p{\seq{}
{t}} (i.e., \p{t} is a theorem). This is one important piece of the puzzle.


\vskip 2em
\item Let's move to another piece of the puzzle. We will show that for any 
 sentence \p{s} and interpretation \p{\setM}, either we can prove that \p{\seq{\setM}
 {s}} or we can prove that \p{\seq{\setM}{\lnot s}}. We will make use of the 
 classification of sentences discussed in Chapter
 \ref{sec:enumeratingSentences}.

 \begin{enumerate}

 

\item First, let \p{p} be a class-0 sentence.  Show that for any interpretation 
 \setM{}, either we can \emph{prove} \Seq{p} or we can \emph{prove} \Seq{\lnot p}
 . (Hint: notice that \setM{} must either contain \p{p} or contain \p{\lnot p}.)

\opts{

 \dotline

\dotline

\dotline

\dotline

\dotline

\dotline

\dotline

\dotline

\dotline

\dotline
}
{
 \answerk{

  case 1: \setM{} contains \p{p}:

  \begin{argument*}
   \ai{p}{p}{A}
   \ai{\setM,p}{p}{1}
   \ai{\setM}{p}{2}
  \end{argument*}

  case 2: \setM{} contains \p{\lnot p}

  \begin{argument*}
   \ai{\lnot p}{\lnot p}{A}
   \ai{\setM,\lnot p}{\lnot p}{1}
   \ai{\setM}{\lnot p}{2}
  \end{argument*}

  Note: the moves to the respective lines 3 are justified because we assume that 
  \p{p} (or \p{\lnot p}) is contained in \setM{}.
 }
}

  \item Secondly, suppose that for any sentence \p{p} of class-n either \Seq{p} 
   or \Seq{\lnot p}.  All the subquestions of this problem are under this 
   supposition.  Let us show that on this supposition, it is also true that for 
   any sentence \p{s} of class-(n+1) either \Seq{s} or \Seq{\lnot s}.  


We can show this by working through all the possible cases. 


Construct each of the following indicated derivations---except x, xi, xii for 
which you should fill in the blanks. Some derivations are trivial; others 
require a bit more work; and not all derivations will make use of all the 
indicated premises.  Most importantly, you can find clues about how to do most 
of these from previous exercises/practice material/tests (or even the current 
exercise):

\begin{enumerate}

 \item From \Seq{p} and \Seq{q}, derive \Seq{p\land q}.

\opts{
\dotline

\dotline

\dotline

\dotline

\dotline

\dotline

\dotline
}{\answerk{
 \begin{argument*}
  \ai{\setM}{p}{premise}
  \ai{\setM}{q}{premise}
  \ai{\setM}{p\land q}{1,2,\conjI}
\end{argument*}}
}

\item From \Seq{p} and \Seq{q}, derive \Seq{p\lor q}.

\opts{
 \dotline

\dotline

\dotline

\dotline

\dotline

\dotline
}
{\answerk{

  \begin{argument*}
   \ai{\setM}{p}{premise}
   \ai{\setM}{q}{premise}
   \ai{\setM}{p\lor q}{1,\disjI}
  \end{argument*}
 }
}




\item From \Seq{p} and \Seq{q}, derive \Seq{p\limplies q}.

\opts{
\dotline

\dotline

\dotline

\dotline

\dotline

\dotline

\dotline
}
{
 \answerk{

  \begin{argument*}
   \ai{\setM}{p}{premise}
  \ai{\setM}{q}{premise}
  \ai{\setM,p}{q}{2}
  \ai{\setM}{p\limplies q}{3,\condI}
 \end{argument*}
}
}





\item From \Seq{\lnot p} and \Seq{q}, derive \Seq{\lnot(p\land q)}.

\opts{

 \dotline

\dotline

\dotline

\dotline

\dotline

\dotline

\dotline

\dotline
}{
\answerk{

 \begin{argument*}

  \ai{\setM}{\lnot p}{premise}
  \ai{\setM}{q}{premise}
  \ai{p\land q}{p\land q}{A}
  \ai{p\land q}{p}{3,\conjE}
  \ai{\setM,p\land q}{\lnot p}{1}
  \ai{\setM}{\lnot(p\land q)}{4,5,\negI}
 \end{argument*}
}
}
\newpage
\item From \Seq{\lnot p} and \Seq{q}, derive \Seq{p\lor q}.

\opts{
\dotline

\dotline

\dotline

\dotline

\dotline
}
{
 \answerk{

  \begin{argument*}
   \ai{\setM}{\lnot p}{premise}
   \ai{\setM}{q}{premise}
   \ai{\setM}{p\lor q}{2,\disjI}
  \end{argument*}
 }
}


\item From \Seq{\lnot p} and \Seq{q}, derive \Seq{p\limplies q}.

\opts{
\dotline

\dotline

\dotline

\dotline

\dotline

\dotline
}
{
\answerk{
  \begin{argument*}
   \ai{\setM}{\lnot p}{premise}
  \ai{\setM}{q}{premise}
  \ai{\setM,p}{q}{2}
  \ai{\setM}{p\limplies q}{3,\condI}
 \end{argument*}
}
}

\item From \Seq{p} and \Seq{\lnot q}, derive \Seq{\lnot(p\land q)}.

\opts{
\dotline

\dotline

\dotline

\dotline

\dotline

\dotline

\dotline

\dotline
}{
\answerk{

 \begin{argument*}

  \ai{\setM}{p}{premise}
  \ai{\setM}{\lnot q}{premise}
  \ai{p\land q}{p\land q}{A}
  \ai{p\land q}{q}{3,\conjE}
  \ai{\setM,p\land q}{\lnot q}{2}
  \ai{\setM}{\lnot(p\land q)}{4,5,\negI}
 \end{argument*}
}
}


\item From \Seq{p} and \Seq{\lnot q}, derive \Seq{p\lor q}.

\opts{
 \dotline

\dotline

\dotline

\dotline

\dotline

\dotline

\dotline

}{

 \answerk{

  \begin{argument*}
   \ai{\setM}{p}{premise}
   \ai{\setM}{\lnot q}{premise}
   \ai{\setM}{p\lor q}{1,\disjI}
  \end{argument*}
 }
}

 \item From \Seq{p} and \Seq{\lnot q}, derive \Seq{\lnot(p\limplies q)}.

\opts{

\dotline

\dotline

\dotline

\dotline

\dotline

\dotline

\dotline
}
{
\answerk{

 \begin{argument*}
  \ai{\setM}{p}{premise}
  \ai{\setM}{\lnot q}{premise}
  \ai{p\limplies q}{p\limplies q}{A}
  \ai{\setM,p\limplies q}{q}{1,3,\condE}
  \ai{\setM,p\limplies q}{\lnot q}{2}
  \ai{\setM}{\lnot(p\limplies q)}{4,5,\negI}
 \end{argument*}
}
}




\setlength{\itemsep}{2em}

\item Fill in missing items. This derives from  \Seq{\lnot p} and \Seq{\lnot q} 
 to \Seq{\lnot(p\land q)}.


\begin{argumentN}[1]
%generated by gentzen
 \setlength{\itemsep}{0.5em}
\ai{\setM}{\mc{\lnot }p}{premise}

\ai{\setM}{\mc{\lnot }q}{premise}

\ai{p\mc{\land }q}{p\mc{\land }q}{A}

\ai{\mask{p\mc{\land }q}}{\mask{p}}{3,\mask{\conjE}}

\ai{\setM, \mask{p\mc{\land }q}}{\mask{\mc{\lnot }p}}{1}

\ai{\setM}{\mc{\lnot }(p\land q)}{4,5,\negI}

\end{argumentN}

 \item Fill in missing items. This derives from \Seq{\lnot p} and \Seq{\lnot q} 
  to \Seq{\lnot(p\lor q)}.



\begin{argumentN}[1]
%generated by gentzen

 \setlength{\itemsep}{0.5em}
\ai{\setM}{\mc{\lnot }p}{premise}

\ai{\setM}{\mc{\lnot }q}{premise}

\ai{p\mc{\lor }q}{p\mc{\lor }q}{\mask{A}}

\ai{p}{p}{\mask{A}}

\ai{p, p\mc{\lor }q}{p}{\mask{4}}

\ai{\setM, p\mc{\lor }q}{\mc{\lnot }p}{\mask{1}}

\ai{\setM, p}{\mc{\lnot }(p\lor q)}{\mask{5,6,\negI}}

\ai{q}{q}{\mask{A}}

\ai{q, p\mc{\lor }q}{q}{\mask{8}}

\ai{\setM, p\mc{\lor }q}{\mc{\lnot }q}{\mask{2}}

\ai{\setM, q}{\mc{\lnot }(p\lor q)}{\mask{9,10,\negI}}

\ai{\setM, p\mc{\lor }q}{\mc{\lnot }(p\lor q)}{\mask{3,7,11,\disjE}}

\ai{\setM}{\mc{\lnot }(p\lor q)}{\mask{3,12,\negI}}

\end{argumentN}

 \item Fill in missing items. This derives from \Seq{\lnot p} and \Seq{\lnot q} 
  to \Seq{p\limplies q}.


\begin{argumentN}[1]
%generated by gentzen

 \setlength{\itemsep}{0.5em}
\ai{\setM}{\mc{\lnot }p}{premise}

\ai{\setM}{\mc{\lnot }q}{premise}

\ai{p}{p}{A}

\ai{\mask{\setM, \mc{\lnot }q}}{\mask{\mc{\lnot }p}}{1}

\ai{\mask{p, \mc{\lnot }q}}{\mask{p}}{3}

\ai{\setM, p}{\mask{\mc{\lnot }\lnot q}}{4,5,\negI}

\ai{\setM, p}{\mask{q}}{6,\negE}

\ai{\setM}{p\mc{\limplies }q}{7,\condI}

\end{argumentN}

\item From \Seq{p}, derive \Seq{\lnot\lnot p}.

\opts{
\dotline

\dotline

\dotline

\dotline

\dotline

\dotline
}
{
 \answerk{

  \begin{argument*}

\ai{\setM}{p}{premise}
   \ai{\lnot p}{\lnot p}{A}
   \ai{\setM,\lnot p}{p}{1}
   \ai{\setM}{\lnot\lnot p}{2,3,\negI}

  \end{argument*}
  }
 }

\end{enumerate}
\vskip 1.5em
\item   Let's make sure you see the point of the above:

  \begin{enumerate}
	
	\item Explain why if \p{s} is a conjunction of two class-n sentences, then 
	 either \Seq{s} or \Seq{\lnot s}.  

	 
\opts{
\dotline

\dotline

\dotline

\dotline

\dotline

\dotline
}{
\answerk{

 Let \p{s} be \p{p\land q} where \p{p} and \p{q} are class-n sentences. We know 
 from 2-c-i that \p{\setM\lproves s} if \Seq{p} and \Seq{q}, and we know that
 \p{\setM\lproves \lnot s} otherwise from 2-c-iv, vii, and x. So if \Seq{r} or 
 \Seq{\lnot r} for any class-n sentence \p{r}, then \Seq{s} or \Seq{\lnot s} if 
 \p{s} is a conjunction of class-n sentences.

}
}


\item Explain why if \p{s} is a disjunction of two class-n sentences, then 
 either \Seq{s} or \Seq{\lnot s}.

\opts{
 \dotline

\dotline

\dotline

\dotline

\dotline

\dotline
}
{
 \answerk{
 
  Let \p{s} be \p{p\lor q} where \p{p} and \p{q} are class-n sentences. We know 
  from 2-c-xi that \p{\setM\lproves \lnot s} if \Seq{\lnot p} and \Seq{\lnot q}, 
  and we know that \p{\setM\lproves s} otherwise from 2-c-ii, v, and viii.  So 
  if \Seq{r} or \Seq{\lnot r} for any class-n sentence \p{r}, then \Seq{s} or 
  \Seq{\lnot s} if \p{s} is a conjunction of class-n sentences.

 }
}


\item Explain why if \p{s} is a conditional of two class-n sentences, then 
 either \Seq{s} or \Seq{\lnot s}.

\opts{
\dotline

\dotline

\dotline

\dotline

\dotline

\dotline
}
{
 \answerk{

  Let \p{s} be \p{p\limplies  q} where \p{p} and \p{q} are class-n sentences.  
  We know from 2-c-ix that \p{\setM\lproves \lnot s} if \Seq{p} and \Seq{\lnot 
  q}, and we know that \p{\setM\lproves s} otherwise from 2-c-iii, vi, and xii.  
  So if \Seq{r} or \Seq{\lnot r} for any class-n sentence \p{r}, then \Seq{s} or 
  \Seq{\lnot s} if \p{s} is a conjunction of class-n sentences.
 }
}


 \item Explain why if \p{s} is a negation of a class-n sentence, then either 
  \Seq{s} or \Seq{\lnot s}.

\opts{

\dotline

\dotline

\dotline

\dotline

\dotline

\dotline
}
{
 \answerk{

  Let \p{s} be \p{\lnot p} where \p{p} is a class-n sentence. We know from 
  2-c-xiii that if \Seq{p}, then \Seq{\lnot s}. If \Seq{\lnot p}, then \Seq{s}. 
  So if \Seq{r} or \Seq{\lnot r} for any class-n sentence \p{r}, then \Seq{s} or 
  \Seq{\lnot s} if \p{s} is a conjunction of class-n sentences.
 }
}

\end{enumerate}   





 \item Explain why it follows that: 
	 
	If for any sentence \p{p} of class-n either \Seq{p} or \Seq{\lnot p}, then 
	any sentence \p{s} of class-(n+1) is such that either \Seq{s} or \Seq{\lnot 
	s}.

\opts{
\dotline

\dotline

\dotline

\dotline

\dotline

\dotline

\dotline
}{
\answerk{

 Any sentence \p{s} of class-(n+1) must be one of: a class-n sentence; a 
 conjunction, disjunction, conditional between two class-n sentences; or a 
 negation of a class-n sentence. The results of 2d show that \Seq{s} or 
 \Seq{\lnot s} given that \Seq{p} or \Seq{\lnot p} for any sentence of class-n.

}

}



 \end{enumerate}
\item We now assemble our pieces of the puzzle. 

 \begin{enumerate}

  \item Explain why if \p{t} is a tautology, it is not the case that 
   \p{\setM\lentails \lnot t} for any interpretation \setM.

\opts{

\dotline

\dotline

\dotline

\dotline

\dotline

\dotline
}{

 \answerk{

 If \p{t} is a tautology, it is true in all interpretations. So \p{\lnot t} is false in 
 all interpretations, which means \p{\setM\lentails \lnot t} is not the case for any 
 interpretation \p{\setM}.

}
}

\item Explain why if \p{\seq{\setM}{\lnot t}}, then \p{\setM\lentails\lnot t} 
 (Hint: recall we already know an important feature of our proof system).
 
\opts{

\dotline

\dotline

\dotline

\dotline

\dotline

\dotline
}{

 \answerk{

  Since our proof system is sound, we know that for any \p{p} if 
  \p{\Gamma\lproves p}, then \p{\Gamma\lentails p}. So if \p{\seq{\setM}
   {\lnot t}}, then \p{\setM\lentails \lnot t}.
 }
}


 \cover{\newpage}
  \item Explain why if \p{\lentails t}, then \p{\seq{\setM}{t}} for any 
   interpretation \setM.

\opts{
\dotline

\dotline

\dotline

\dotline

\dotline

\dotline

\dotline
}{
\answerk{

 Suppose \p{\lentails t}. From the results of 2b and 2e, it follows that \Seq{s} 
 or \Seq{\lnot s} for any sentence \p{s} and interpretation \setM. Thus, either 
 \Seq{t} or \Seq{\lnot t}.
 If \Seq{\lnot t}, it follows from 3b that \p{\setM\lentails \lnot t}.  But we 
 know from 3a that this cannot be given that \p{t} is a tautology.  Thus, if 
 \p{\lentails t}, then \Seq{t} for any interpretation \setM.
}
}

 \end{enumerate}


\item Explain why it follows from what we have shown that our proof system is 
 complete.

\opts{
\dotline

\dotline

\dotline

\dotline

\dotline

\dotline

\dotline
}{
\answerk{

 We know from 3c that if \p{\lentails t}, then \Seq{t} for any interpretation \setM.  
 We know from Q1 that if \Seq{t} for any interpretation \setM, then \p{\seq{}{t}}.  
 Thus, if \p{\lentails t}, then \p{\seq{}{t}}.

}
}



\end{enumerate}


Q.E.D. (\emph{quod erat demonstandum}. `This is what was to be shown' in Latin.) 
Congratulations! You are done.


\subsection*{Note}

The above argument also gives you a recipe for proving any tautology by 
mimicking the construction of truth tables.  For example, here is a proof of 
\p{\seq{}
   {(p\land q)
  \limplies p}} suggested by the argument:
\footnotesize

\begin{argumentN}[1]
 \setlength{\itemsep}{0em}
\ai{p}{p}{A}

\ai{q}{q}{A}

\ai{\mc{\lnot }p}{\mc{\lnot }p}{A}

\ai{\mc{\lnot }q}{\mc{\lnot }q}{A}

\ai{p, q}{p}{1}

\ai{p, q}{q}{2}

\ai{\mc{\lnot }p, q}{\mc{\lnot }p}{3}

\ai{\mc{\lnot }p, q}{q}{2}

\ai{p, \mc{\lnot }q}{p}{1}

\ai{p, \mc{\lnot }q}{\mc{\lnot }q}{4}

\ai{\mc{\lnot }p, \mc{\lnot }q}{\mc{\lnot }p}{3}

\ai{\mc{\lnot }p, \mc{\lnot }q}{\mc{\lnot }q}{4}

\ai{p, q}{p\mc{\land }q}{5,6,\conjI}

\ai{p\mc{\land }q}{p\mc{\land }q}{A}

\ai{p\mc{\land }q}{p}{14,\conjE}

\ai{\mc{\lnot }p, q, p\mc{\land }q}{\mc{\lnot }p}{7}

\ai{\mc{\lnot }p, q}{\mc{\lnot }(p\land q)}{15,16,\negI}

\ai{p\mc{\land }q}{q}{14,\conjE}

\ai{p, \mc{\lnot }q, p\mc{\land }q}{\mc{\lnot }q}{10}

\ai{p, \mc{\lnot }q}{\mc{\lnot }(p\land q)}{18,19,\negI}

\ai{\mc{\lnot }p, \mc{\lnot }q, p\mc{\land }q}{\mc{\lnot }p}{11}

\ai{\mc{\lnot }p, \mc{\lnot }q}{\mc{\lnot }(p\land q)}{15,21,\negI}

\ai{p, q, p\mc{\land }q}{p}{5}

\ai{p, q}{(p\land q)\mc{\limplies }p}{23,\condI}

\ai{\mc{\lnot }p, q, \mc{\lnot }p}{\mc{\lnot }p}{7}

\ai{p\mc{\land }q, \mc{\lnot }p}{p}{15}

\ai{\mc{\lnot }p, q, p\mc{\land }q}{\mc{\lnot }\lnot p}{25,26,\negI}

\ai{\mc{\lnot }p, q, p\mc{\land }q}{p}{27,\negE}

\ai{\mc{\lnot }p, q}{(p\land q)\mc{\limplies }p}{28,\condI}

\ai{p, \mc{\lnot }q, p\mc{\land }q}{p}{9}

\ai{p, \mc{\lnot }q}{(p\land q)\mc{\limplies }p}{30,\condI}

\ai{\mc{\lnot }p, \mc{\lnot }q, \mc{\lnot }p}{\mc{\lnot }p}{11}

\ai{\mc{\lnot }p, \mc{\lnot }q, p\mc{\land }q}{\mc{\lnot }\lnot p}{26,32,\negI}

\ai{\mc{\lnot }p, \mc{\lnot }q, p\mc{\land }q}{p}{33,\negE}

\ai{\mc{\lnot }p, \mc{\lnot }q}{(p\land q)\mc{\limplies }p}{34,\condI}

\ai{}{p\mc{\lor }\lnot p}{EM}

\ai{q}{(p\land q)\mc{\limplies }p}{24,29,36,\disjE}

\ai{\mc{\lnot }q}{(p\land q)\mc{\limplies }p}{31,35,36,\disjE}

\ai{}{q\mc{\lor }\lnot q}{EM}

\ai{}{(p\land q)\mc{\limplies }p}{37,38,39,\disjE}

\end{argumentN}


\normalsize

\setlength{\parindent}{0em}
You see why no one teaches this recipe for generating proofs. It's spending 40 
lines to prove something that can be proved in three lines.  But it shows a 
point of principle.  It is not too difficult to write computer code to generate 
such a proof for any theorem no matter how complex the theorem.  












\section{Significance of Soundness and Completeness}

What is the point of proving soundness and completeness?  We started our 
discussion of the proof system by considering ordinary language arguments and 
how they might be formalized. Our formal system is designed to keep track of
what supports what. Soundness of our system means that if we start with some 
claims conclusively supported by the evidence (the datums of the premises are 
the evidence, the succedents the claims), then inferences drawn in accordance 
with our proof system will correctly identify what else is supported by that 
evidence. Our proof system is immune to fallacious inferences. 

Secondly, since our proof system is complete, it is possible to infer to 
everything that is conclusively supported by the evidence. There is no need to 
worry that reasoning in accordance with the proof system will make us miss out 
on some things supported by the evidence. 

Taken together, this means that our proof system is just right. We can reach all 
conclusions we want to reach, while keeping us from drawing conclusions that are 
illicit. 

Our proof system is extracted from observations of what we regard as  good 
reasoning.  What we have shown is that what we regard as good cases of slow and 
careful reasoning really are good. While there are some important limitations of 
our proof system that we have noted earlier---and we will see more very
soon---completeness shows that what we can do, we can do well.  And that is a 
good thing to know about ourselves. 

The arguments for soundness and completeness are not arguments formulated within 
our proof system. They are theorems \emph{about} our proof system and are not 
themselves theorems of the proof system. They are known as \emph{meta-theorems}.  
Meta-theorems are proven in the meta-language. That is why, for instance,  many 
`if, then' statements used in the course of the argument did not use the 
horseshoe symbol (\p{\limplies})---our meta-language is English, and the 
horseshoe symbol is an expression of \lL{} but not of English. 


\flushbottom
\widowpenalty=150

 



\section*{The language of predicate logic}


\newcommand{\ansk}[1]{\opts{
  
  \vskip 0.5em
  \dotline

 }
 {#1}
}

\newcommand{\ansF}{\dotfill True/\anscircled{False}}
\newcommand{\ansT}{\dotfill \anscircled{True}/False}

\renewcommand{\lbh}[1]{\input{|"lbhelper -p -f #1"}}

\begin{enumerate}

 \item \p{Rx} means that x is a rodent. \p{Fx} means that x is a feline. \p{Exy} 
  means that x eats y. \p{Cx} means that x is cute. Let \p{j} be a constant 
  referring to Jerry, \p{t} be a constant referring to Tom, \p{m} be a constant 
  referring to Minnie, and \p{k} be a constant referring to Kitty.  Translate 
  the following English sentences into sentences of predicate logic.

  \coverbox{The provided answers are `standard' answers. There might be other 
  translations that are just as good.}

  \begin{enumerate}
   \setlength{\itemsep}{1em}
   \item Jerry is a rodent.

	\ansk{\p{Rj}}

   \item Kitty is a feline.

	\ansk{\p{Fk}}

   \item Felines are cute.

	\ansk{\p{\lforall x (Fx \limplies Cx)}}

   \item Tom is not cute.

	\ansk{\p{\lnot Ct}}

	\item If Tom is feline, then some felines are not cute.

	 \ansk{\p{Ft \limplies \lthereis x (Fx\land \lnot Cx)}}

	\item Tom does not eat Jerry.

	 \ansk{\p{\lnot Etj}}

	\item Minnie is not a feline.

	 \ansk{\p{\lnot Fm}}

	\item If something is not a feline, it is a rodent.

	 \ansk{\p{\lforall x (\lnot Fx \limplies Rx)}}

	\item Some things are rodents.

	 \ansk{\p{\lthereis x Rx}}

	\item There are rodents.

	 \ansk{\p{\lthereis x Rx}}


	\item Rodents are not cute.

	 \ansk{\p{\lforall x (Rx \limplies \lnot Cx)}}

	\item No rodent is cute.

	 \ansk{\p{\lnot \lthereis x (Rx \land Cx)}}

	\item If Minnie is a rodent, Kitty eats Minnie.

	 \ansk{\p{Rm \limplies Ekm}}

	\item Some rodents eat felines.

	 \ansk{\p{\lthereis x (Rx \land \lthereis y (Fy \land Exy))}}

	\item There are rodents that eat felines.

	 \ansk{\p{\lthereis x (Rx \land \lthereis y (Fy \land Exy))}}

	\item No rodents eat felines.

	 \ansk{\p{\lnot\lthereis x (Rx \land \lthereis y (Fy \land Exy))}}

	\item Kitty does not eat rodents.

	 \ansk{\p{\lnot\lthereis x (Rx \land Ekx)}}

	\item If something is a rodent, Kitty does not eat it.

	 \ansk{\p{\lforall x (Rx \limplies \lnot Ekx)}}

	\item Some rodents are cute.

	 \ansk{\p{\lthereis x (Rx \land Cx)}}

	\item Some cute things eat cute things.

	 \ansk{\p{\lthereis x \lthereis y [Cx \land (Cy \land Exy)]}}

   \end{enumerate}

\newpage

  \item Let the domain of discourse be all 5C (Pomona, Pitzer, Scripps, Harvey 
   Mudd, CMC) students.  Let \p{Fx} mean that x is a Pomona student, \p{Gx} mean 
   that x is a Scripps student, \p{Hx} mean that x is a Harvey Mudd student, 
   \p{Jx} mean that x is a Pitzer student.  Let's also say no one attends two 5C 
   colleges at once (I think that's true),  every 5C college has some students.  
   Let \p{Px} mean that x is currently taking PHIL60, and let's say that PHIL60 
   has a mix of 5C students except CMC students.  Finally,
	let \p{Sxy} mean that x and y are taking the same class. 

   Given this interpretation, indicate for each of the following sentences 
   whether or not it is true.

   \begin{enumerate}

   \setlength{\itemsep}{1em}


   \item \lbh{KXxGxXxZx}

	\ansT

	\coverbox{The sentence says that there are Scripps students and there are 
	Pitzer students.}

   \item \lbh{XxKFxHx}

	\ansF

	\coverbox{The sentence says that there is someone who is both a Pomona and a 
	Mudd student.}



   \item \lbh{UxCFxZx}

	 \ansF

	 \coverbox{The sentence says that every Pomona student is a Pitzer student.}

	\item \lbh{UxCGxNHx}

	 \ansT

	 \coverbox{The sentence says that anyone who is a Scripps student is not a Mudd 
	 student.}

	\item \lbh{XxKZxHx}

	 \ansF

	 \coverbox{The sentences says that there is someone who is both a Pitzer 
	 student and a Mudd student.}

	\item \lbh{XxKPxFx}

	 \ansT

	 \coverbox{The sentence says that there is someone who is taking PHIL60 and is 
	  a Pomona student. Or, more colloquially, there is a Pomona student who is 
	 taking PHIL60.}

	\item \lbh{XyKPyFy}

	 \ansT

	 \coverbox{This means exactly the same thing as the previous sentence.}

	\item \lbh{NUxCPxHx}

	 \ansT

	 \coverbox{The sentence says that not everyone who is taking PHIL60 is a Mudd 
	 student.}

	\item \lbh{UxUyCKPxPySxy}

	 \ansT

	 \coverbox{The sentence says that if any two people are taking PHIL60, then 
	 they are taking the same course.}

	\item \lbh{UxUyCSxyKPxPy}

	 \ansF

	 \coverbox{The sentence says that if two people are taking the same course, 
	 then they are both taking PHIL60.}

	\item \lbh{UxAFxAGxAHxZx}

	 \ansF

	 \coverbox{The sentence says that everyone is a Pomona, Scripps, Mudd, or 
	 Pitzer student. (That's false since the domain is all the 5C students.)}


	\item \lbh{XxNAFxAGxAHxZx}

	 \ansT

	 \coverbox{The sentence says that there is someone who is not attending Pomona, 
	 Scripps, Mudd, or Pitzer.}

	\item \lbh{XxKNFxNGx}

	 \ansT

	 \coverbox{The sentence says that some people are neither Pomona nor Scripps 
	 students.}

	\item \lbh{XxKKNFxNGxPx}

	 \ansT

	 \coverbox{The sentence says that there are some who are neither Pomona nor 
	 Scripps students who are taking PHIL60.}

	\item \lbh{UxUyCKGxHyNSxy}

	 \ansF

	 \coverbox{The sentence says that no Scripps and Mudd students are taking the 
	 same class.}

	\item \lbh{XxXyKKFxZySxy}

	 \ansT

	 \coverbox{The sentence says that some Pomona and Pitzer students are taking 
	 the same class.}

	\item \lbh{UzCPzAFzAGzAHzZz}

	 \ansT

	 \coverbox{The sentence says that anyone taking PHIL60 is a Pomona, Scripps, 
	 Mudd, or Pitzer student.}


	\item \lbh{UxCPxXyKZySxy}

	 \ansT

	 \coverbox{The sentence says that anyone who is taking PHIL60 is taking the 
	 same class with some Pitzer student.}

	\item \lbh{UxXyCPxKZySxy}

	 \ansT

	 \coverbox{This says the same as the previous sentence.}

	\item \lbh{NXxKPxNAFxAGxAHxZx}

	 \ansT

	 \coverbox{The sentence says that there is no one who is taking PHIL60 but 
	 isn't a Pomona, Scripps, Mudd, or Pitzer student.}

	\item \lbh{NXxKPxKNFxKNGxKNHxNZx}

	 \ansT

	 \coverbox{This is saying the same as the previous sentence.}

	 

   \end{enumerate}

\newpage


  \item For each of the following sentences, create an interpretation that makes the 
   sentence true.
\coverbox{
My answers are only examples. They are kept very simple to give you a sense of 
how to produce models without too much thought.}


   \begin{enumerate}

	\item \p{Gb}

	 \opts{

	  \dotline
	  \dotline
	  \dotline
	  \dotline
	  \dotline

	 }
	 {
	  \begin{itemize}

	  \item Domain: Just Bernie

	  \item Extension of \p{G}: Bernie

	  \item Referent of \p{b}: Bernie
	  \end{itemize}


	 }

	\item \p{Gc \limplies Fc}

	 \opts{

	  \dotline
	  \dotline
	  \dotline
	  \dotline
	  \dotline

	 }
	 {
	\begin{itemize}

	 \item Domain: Just Charlie

	 \item Extension of \p{G}: empty

	 \item Extension of \p{F}: empty

	 \item Referent of \p{c}: Charlie

	\end{itemize}
	 }

	\item \lbh{KCGcFcNFd}

	 \opts{

	  \dotline
	  \dotline
	  \dotline
	  \dotline
	  \dotline

	 }
	 {

	 \begin{itemize}

	  \item Domain: Charlie and Dannie

	  \item Extension of F: Charlie

	  \item Extension of G: Charlie

	  \item Referent c: Charlie

	  \item Referent of d: Dannie

	\end{itemize}

}


	\item \lbh{XxFx}

	 \opts{

	  \dotline
	  \dotline
	  \dotline
	  \dotline
	  \dotline

	 }
	 {

	 \begin{itemize}

	 \item Domain: Taylor

	\item Extension of F: Taylor

  \end{itemize}

 }

	\item \lbh{XxGx}

	 \opts{

	  \dotline
	  \dotline
	  \dotline
	  \dotline
	  \dotline

	 }
	 {
	\begin{itemize}

	 \item Domain: Britney

	 \item Extension of G: Britney

	\end{itemize}

	 }

	\item \lbh{KXxFxXxGx}

	 \opts{

	  \dotline
	  \dotline
	  \dotline
	  \dotline
	  \dotline

	 }
	 {
	
	  \begin{itemize}

	   \item Domain: Taylor and Britney

	   \item Extension of F: Taylor

	   \item Extension of G: Britney

	  \end{itemize}
	 

	 }

	\item \lbh{KXxFxNXxGx}

	 \opts{

	  \dotline
	  \dotline
	  \dotline
	  \dotline
	  \dotline

	 }
	 {

	\begin{itemize}

	 \item Domain: Lexi

	 \item Extension of F: Lexi

	 \item Extension of G: Empty

	\end{itemize}

	 }



	\item \lbh{UxCGxFx}

	 \opts{

	  \dotline
	  \dotline
	  \dotline
	  \dotline
	  \dotline

	 }
	 {

	  \begin{itemize}

	   \item Domain: Venus

	   \item Extension of G: Venus

	   \item Extension of F: Venus

   \end{itemize}

  }



	\item \lbh{NUxCGxFx}

	 \opts{

	  \dotline
	  \dotline
	  \dotline
	  \dotline
	  \dotline

	 }
	 {
	  \begin{itemize}

	   \item Domain: Venus and Pluto

	  \item Extension of G: Venus and Pluto

	 \item Extension of F: Venus

   \end{itemize}}

	\item \lbh{NXxAFxGx}

	 \opts{

	  \dotline
	  \dotline
	  \dotline
	  \dotline
	  \dotline

	 }
	 {

	  \begin{itemize}

	   \item Domain: Ivy

	  \item Extension of F: empty

	 \item Extension of G: empty

   \end{itemize}
  }



 \end{enumerate}


\end{enumerate}







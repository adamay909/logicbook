\section*{Extra Exercises for Derivations/Proofs}

\begin{enumerate}
 \setlength{\itemsep}{2em}

 \item For each of the following short derivations, explain why they are 
  illegal. (1pt each)

  \begin{enumerate}
   \setlength{\itemsep}{1.5em}
 \item 

  \begin{argument*}
   \ai{p}{q}{premise}
   \ai{p}{q\land r}{1,\conjI}
  \end{argument*}

  \opts{
  \dotline
  \dotline
  \dotline
 }
 {You need to refer to two lines to use \conjI.
 }
  \item

   \begin{argument*}
	\ai{p\land p}{r}{premise}
	\ai{p}{r}{1,\conjE}
   \end{argument*}

   \opts{
  \dotline
  \dotline
 \dotline}
 {
  \conjE{} cannot change datum.
 }

   \item

   \begin{argument*}
	\ai{p\land p}{r}{premise}
	\ai{p}{r}{1}
   \end{argument*}

   \opts{
  \dotline
  \dotline
 \dotline}{

  Sequent rewrite cannot eliminate a conjunct in the datum.
 }

  \item 

   \begin{argument*}
	\ai{\Gamma}{s\lor t}{premise}
	\ai{\Gamma}{s}{1,\disjE}
   \end{argument*}

   \opts{
  \dotline
  \dotline
 \dotline}{

 \disjE{} does not work that way.}


   \item

	\begin{argument*}
	 \ai{s}{p\limplies q}{premise}
	 \ai{s\lor t}{p\limplies q}{1,\disjI}
	\end{argument*}

	\opts{
  \dotline
  \dotline
 \dotline}
{
 Cannot use \disjI{} to change datum.
}

\newpage	\item

	 \begin{argument*}
	  \ai{\Gamma,p}{q}{premise}
	  \ai{\Delta,p}{r}{premise}
	  \ai{\Gamma,\Delta}{\lnot p}{1,2,\negI}
	 \end{argument*}

	 \opts{
  \dotline
  \dotline
 \dotline}
 {
  Lines 1 and 2 do not contradicting succedents.
 }

	\item 

	 \begin{argument*}
		\ai{\Gamma,p}{q}{premise}
		\ai{\Delta,\lnot p}{q}{premise}
		\ai{\Gamma,\Delta}{\lnot q}{1,2,\negI}
	 \end{argument*}

	 \opts{
  \dotline
  \dotline
 \dotline}{

  \negI{} misapplied. Contradiction must be on  the succedent side.
 }
	 \item

	  \begin{argument*}
		\ai{\Delta}{p\limplies q}{premise}
		\ai{\Delta,p}{q}{1,\condE}
	   \end{argument*}
\opts{
  \dotline
  \dotline
 \dotline}
 {

 Need another line to use \condE.}

	  \item 

	   \begin{argument*}
		\ai{\Delta}{p\limplies q}{premise}
		\ai{\Gamma}{q}{premise}
		\ai{\Delta,\Gamma}{p}{\condE}
	   \end{argument*}

	   \opts{
  \dotline
  \dotline
 \dotline}{

  \condE{} misapplied. You need the antecedent of the conditional in the 
  succedent of second sequent.


 }

	  \item 

	   \begin{argument*}
	   \ai{\Delta,s}{t}{premise}
	   \ai{\Delta}{t\limplies s}{1,\condI}
	  \end{argument*}
\opts{
  \dotline
  \dotline
  \dotline
 }{
  \condI{} misapplied. The item from the datum goes into the antecedent of the 
 conditional.}



	\end{enumerate}

 \item Here is a proof of Excluded Middle that's different from the one given in 
  the readings. Add any missing datums:



\begin{argumentN}[1]
%generated by gentzen

 \ai{\mask{p\mc{\land }\lnot p}}{p\mc{\land }\lnot p}{A}

 \ai{\mask{p\mc{\land }\lnot p}}{p}{1,\conjE}

 \ai{\mask{p\mc{\land }\lnot p}}{\mc{\lnot }p}{1,\conjE}

 \ai{\mask{}}{\mc{\lnot }(p\land \lnot p)}{2,3,\negI}

\ai{\mask{\mc{\lnot }(p\lor \lnot p)}}{\mc{\lnot }(p\lor \lnot p)}{A}

\ai{\mask{p}}{p}{A}

\ai{\mask{p}}{p\mc{\lor }\lnot p}{6,\disjI}

\ai{\mask{\mc{\lnot }(p\lor \lnot p), p}}{\mc{\lnot }(p\lor \lnot p)}{5}

\ai{\mask{\mc{\lnot }(p\lor \lnot p)}}{\mc{\lnot }p}{7,8,\negI}

\ai{\mask{\mc{\lnot }p}}{\mc{\lnot }p}{A}

\ai{\mask{\mc{\lnot }p}}{p\mc{\lor }\lnot p}{10,\disjI}

\ai{\mask{\mc{\lnot }(p\lor \lnot p), \mc{\lnot }p}}{\mc{\lnot }(p\lor \lnot p)}
{5}

\ai{\mask{\mc{\lnot }(p\lor \lnot p)}}{\mc{\lnot }\lnot p}{11,12,\negI}

\ai{\mask{\mc{\lnot }(p\lor \lnot p)}}{p}{13,\negE}

\ai{\mask{\mc{\lnot }(p\lor \lnot p)}}{p\mc{\land }\lnot p}{9,14,\conjI}

\ai{\mask{\mc{\lnot }(p\lor \lnot p)}}{\mc{\lnot }(p\land \lnot p)}{4}

\ai{\mask{}}{\mc{\lnot }\lnot (p\lor \lnot p)}{15,16,\negI}

\ai{\mask{}}{p\mc{\lor }\lnot p}{17,\negE}

\end{argumentN}

\newpage


 \item Add missing annotations:


\begin{argumentN}[1]
%generated by gentzen

\ai{Γ}{P\mc{\lor }Q}{premise}

\ai{Δ}{P\mc{\limplies }R}{premise}

\ai{Θ}{Q\mc{\limplies }S}{premise}

\ai{P}{P}{\mask{A}}

\ai{Δ, P}{R}{\mask{2,4,\condE}}

\ai{Δ, P}{R\mc{\lor }S}{\mask{5,\disjI}}

\ai{Q}{Q}{\mask{A}}

\ai{Θ, Q}{S}{\mask{3,7,\condE}}

\ai{Θ, Q}{R\mc{\lor }S}{\mask{8,\disjI}}

\ai{Γ, Δ, Θ}{R\mc{\lor }S}{\mask{1,6,9,\disjE}}

\end{argumentN}

\item Fill in the missing items.

\begin{argumentN}[1]
%generated by gentzen

\ai{Γ}{P\mc{\lor }(Q\lor R)}{premise}

\ai{P}{P}{A}

\ai{\mask{R}}{\mask{R}}{A}

\ai{R}{(P\lor Q)\mc{\lor }R}{3,\mask{\disjI}}

\ai{\mask{P}}{\mask{P\mc{\lor }Q}}{2,\disjI}

\ai{\mask{Q}}{\mask{Q}}{A}

\ai{Q}{\mask{P\mc{\lor }Q}}{6,\mask{\disjI}}

\ai{\mask{P}}{\mask{(P\lor Q)\mc{\lor }R}}{5,\disjI}

\ai{Q\mc{\lor }R}{Q\mc{\lor }R}{A}

\ai{\mask{Q}}{(P\lor Q)\mc{\lor }R}{7,\disjI}

\ai{\mask{Q\mc{\lor }R}}{\mask{(P\lor Q)\mc{\lor }R}}{\mask{4,9,10,\disjE}}

\ai{Γ}{(P\lor Q)\mc{\lor }R}{1,8,11,\disjE}

\end{argumentN}

\newpage
\item The following contains one illegal move. Where is it?

\begin{argumentN}[1]
%generated by gentzen

\ai{Γ}{W\mc{\lor }S}{premise}

\ai{Δ}{\mc{\lnot }W}{premise}

\ai{S}{S}{A}

\ai{S, \mc{\lnot }W}{S}{3}

\ai{S}{\lnot W\mc{\limplies }S}{4,\condI}

\ai{Δ, S}{S}{2,5,\condE}

\ai{W}{W}{A}

\ai{\mc{\lnot }W}{\mc{\lnot }W}{A}

\ai{W, \mc{\lnot }S}{W}{7}

\ai{\mc{\lnot }W, \mc{\lnot }S}{\mc{\lnot }W}{8}

\ai{W, \mc{\lnot }W}{\mc{\lnot }\lnot S}{9,10,\negI}

\ai{W}{\lnot W\mc{\limplies }\lnot \lnot S}{11,\condI}

\ai{W}{\lnot W\mc{\limplies }S}{12,\negE}

\ai{Δ, W}{S}{2,13,\condE}

\ai{Γ, Δ}{S}{1,6,14,\disjE}

\end{argumentN}
\answer{

 The move to line 13 is not allowed by our inference rules. Notice that the main 
 connective of the succedent of line 12 is the conditional.
}

\item  Fill in the missing items.


\begin{argumentN}[1]
%generated by gentzen

\ai{Γ}{W\mc{\lor }(S\limplies\lnot T)}{premise}

\ai{Δ}{\mc{\lnot }W}{premise}

\ai{\mask{W}}{\mask{W}}{A}

\ai{\mask{Δ, \mc{\lnot }(S\limplies\lnot T)}}{\mask{\mc{\lnot }W}}{2}

\ai{\mask{W, \mc{\lnot }(S\limplies\lnot T)}}{\mask{W}}{3}

\ai{\mask{Δ, W}}{\mask{\mc{\lnot }\lnot (S\limplies\lnot T)}}{4,5,\negI}

\ai{\mask{Δ, W}}{\mask{(S\limplies\lnot T)}}{6,\negE}

\ai{\mask{(S\limplies\lnot T)}}{\mask{(S\limplies\lnot T)}}{A}

\ai{Γ, Δ}{S\limplies\lnot T}{1,7,8,\disjE}

\end{argumentN}

\newpage

\item Here is yet another proof of Excluded Middle. Fill in any missing items.


\begin{argumentN}[1]
%generated by gentzen

\ai{\mc{\lnot }(p\lor \lnot p)}{\mc{\lnot }(p\lor \lnot p)}{A}

\ai{\mc{\lnot }p}{\mc{\lnot }p}{A}

\ai{\mask{\mc{\lnot }p}}{\mask{p\mc{\lor }\lnot p}}{2,\disjI}

\ai{\mask{\mc{\lnot }(p\lor \lnot p), \mc{\lnot }p}}{\mask{\mc{\lnot }(p\lor 
\lnot p)}}
 {1}

\ai{\mask{\mc{\lnot }(p\lor \lnot p)}}{\mask{\mc{\lnot }\lnot p}}{3,4,\negI}

\ai{p}{p}{A}

\ai{\mask{p}}{\mask{p\mc{\lor }\lnot p}}{6,\disjI}

\ai{\mask{\mc{\lnot }(p\lor \lnot p), p}}{\mask{\mc{\lnot }(p\lor \lnot p)}}{1}

\ai{\mask{\mc{\lnot }(p\lor \lnot p)}}{\mask{\mc{\lnot }p}}{7,8,\negI}

\ai{\mask{}}{\mask{\mc{\lnot }\lnot (p\lor \lnot p)}}{5,9,\negI}

\ai{}{p\mc{\lor }\lnot p}{10,\negE}

\end{argumentN}

%\newpage

\item Prove the following sequents (keep in mind what it means to prove a 
 sequent). Each of these require only three lines:

 \begin{enumerate}
  \cover{\setlength{\itemsep}{2em}}
  \item \p{\seq{p,q}{p\lor q}}
\opts{
   \dotline

   \dotline

   \dotline

   \dotline
  }
  {

\begin{argumentN}[1]
%generated by gentzen

\ai{p}{p}{A}

\ai{p}{p\mc{\lor }q}{1,\disjI}

\ai{p, q}{p\mc{\lor }q}{2}

\end{argumentN}

  }


  \item \p{\seq{\lnot p,q}{p\lor q}}
\opts{
   \dotline

   \dotline

   \dotline

   \dotline
  }
  {
\begin{argumentN}[1]
%generated by gentzen

\ai{q}{q}{A}

\ai{q}{p\mc{\lor }q}{1,\disjI}

\ai{\mc{\lnot }p, q}{p\mc{\lor }q}{2}

\end{argumentN}

}

  \item \p{\seq{p,\lnot q}{p\lor q}}
\opts{
   \dotline

   \dotline

   \dotline

   \dotline
  }
  {

\begin{argumentN}[1]
%generated by gentzen

\ai{p}{p}{A}

\ai{p}{p\mc{\lor }q}{1,\disjI}

\ai{p, \lnot q}{p\mc{\lor }q}{2}

\end{argumentN}
\newpage
  }

 \end{enumerate}
\newpage

\item Fill in the missing items of the following proof of \p{\seq{\lnot p, \lnot 
 q}{\lnot(p\lor q)}}:

\begin{argumentN}[1]
%generated by gentzen

\ai{\mc{\lnot }p}{\mc{\lnot }p}{A}

\ai{\mc{\lnot }q}{\mc{\lnot }q}{A}

\ai{p\mc{\lor }q}{p\mc{\lor }q}{A}

\ai{p}{p}{A}

\ai{\mask{\mc{\lnot }p, p\mc{\lor }q}}{\mask{\mc{\lnot }p}}{1}

\ai{\mask{p, p\mc{\lor }q}}{\mask{p}}{4}

\ai{\mc{\lnot }p, p}{\mc{\lnot }(p\lor q)}{5,6,\negI}

\ai{q}{q}{A}

\ai{\mask{q, p\mc{\lor }q}}{\mask{q}}{8}

\ai{\mask{\mc{\lnot }q, p\mc{\lor }q}}{\mask{\mc{\lnot }q}}{2}

\ai{\mask{\mc{\lnot }q, q}}{\mc{\lnot }(p\lor q)}{9,10,\negI}

\ai{p\mc{\lor }q, \mc{\lnot }p, \mc{\lnot }q}{\mc{\lnot }(p\lor q)}{\mask{3,7,11,
\disjE}}

\ai{\mc{\lnot }p, \mc{\lnot }q}{\mc{\lnot }(p\lor q)}{\mask{3,12,\negI}}

\end{argumentN}


\newpage

\item Prove Implication: \lbh{-s :CCpqANpq}. Hint: you can adapt one of the 
 derivations we have done earlier in our exercises.
\opts{
 \dotline

 \dotline

 \dotline

 \dotline

 \dotline

 \dotline

 \dotline

 \dotline

 \dotline

 \dotline

 \dotline

 \dotline

 \dotline

 \dotline

 \dotline

 \dotline

 \dotline

 \dotline

 \dotline
}
{
\answer{

 You can adapt a derivation we have done earlier in our exercises.

 \begin{argument*}
  \renewcommand{\mask}[1]{#1}
\ai[0.3]{p\limplies q}{p \limplies q}{A}

\ai[0.3]{\lnot (\lnot p \lor q)}{\mask{\lnot (\lnot p \lor q)}}{A}

\ai[0.3]{\mask{\lnot p}}{\lnot p}{A}

\ai[0.3]{\lnot p}{\lnot p \lor q}{\mask{3,\disjI}}

\ai[0.3]{\mask{\lnot (\lnot p \lor q),\lnot p}}{\mask{\lnot (\lnot p \lor q)}}{2}

\ai[0.3]{\lnot (\lnot p \lor q)}{\lnot \lnot p}{4,5,\negI}

\ai[0.3]{\mask{\lnot (\lnot p \lor q)}}{\mask{p}}{6,\negE}

\ai[0.3]{\mask{p\limplies q,\lnot (\lnot p \lor q)}}{\mask{q}}{1,7,\condE}

\ai[0.3]{p\limplies q,\lnot (\lnot p \lor q)}{\lnot p \lor q}{8,\disjI}

\ai[0.3]{p\limplies q}{\lnot \lnot (\lnot p \lor q)}{\mask{2,9,\negI}}

\ai[0.3]{p\limplies q}{\lnot p \lor q}{\mask{10,\negE}}

\ai[0.3]{}{(p\limplies q)\limplies (\lnot p \lor q)}{12,\condI}
\end{argument*}
}
}
 \newpage


\item Prove Contraposition. Hint: you can adapt one of the derivations in the 
 earlier exercises.
\opts{
 \dotline

 \dotline

 \dotline

 \dotline

 \dotline

 \dotline

 \dotline

 \dotline

 \dotline

 \dotline

 \dotline

 \dotline

 \dotline

 \dotline

 \dotline

 \dotline

 \dotline

 \dotline

 \dotline
}
{
\answer{

 You can adapt a derivation in one of our earlier exercises.

\begin{argument*}

\ai{p\limplies q}{p \limplies q}{A}

\ai{\lnot q}{\lnot q}{A}

\ai{p}{p}{A}

\ai{p\limplies q,p}{q}{1,3,\condE}

\ai{\lnot q,p}{\lnot q}{2}

\ai{p\limplies q,\lnot q}{\lnot p}{4,5,\negI}

\ai{p\limplies q}{\lnot q \limplies \lnot p}{6,\condI}

\ai{}{(p\limplies q)\limplies (\lnot q\limplies \lnot p)}{7,\condI}
\end{argument*}
}
}

 \newpage
\item Derive from \lbh{-s /G:NANPNQ} to \lbh{-s /G:KPQ}. Hint: you can adapt one 
 of our earlier exercises.
\opts{
 \dotline

 \dotline

 \dotline

 \dotline

 \dotline

 \dotline

 \dotline

 \dotline

 \dotline

 \dotline

 \dotline

 \dotline

 \dotline

 \dotline

 \dotline

 \dotline

 \dotline

 \dotline

 \dotline
}
{
\answer{


\begin{argumentN}[1]
%generated by gentzen

\ai{Γ}{\mc{\lnot }(\lnot P\lor \lnot Q)}{premise}

\ai{\lnot P}{\lnot P}{A}

\ai{\lnot P}{\lnot P\mc{\lor }\lnot Q}{2,\disjI}

\ai{Γ, \lnot P}{\mc{\lnot }(\lnot P\lor \lnot Q)}{1}

\ai{Γ}{\mc{\lnot }\lnot P}{3,4,\negI}

\ai{\Gamma}{P}{5,\negE}

\ai{\lnot Q}{\lnot Q}{A}

\ai{\lnot Q}{\lnot P\mc{\lor }\lnot Q}{7,\disjI}

\ai{Γ, \lnot Q}{\mc{\lnot }(\lnot P\lor \lnot Q)}{1}

\ai{Γ}{\mc{\lnot }\lnot Q}{8,9,\negI}

\ai{\Gamma}{Q}{10,\negE}

\ai{Γ}{P\mc{\land }Q}{6,11,\conjI}

\end{argumentN}
}}
 \newpage
\item Prove \lbh{-s :CNKpqANpNq}. Hint: the overall proof is a proof by 
 contradiction (assume the negation of the consequent of the conditional), and 
 the previous question shows how to do most of the needed work.
\opts{
 \dotline

 \dotline

 \dotline

 \dotline

 \dotline

 \dotline

 \dotline

 \dotline

 \dotline

 \dotline

 \dotline

 \dotline

 \dotline

 \dotline

 \dotline

 \dotline

 \dotline

 \dotline

 \dotline
}
{
 \answer{

Notice that lines 2 through 13 replicate the previous derivation where we plug 
\p{\lnot(\lnot p \lor\lnot q)} into \p{\Gamma}.

\begin{argumentN}[1]
%generated by gentzen

\ai[0.35]{\mc{\lnot }(p\land q)}{\mc{\lnot }(p\land q)}{A}

\ai[0.35]{\mc{\lnot }(\lnot p\lor \lnot q)}{\mc{\lnot }(\lnot p\lor \lnot q)}{A}

\ai[0.35]{\mc{\lnot }p}{\mc{\lnot }p}{A}

\ai[0.35]{\mc{\lnot }p}{\lnot p\mc{\lor }\lnot q}{3,\disjI}

\ai[0.35]{\mc{\lnot }(\lnot p\lor \lnot q), \mc{\lnot }p}{\mc{\lnot }(\lnot p\lor \lnot q)}{2}

\ai[0.35]{\mc{\lnot }(\lnot p\lor \lnot q)}{\mc{\lnot }\lnot p}{4,5,\negI}

\ai[0.35]{\mc{\lnot }(\lnot p\lor \lnot q)}{p}{6,\negE}

\ai[0.35]{\mc{\lnot }q}{\mc{\lnot }q}{A}

\ai[0.35]{\mc{\lnot }q}{\lnot p\mc{\lor }\lnot q}{8,\disjI}

\ai[0.35]{\mc{\lnot }(\lnot p\lor \lnot q), \mc{\lnot }q}{\mc{\lnot }(\lnot p\lor \lnot q)}{2}

\ai[0.35]{\mc{\lnot }(\lnot p\lor \lnot q)}{\mc{\lnot }\lnot q}{9,10,\negI}

\ai[0.35]{\mc{\lnot }(\lnot p\lor \lnot q)}{q}{11,\negE}

\ai[0.35]{\mc{\lnot }(\lnot p\lor \lnot q)}{p\mc{\land }q}{7,12,\conjI}

\ai[0.35]{\mc{\lnot }(p\land q), \mc{\lnot }(\lnot p\lor \lnot q)}{\mc{\lnot }(p\land q)}{1}

\ai[0.35]{\mc{\lnot }(p\land q)}{\mc{\lnot }\lnot (\lnot p\lor \lnot q)}{13,14,\negI}

\ai[0.35]{\mc{\lnot }(p\land q)}{\lnot p\mc{\lor }\lnot q}{15,\negE}

\end{argumentN}

}}


 \newpage

\item Construct a derivation from \lbh{-s /G:APQ} and \lbh{-s /D:CQR} 
 to \lbh{-s /G,/D:APR}. (Here is an example of this in plain English: Joe 
 is eating Chinese place or Italian. If he is eating Italian, he is eating 
 eggplant parmigiana. So Joe is eating Chinese or he is eating eggplant 
 parmigiana.) Hint: \disjE{} is your friend.
\opts{
 \dotline

 \dotline

 \dotline

 \dotline

 \dotline

 \dotline

 \dotline

 \dotline

 \dotline

 \dotline

 \dotline

 \dotline

 \dotline

 \dotline

 \dotline

 \dotline

 \dotline

 \dotline

 \dotline
}
{
 \answer{


\begin{argumentN}[1]
%generated by gentzen

\ai{Γ}{P\mc{\lor }Q}{premise}

\ai{Δ}{Q\mc{\limplies }R}{premise}

\ai{P}{P}{A}

\ai{P}{P\mc{\lor }R}{3,\disjI}

\ai{Q}{Q}{A}

\ai{Δ, Q}{R}{2,5,\condE}

\ai{Δ, Q}{P\mc{\lor }R}{6,\disjI}

\ai{Γ, Δ}{P\mc{\lor }R}{1,4,7,\disjE}

\end{argumentN}


 }
}

 \newpage
\item Prove Double Negation Introduction. Hint: use \negI{}.
\opts{
 \dotline

 \dotline

 \dotline

 \dotline

 \dotline

 \dotline

 \dotline
}
{

 \answer{

\begin{argumentN}[1]
%generated by gentzen

\ai{p}{p}{A}

\ai{\mc{\lnot }p}{\mc{\lnot }p}{A}

\ai{p, \mc{\lnot }p}{p}{1}

\ai{p}{\mc{\lnot }\lnot p}{2,3,\negI}

\end{argumentN}
}
\vskip 2em
}

\item In a single proof, prove \p{\seq{p,q}{p\limplies q}} and \p{\seq{\lnot p,
 q}{p\limplies q}} (recall that a proof proves each sequent; apart from 
 Assumption Introduction, you only need \condI{} and sequent rewrites).
\opts{
 \dotline

 \dotline

 \dotline

 \dotline

 \dotline

 \dotline

 \dotline
}
{
\answer{
\begin{argumentN}[1]
%generated by gentzen

\ai{q}{q}{A}

\ai{q, p}{q}{1}

\ai{q}{p\mc{\limplies }q}{2,\condI}

\ai{p, q}{p\mc{\limplies }q}{3}

\ai{\mc{\lnot }p, q}{p\mc{\limplies }q}{3}

\end{argumentN}

}

}
\item Prove \p{\seq{p,\lnot q}{\lnot(p\limplies q)}}. Hint: assume \p{p\limplies 
 q}, \p{p}, and \p{\lnot q}.
\opts{
 \dotline

 \dotline

 \dotline

 \dotline

 \dotline

 \dotline

 \dotline
}
{
\answer{
\begin{argumentN}[1]
%generated by gentzen

\ai{p\mc{\limplies }q}{p\mc{\limplies }q}{A}

\ai{p}{p}{A}

\ai{\mc{\lnot }q}{\mc{\lnot }q}{A}

\ai{p\mc{\limplies }q, p}{q}{1,2,\condE}

\ai{p\mc{\limplies }q, \mc{\lnot }q}{\mc{\lnot }q}{3}

\ai{p, \mc{\lnot }q}{\mc{\lnot }(p\limplies q)}{4,5,\negI}

\end{argumentN}
}
\newpage
}
\newpage

\item Prove \p{\seq{\lnot p, \lnot q}{p\limplies q}}. Hint: adapt the derivation 
 from \p{\seq{\Gamma}{P}} to \p{\seq{\Gamma}{\lnot P \limplies Q}}.
\opts{
 \dotline

 \dotline

 \dotline

 \dotline

 \dotline
 
 \dotline

 \dotline

 \dotline

 \dotline

 \dotline
}
{
\answer{

 Notice that \p{p} is equivalent to \p{\lnot(\lnot p)}.

\begin{argumentN}[1]
%generated by gentzen

\ai{\mc{\lnot }p}{\mc{\lnot }p}{A}

\ai{\mc{\lnot }p, \mc{\lnot }q}{\mc{\lnot }p}{1}

\ai{p}{p}{A}

\ai{p, \mc{\lnot }q}{p}{3}

\ai{\mc{\lnot }p, p}{\mc{\lnot }\lnot q}{2,4,\negI}

\ai{\mc{\lnot }p, p}{q}{5,\negE}

\ai{\mc{\lnot }p}{p\mc{\limplies }q}{6,\condI}

\ai{\mc{\lnot }p, \mc{\lnot }q}{p\mc{\limplies }q}{7}

\end{argumentN}

}
}
\item Derive from \p{\seq{p,q}{r}} to \p{\seq{}{(p\land q)\limplies r}}.
\opts{
 \dotline

 \dotline

 \dotline

 \dotline

 \dotline
 
 \dotline

 \dotline

 \dotline

 \dotline

 \dotline
}
{
\answer{



\begin{argumentN}[1]
%generated by gentzen

\ai{p, q}{r}{premise}

\ai{p\mc{\land }q}{p\mc{\land }q}{A}

\ai{p}{q\mc{\limplies }r}{1,\condI}

\ai{p\mc{\land }q}{q}{2,\conjE}

\ai{p\mc{\land }q, p}{r}{3,4,\condE}

\ai{p\mc{\land }q}{p\mc{\limplies }r}{5,\condI}

\ai{p\mc{\land }q}{p}{2,\conjE}

\ai{p\mc{\land }q}{r}{6,7,\condE}

\ai{}{(p\land q)\mc{\limplies }r}{8,\condI}

\end{argumentN}

}
}
\end{enumerate}

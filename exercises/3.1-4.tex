\subsection*{I.}

\begin{enumerate}

\item Consider this snippet: ``The only thing that matters for a business is 
 profitability and improving productivity is one obvious way to improve 
 profitability. But not all efforts to increase productivity are beneficial.  
 Often, attempts to increase productivity
decrease the number of employees which negatively affects the motivation of the 
retained employees.''

Which of the following is closest to the conclusion of this passage?

\begin{enumerate}

 \item Profitability is not the only thing that matters for a business.

 \item Employees should have a right not be dismissed.

 \item Some efforts to improve productivity harm profitability.

\end{enumerate}
\answer{

 (c)

}
\item Consider this snippet: ``From an ecological perspective, insects have a 
 lot to recommend them.  They are renowned for their small `foodprint'; being 
 cold-blooded, they are about four times as efficient at converting feed to meat 
 as are cattle, which waste energy keeping themselves warm. Ounce for ounce, 
 many have the same amount of protein as beef—fried grasshoppers have three 
 times as much—and are rich in micronutrients like iron and zinc. Genetically, 
 they are so distant from humans that there is little likelihood of diseases 
 jumping species, as swine flu did.  They are natural recyclers, capable of 
 eating old cardboard, manure, and by-products from food manufacturing. And 
 insect husbandry is humane: bugs like teeming, and thrive in filthy, crowded 
 conditions.'' (from \emph{The New Yorker}, Aug. 15/22, 2011)

 Which of the following is closest to the conclusion of this passage:

 \begin{enumerate}

  \item Insect husbandry is humane.

  \item Insects are delicious.

  \item Eating insects is environmentally friendly.

  \item Rather than becoming vegan, we should start eating insects.

  \item Insect husbandry is environmentally friendly and humane.

 \end{enumerate}

 \answer{

  (e). Reading the last sentence as a separate point from the first sentence of 
  the paragraph.

 }

\item Consider: `` We have everywhere tried to merge with the peoples around us,
and to only preserve the faith of our fathers. We are not allowed to do so. In 
vain are we loyal patriots, our loyalty in some places even running to extremes; 
in vain do we make the same sacrifices of property and blood as our
fellow-citizens; in vain do we strive to increase the fame of our
home countries in the arts and sciences, and their wealth by trade and commerce.
In our home-countries  where we have lived for centuries we are still decried as 
foreigners, and often by those whose families were not yet
present in the land when our fathers were already toiling there.'' (from Theodor 
Herzl, \emph{Der Judenstaat (The Jewish State)}. 1896. ``We'' are the Jewish 
people. If you don't know who Herzl was, look him up.)

What is the conclusion of this passage?

\begin{enumerate}

 \item More needs to be done to assimilate Jewish people into mainstream Europe.

 \item Jewish people must be integrated without denying their separate identity.

 \item Jews should pack up and leave.

 \item None of the above.

\end{enumerate}

\answer{

 Herzl himself is well-known to  have advocated something like (c), but the 
 passage merely lists some facts and does not draw any conclusions. Stefan Zweig, 
 another famous Austrian Jew and one-time prot\'eg\'e of Herzl lists the same 
 facts in his \emph{The World of Yesterday} but is adamantly opposed to Zionism.

}
\end{enumerate}

\subsection*{II.}

\begin{enumerate}

\item Upon meeting John Watson, Sherlock Holmes immediately figured out Watson 
 had returned from Afghanistan recently. Here is his explanation of how he 
 figured that out (from Conan Doyle, \emph{A Study in Scarlet}):
 \begin{quote}
   I knew you came from Afghanistan. From long habit the train of thoughts ran 
   so swiftly through my mind, that I arrived at the conclusion without being 
   conscious of intermediate steps. There were such steps, how­ever.  The train 
   of reasoning ran, `Here is a gentleman of a medical type, but with the air of 
   a military man. Clearly an army doc­tor, then. He has just come from the 
   tropics, for his face is dark, and that is not the natural tint of his skin, 
   for his wrists are fair. He has undergone hardship and sickness, as his 
   haggard face says clearly. His left arm has been injured.  He holds it in a 
   stiff and unnatural manner. Where in the tropics could an English army doctor 
  have seen much hard­ship and got his arm wounded?  Clearly in Afghanistan.' 
 \end{quote}

Put Holmes's reasoning that Watson has just come from the tropics as  an 
argument in standardized form.  

\answer{

 Notice that the part that matters is:

 ``He has just come from the tropics, for his face is dark, and that is not the 
 natural tint of his skin, for his wrists are fair.''

 \begin{argument*}

 \aitem If Watson's face is dark and his natural tint of skin is not dark, then 
 he has just come from the tropics.\texpl{premise}

 \aitem If Watson's wrists are fair, then his natural tint of skin is not 
 dark.\texpl{premise}

 \aitem Watson's wrists are fair.\texpl{premise}

 \aitem Watson's natural tint of skin is not dark.\texpl{from 2,3,MP}

 \aitem Watson's face is dark. \texpl{premise}

 \aitem Watson's face is dark and Watson's natural tint of skin is not 
 dark.\texpl{from 4,5, ??}

 \aergo Watson has just come from the tropics. \texpl{from 1,6,MP}


\end{argument*}

}



\item Here is a problem known as the Prisoner's Dilemma:

 \begin{quote}
Ben and Jerry have been arrested for robbing the Sleepytown Savings Bank. They 
have been placed in separate isolation cells. The prosecutor makes the following 
offer to each: ``You may choose to confess or remain silent.  If you confess and 
your accomplice remains silent I will drop all charges against you and use your 
testimony to ensure that your accomplice does serious time.  Likewise, if your 
accomplice confesses while you remain silent, they will go free while you do the 
time. If you both confess I get two convictions, but I'll see to it that you 
both get early parole. If you both remain silent, I'll have to charge you both 
with the lesser crime of firearms possession.  Think about whether or not you 
want to confess, and let me know by tomorrow morning.'' What should Ben do 
assuming he is only interested in getting the best deal for himself?
\end{quote}

Ben should confess because whatever happens Ben will be better off 
confessing.  Here is the reasoning:

\begin{quote}

Jerry will either confess or remain silent. If Jerry confesses, Ben is 
better off confessing too (because Ben  will have to do serious time  
otherwise).  If Jerry remains silent, Ben is better off confessing (because all 
charges against him will be dropped). So either way Ben is better off 
confessing.

\end{quote}


Put this reasoning in standardized form by using Argument By Cases.

\answer{
 \begin{argument*}
  
  \aitem Jerry will confess or Jerry will remain silent.  \texpl{premise}

  \aitem If Jerry confesses, then Ben is better off confessing. \texpl{premise} 

  \aitem If Jerry remains silent, then Ben is better off confessing.  
  \texpl{premise}

  \aergo Ben is better off confessing. \texpl{from 1,2,3,AC}
 \end{argument*}
 }

\item Descartes---by common acclaim the founder of modern philosophy---argues 
 that while he may be ignorant about many things, he can be absolutely certain 
 that he himself exists. Here is his argument early in his Second Meditation:

 \begin{quote}
  I have convinced myself that there is absolutely nothing in the world, no sky, 
  no earth, no minds, no bodies. Does it now follow that I too do not exist? No: 
  if I convinced myself of something then I certainly existed. But there is a 
  deceiver of supreme power and cunning who is deliberately and constantly 
  deceiving me. In that case I too undoubtedly exist, if he is deceiving me; and 
  let him deceive me as much as he can, he will never bring it about that I am 
  nothing so long as I think that I am something. So after considering 
  everything very thoroughly, I must finally conclude that this proposition, I 
  am, I exist, is necessarily true whenever it is put forward by me or conceived 
  in my mind.
 \end{quote}

 Put this argument in standardized form. The argument form is Reductio Ad 
 Absurdum, and the two crucial sentences involved are: \pp{I am being deceived 
 about my own existence}, and \pp{I exist}. The conclusion is the negation of 
 \pp{I am being deceived about my own existence}.

 \answer{

  \begin{argument*}
  \aitem If I am being deceived about my own existence, then I do not 
  exist.\texpl{premise}

  \aitem If I am being deceived about my own existence, then I exist.  
  \texpl{premise}

  \aergo I am not being deceived about my own existence.\texpl{from 1,2,RAA}

 \end{argument*}

 }


\end{enumerate}


\section*{Exercises for 2.7}

\begin{enumerate}

 
 \item I say in the text that \p{s_1\limplies s_2} can be read as `\p{s_1} only 
  if \p{s_2}'. Explain why, given the truth table of \plsh{Cs_1s_2}, it makes 
  sense to read is as `\p{s_1} only if \p{s_2}'.
\answer{

 Suppose someone asserts \plsh{Cs_1s_2}. What they assert is that the world is 
 such that \pp{both \p{s_1} and \p{s_2} are true (1st valuation), or \p{s_1} is 
  false but \p{s_2} is true (2nd valuation), or both \p{s_1} and \p{s_2} are 
 false (4th valuation)}. We can see that given the truth of the conditional, the 
 falsity of \p{s_2} guarantees the falsity of \p{s_1}. So the truth of \p{s_2} 
 is necessary for the truth of \p{s_1} and that is what `\p{s_1} only if   
 \p{s_2}' means.

}



 \item One way of reading \p{s_1 \limplies s_2} that I do not give in the text 
  is `\p{s_2} if \p{s_1}'. Explain why this is also a good way of reading 
  \p{s_1\limplies s_2}.

  \answer{

   We have seen above that \plsh{Cs_1s_2} says that the truth of \p{s_2} is 
   necessary for the truth of \p{s_1}. This means that given the truth of 
   \plsh{Cs_1s_2}, it cannot be that the \p{s_2} is false even though \p{s_1} is 
   true. Hence, the truth of \p{s_1} is sufficient for the truth of \p{s_2}. And 
   that is what `\p{s_2} if \p{s_1}' says.

  }
 \item How could we say `\p{s_1} if and only if \p{s_2}' by utilizing the 
  availability of \p{\limplies}?

\answer{

 \plsh{KCs_1s_2Cs_2s_1}

}


 \item Give the syntax tree and truth table of each of the following sentences.

  \begin{enumerate}

   \item \plsh{ANPQ}
	\answer{

	 \plsts{ANPQ}
	 \vskip 2em

	 \pltt{ANPQ}
   }
   \item \plsh{CPNNP}

	\answer{
	 
	 \plsts{CPNNP}
	 \vskip 2em
	\pltt{CPNNP}
}

   \item \plsh{CNNPP}

	\answer{

	 \plsts{CNNPP}
	 \vskip 2em
	\pltt{CNNPP}
   }

   \item \plsh{CAPQR}

	\answer{

	 \plsts{CAPQR}
	 \vskip 2em \pltt{CAPQR}
	}
	
   \item \plsh{KCPRCQR}

	\answer{

	 \plsts{KCPRCQR}
	 \vskip 2em
	 \pltt{KCPRCQR}
	}

  \end{enumerate}

 \item Show, using truth tables, that \p{s_1 \limplies s_2} is logically 
  equivalent to \p{\lnot s_1 \lor s_2}.

  \answer{

   Here is the truth table for \p{\lnot s_1 \lor s_2}:

   
%generated by gentzen
\begin{tabular}{cc|c||c}
\p{s_1} & \p{s_2} & \p{\mc{\lnot }s_1} & \p{\lnot s_1\mc{\lor }s_2}\\
\hline
\emph{T} & \emph{T} & \emph{F} & \emph{T}\\
\emph{F} & \emph{T} & \emph{T} & \emph{T}\\
\emph{T} & \emph{F} & \emph{F} & \emph{F}\\
\emph{F} & \emph{F} & \emph{T} & \emph{T}\\
\end{tabular}
}


\item Explain why the following is true: If \p{s_1} and \p{s_2} are logically 
 equivalent, then \p{s_1 \limplies s_2} is a tautology.

 \answer{

  The truth values of \p{s_1} and \p{s_2} are the same in every interpretation.  
  Given the truth table of the conditional, if both \p{s_1} and \p{s_2} are true, 
  \p{s_1\limplies s_2} is true. If both are false, \p{s_1 \limplies s_2} is also 
  true. So the conditional is true in every interpretation which means it is a 
  tautology.

 }


\end{enumerate}

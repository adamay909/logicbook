
\section*{Exercises for 3.5--3.11}

These exercises are intended to get you used to using sequents.

\begin{enumerate}
%\renewcommand{\mask}[1]{#1}
 \item A derivation is a series of sequents. Because of that, derivations 
  contain information that the standardized presentation of arguments do not 
  because that style only tracks the succedent side explicitly. The inference 
  rules of our proof system tell you, among other things, how to keep track of 
  things on the datum side.
 
  Consider the following derivation which is missing the datum on line 3: 

  %Derive from \p{\seq{\Gamma}{A \limplies B}} to \p{\seq{\Gamma, \Delta}{B}}

\begin{argument*}
%generated by gentzen

\ais{\Gamma}{A \limplies B}{premise}

\ais{\Delta}{A}{premise}

\ais{\mask{\Gamma, \Delta}}{B}{1,2,\condE}

\end{argument*}


  What goes in the datum of line 3? You can see in the annotation that line 3 
  got there by applying Conditional Elimination to lines 1 and 2. Line 1's datum 
  is \p{\Gamma} and line 2's is \p{\Delta}. Conditional Elimination tells us 
  that in that case the datum of line 3 is \p{\Gamma,\Delta} so that's what goes 
  in there. 

  Let's do a couple more of these to get used to paying attention to the datum.

  Fill in the missing datums in the following two derivations:
\begin{enumerate}\setlength{\itemsep}{1.5em}
\item

\begin{argument*}
%generated by gentzen

\ais{\Gamma}{(P \lor Q) \limplies R}{premise}

\ais{\Delta}{P}{premise}

\ais{\mask{\Delta}}{P \lor Q}{2,\disjI}

\ais{\mask{\Gamma, \Delta}}{R}{1,3,\condE}

\end{argument*}

\item

\begin{argument*}
%generated by gentzen

\ais{\Gamma}{(P \lor Q) \limplies R}{premise}

\ais{\mask{P}}{P}{A}

\ais{\mask{P}}{P \lor Q}{2,\disjI}

\ais{\mask{\Gamma, P}}{R}{1,3,\condE}

\end{argument*}



   \item You can think of the argument (a) as formalizing something like `The 
	college catalog says that if Masha has taken logic or has taken calculus, 
	then she has satisfied the formal reasoning requirement.  Her transcript 
	says that she have taken logic.  It follows that Masha has satisfied the 
	formal reasoning requirement.' What would argument (b) be formalizing? 

	\answer{

	Something like:  `The college catalog says that if Masha has taken logic or 
	has taken calculus, then she has satisfied the formal reasoning requirement.  
	Assuming she has taken logic, it follows that Masha has satisfied the formal 
	reasoning requirement.'
   }
\end{enumerate}

\item Our inference rules keep track of \emph{what} supports \emph{what}
 ---the datum is the former what, the succedent the latter. Let's practice 
 using our inference rules to keep track of the succedent side.

 \begin{enumerate}

  \item  We know that \p{P\land Q} and \p{Q\land P} are logically equivalent.  
   That means that if \p{\Gamma} supports \p{P\land Q} it also supports 
   \p{Q\land P}. Any decent proof system should tells us that, and ours does.  
   Add the missing succedents in the following derivation:
%Derive from \p{\seq{\Gamma}{S_1 \land S_2}} to \p{\seq{\Gamma}{S_2 \land S_1}}

\begin{argument*}
%generated by gentzen

\ais{\Gamma}{P \land Q}{premise}

\ais{\Gamma}{\mask{P}}{1,\conjE}

\ais{\Gamma}{Q}{1,\conjE}

\ais{\Gamma, \Gamma}{\mask{Q \land P}}{2,3,\conjI}

\ais{\Gamma}{Q \land P}{4}

\end{argument*}


   
   Hint: according to the annotations, line 5 is a rewrite of line 4. That tells 
   you what the succedent of line 4 is.


\item We also know that \p{P\lor Q} and \p{Q\lor P} are logically equivalent.  
 That means that if \p{\Gamma} supports \p{P \lor Q} it also supports    
 \p{Q\lor P}. Our proof system shows that, too.  Add the missing succedents in 
 the following derivation:
 \begin{argument*}

  \ais{\Gamma}{P \lor Q}{premise}

  \ais{P}{\mask{P}}{A}

  \ais{P}{\mask{Q \lor P}}{2,\disjI}

  \ais{Q}{Q}{A}

  \ais{Q}{\mask{Q \lor P}}{4,\disjI}

  \ais{\Gamma}{Q \lor P}{1,3,5,\disjE}

 \end{argument*}


 Notice that there are several lines with exactly the same succedent as the 
 concluding line. An argument in the standardized form would make it much more 
 difficult to discern when one has actually reached the desired conclusion 
 because the standardized form only gives us the succedent side.
\end{enumerate}

\item Comprehending a derivation requires comprehending how the various sequents 
 work together to enable us to infer to the conclusion. Annotations are there to 
 guide our comprehension. In presenting your derivation, it is crucial to make 
 sure that your annotation are correct. Let's practice annotating.

 \p{P\land (Q\lor R)} and \p{(P\land Q) \lor (P\land R)} are logically 
 equivalent. So if we have evidence for the former sentence, we have evidence 
 for the latter. We can show that.  Fill in the missing annotations in the 
 following derivation:

\begin{argument*}

\ais{\Gamma}{P \land (Q \lor R)}{premise}

\ais{\Gamma}{P}{1,\mask{\conjE}}

\ais{\Gamma}{Q \lor R}{\mask{1,\conjE}}

\ais{Q}{Q}{\mask{A}}

\ais{\Gamma,Q}{P \land Q}{\mask{2,4,\conjI}}

\ais{\Gamma,Q}{(P \land Q) \lor (P \land R)}{5,\disjI}

\ais{R}{R}{\mask{A}}

\ais{\Gamma,R}{P \land R}{\mask{2,7,\conjI}}

\ais{\Gamma,R}{(P \land Q) \lor (P \land R)}{\mask{8,\disjI}}

\ais{\Gamma,\Gamma,\Gamma}{(P \land Q) \lor (P \land R)}{\mask{3,6,9,\disjE}}

\ais{\Gamma}{(P \land Q) \lor (P \land R)}{\mask{10}}

\end{argument*}


\item Fill in the missing datums, succedents, annotations in the following  
 derivation from \p{\seq{\Gamma}{(P\lor Q)
 \lor R)}} to \p{\seq{\Gamma}{P\lor(Q\lor R)}}.
 
\begin{argument*}

\ais{\Gamma}{(P \lor Q) \lor R}{premise}

\ais{\mask{P \lor Q}}{P \lor Q}{A}

\ais{P}{P}{A}

\ais{\mask{P}}{P \lor (Q \lor R)}{3,\disjI}

\ais{Q}{\mask{Q}}{A}

\ais{\mask{Q}}{Q \lor R}{5,\disjI}

\ais{Q}{\mask{P \lor (Q \lor R)}}{6,\disjI}

\ais{P \lor Q}{P \lor (Q \lor R)}{2,4,7,\disjE}

\ais{R}{R}{A}

\ais{R}{Q \lor R}{9,\disjI}

\ais{R}{P \lor (Q \lor R)}{\mask{10,\disjI}}

\ais{\Gamma}{P \lor (Q \lor R)}{1,8,11,\disjE}

\end{argument*}

\item I mentioned that \disjE{} need not require three sequents. Here is an 
 example of that:

 \begin{argument*}

  \ais{\Gamma}{P \lor P}{premise}

  \ais{\Delta, P}{Q}{premise}

  \ais{\Gamma,\Delta}{Q}{1,2,2,\disjE}

 \end{argument*}

 Notice that in the third line of the annotation, line 2 is referred to twice.  
 It is used once as \p{\seq{\Lambda_2, s_1}{s_3}} and again as \p{\seq{\Lambda_3, 
 s_2}{s_3}}. This is possible because two of the sequents that must be matched 
 for using \disjE{} have the same form. You can do the same using \conjI{} to 
 derive from \p{\seq{\Gamma}{P}} to \p{\seq{\Gamma}{ P\land P}}
 . Add the missing annotations:

 \begin{argument*}

  \ais{\Gamma}{P}{\mask{premise}}


   \ais{\Gamma}{P \land P}{\mask{1,1,\conjI}}

 \end{argument*}




\item Construct the following derivations:

 \begin{enumerate}

  \item From \plshs{/G:KQP} to \plshs{/G:AQP}.

  \item From \plshs{/G:KNQKQP} to \plshs{/G:Q}.

  \item From \plshs{/G:P} to \plshs{/G:AKPPQ}


\end{enumerate}

\answer{

 \begin{enumerate}
 \setlength{\itemsep}{2em}
  \item 

   \begin{argument*}

	\ais{\Gamma}{Q\land P}{premise}

	\ais{\Gamma}{Q}{1,\conjE}

	\ais{\Gamma}{Q\lor P}{2,\disjI}

   \end{argument*}

   \item

	\begin{argument*}

	 \ais{\Gamma}{\lnot Q\land(Q \land P)}{premise}

	 \ais{\Gamma}{Q\land P}{1,\conjE}

	 \ais{\Gamma}{Q}{2,\conjE}

	\end{argument*}

	\item

	 \begin{argument*}

	  \ais{\Gamma}{P}{premise}

	  \ais{\Gamma}{P\land P}{1,1,\conjI}

	  \ais{\Gamma}{(P\land P)\lor Q}{2,\disjI}
	 \end{argument*}
   \end{enumerate}

  }
\end{enumerate}


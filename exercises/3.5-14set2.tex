
\newcommand{\lbhelper}[1]{\input{|"lbhelper #1"}}

\begin{enumerate}
 \setlength{\itemsep}{2em}

\item Fill in missing items:

\begin{argumentN}[1]
%generated by gentzen

\ai{Γ}{(P\lor Q)\mc{\limplies }R}{premise}

\ai{P}{P}{A}

\ai{\mask{P}}{\mask{P\mc{\lor }Q}}{2,\disjI}

\ai{\mask{Γ, P}}{\mask{R}}{1,3,\condE}

\ai{Γ}{\mask{P\mc{\limplies }R}}{4,\condI}

\ai{Q}{Q}{A}

\ai{\mask{Q}}{\mask{P\mc{\lor }Q}}{6,\disjI}

\ai{\mask{Γ, Q}}{\mask{R}}{1,7,\condE}

\ai{Γ}{Q\mc{\limplies }R}{8,\condI}

\ai{Γ}{\mask{(P\limplies R)\mc{\land }(Q\limplies R)}}{5,9,\conjI}

\end{argumentN}

\item Add missing items. 
%Derive from \p{\seq{Γ}{\lnot P\mc{\land }\lnot Q}} to \p{\seq{Γ}{\mc{\lnot }(P\lor Q)}}

\begin{argumentN}[1]
%generated by gentzen

\ai{Γ}{\lnot P\mc{\land }\lnot Q}{premise}

\ai{P\mc{\lor }Q}{P\mc{\lor }Q}{A}

\ai{P}{\mask{P}}{A}

\ai{Γ}{\mc{\lnot }P}{\mask{1,\conjE}}

\ai{P, P\mc{\lor }Q}{P}{\mask{3}}

\ai{Γ, \mask{P\mc{\lor }Q}}{\mc{\lnot }P}{4}

\ai{\mask{Γ, P}}{\mc{\lnot }(P\lor Q)}{5,6,\negI}

\ai{\mask{Q}}{Q}{A}

\ai{Γ}{\mc{\lnot }Q}{\mask{1,\conjE}}

\ai{Q, \mask{P\mc{\lor }Q}}{Q}{8}

\ai{Γ, \mask{P\mc{\lor }Q}}{\mc{\lnot }Q}{9}

\ai{\mask{Γ, Q}}{\mc{\lnot }(P\lor Q)}{10,11,\negI}

\ai{Γ, P\mc{\lor }Q}{\mc{\lnot }(P\lor Q)}{\mask{2,7,12,\disjE}}

\ai{Γ}{\mc{\lnot }(P\lor Q)}{\mask{2,13,\negI}}

\end{argumentN}

\newpage
\item Here is part of a derivation from \lbhelper{-s /G:NAPQ} to     
 \lbhelper{-s /G:KNPNQ}. Complete the rest.

\begin{argumentN}[1]
%generated by gentzen

\ai{Γ}{\mc{\lnot }(P\lor Q)}{premise}

\ai{P}{P}{A}

\ai{P}{P\mc{\lor }Q}{2,\disjI}

\ai{Γ, P}{\mc{\lnot }(P\lor Q)}{1}

\ai{Γ}{\mc{\lnot }P}{3,4,\negI}

\opts{

 \dotline
 \dotline
 \dotline
 \dotline
 \dotline
 \dotline
 \dotline
}
{
 \answer{ 

\ai{Q}{Q}{A}

\ai{Q}{P\mc{\lor }Q}{6,\disjI}

\ai{Γ, Q}{\mc{\lnot }(P\lor Q)}{1}

\ai{Γ}{\mc{\lnot }Q}{7,8,\negI}

\ai{Γ}{\lnot P\mc{\land }\lnot Q}{5,9,\conjI}
}}


\end{argumentN}

 \item When someone offers considerations that lead to a contradiction, that is 
  usually taken to be a bad thing. One reason why contradictions are bad is 
  captured by the observation known as \emph{ex contradictione quodlibet}: from 
  a contradiction, derive at will. That is, if you had proof of a contradiction 
  you could prove anything you want. The following demonstrates the point. Add 
  the missing annotations:


%title: %ex  contradictione  quodlibet<div   id= "cursor2">&thinsp; </div >
\begin{argumentN}[1]
%generated  by  gentzen

 \ai{\Gamma }{P\mc{\land  }\lnot  P}{\mask{premise}}

 \ai{\Gamma ,  \mc{\lnot  }Q}{P\mc{\land  }\lnot  P}{\mask{1}}

 \ai{\Gamma ,  \mc{\lnot  }Q}{P}{\mask{2,\conjE}}

 \ai{\Gamma ,  \mc{\lnot  }Q}{\mc{\lnot  }P}{\mask{2,\conjE}}

 \ai{\Gamma }{\mc{\lnot  }\lnot  Q}{\mask{3,4,\negI}}

 \ai{\Gamma }{Q}{\mask{5,\negE}}

\end{argumentN}

Notice that you could replace \p{Q} with anything you please. So can equally 
well derive \p{\lnot Q}. Here we have a decisive reason to reject the premise: 
something must have gone wrong in thinking that we have conclusive reason to 
accept the premise.

\newpage
\item Derive from \p{\seq{\Gamma}{P\lor P}} to \p{\seq{\Gamma}{P}}.
\opts{

 \dotline
 \dotline
 \dotline
 \dotline
 \dotline
}
{\answer{


%title: %
\begin{argumentN}[1]
%generated  by  gentzen

\ai{\Gamma }{P\mc{\lor  }P}{premise}

\ai{P}{P}{A}

\ai{\Gamma }{P}{1,2,2,\disjE}

\end{argumentN}

}
}

\item Derive from \p{\seq{\Gamma}{P\limplies (Q \limplies R)}} to \p{\seq{\Gamma}
 {(P\land Q)\limplies R}}. Hint: assume \p{P\land Q}.

 \opts{
\dotline
\dotline
\dotline
\dotline
\dotline
\dotline
\dotline
\dotline
\dotline
\dotline
 }
 {
  \answer{


%title: %
\begin{argumentN}[1]
%generated  by  gentzen

\ai{\Gamma }{P\mc{\limplies  }(Q\limplies  R)}{premise}

\ai{P\mc{\land  }Q}{P\mc{\land  }Q}{A}

\ai{P\mc{\land  }Q}{P}{2,\conjE}

\ai{\Gamma ,  P\mc{\land  }Q}{Q\mc{\limplies  }R}{1,3,\condE}

\ai{P\mc{\land  }Q}{Q}{2,\conjE}

\ai{\Gamma ,  P\mc{\land  }Q}{R}{4,5,\condE}

\ai{\Gamma }{(P\land  Q)\mc{\limplies  }R}{6,\condI}

\end{argumentN}

}
}

\item Derive from \lbhelper{-s /G:CKPQR} to \lbhelper{-s /G:CPCQR}. Hint: assume 
 \p{P} and assume \p{Q}.

 \opts{

\dotline
\dotline
\dotline
\dotline
\dotline
\dotline
\dotline
\dotline
\dotline
 }
 {
  \answer{


%title: %
\begin{argumentN}[1]
%generated  by  gentzen

\ai{\Gamma }{(P\land  Q)\mc{\limplies  }R}{premise}

\ai{P}{P}{A}

\ai{Q}{Q}{A}

\ai{P,  Q}{P\mc{\land  }Q}{2,3,\conjI}

\ai{\Gamma ,  P,  Q}{R}{1,4,\condE}

\ai{\Gamma ,  P}{Q\mc{\limplies  }R}{5,\condI}

\ai{\Gamma }{P\mc{\limplies  }(Q\limplies  R)}{6,\condI}

\end{argumentN}
}
}

\item We noted earlier that the conditional (\p{\limplies}) has some odd 
 features. The oddities show up in our proof system as well.

 \begin{enumerate}

  \item Derive from \p{\seq{\Gamma}{P}} to \p{\seq{\Gamma}{Q\limplies P}}. (Hint: 
   remember you can add anything you want to the datum of a sequent).
\opts{
 
 \dotline
 \dotline
 \dotline
 \dotline
 \dotline
 \dotline

}
{   \answer{
	
%title: %\Gamma \vdash P  to  \Gamma \vdash Q\supset P
\begin{argumentN}[1]
%generated  by  gentzen

\ai{\Gamma }{P}{premise}

\ai{\Gamma ,  Q}{P}{1}

\ai{\Gamma }{Q\mc{\limplies  }P}{2,\condI}

\end{argumentN}

}
}
\item Derive from \p{\seq{\Gamma}{\lnot P}} to \p{\seq{\Gamma}{P\limplies Q}}.  
 (Hint: assume P, and remember you can add anything you want, in particular 
 \p{\lnot Q} to the datum---see also the problem at the top of these exercises.)
\opts{


 \dotline
 \dotline
 \dotline
 \dotline
 \dotline
 \dotline
 \dotline
 \dotline
 \dotline
 \dotline
 \dotline
}
{
  \answer{

%title: %\Gamma \vdash \neg p  to  \Gamma \vdash p\supset q
\begin{argumentN}[1]
%generated  by  gentzen

\ai{\Gamma }{\mc{\lnot  }P}{premise}

\ai{P}{P}{A}

\ai{P,  \mc{\lnot  }Q}{P}{2}

\ai{\Gamma ,  \mc{\lnot  }Q}{\mc{\lnot  }P}{1}

\ai{\Gamma ,  P}{\mc{\lnot  }\lnot  Q}{3,4,\negI}

\ai{\Gamma ,  P}{Q}{5,\negE}

\ai{\Gamma }{P\mc{\limplies  }Q}{6,\condI}

\end{argumentN}
  }
 }

 \item Derive from \p{\seq{\Gamma}{P}} to \p{\seq{\Gamma}{\lnot P \limplies Q}}.  
  (Hint: assume \p{\lnot P}; and don't forget the point about being able to add 
  things to the datum.)

  \opts{


 \dotline
 \dotline
 \dotline
 \dotline
 \dotline
 \dotline
 \dotline
 \dotline

}
{
  \answer{

%title: %\Gamma \vdash p  to  \Gamma \vdash \neg p\supset q
\begin{argumentN}[1]
%generated  by  gentzen

\ai{\Gamma }{P}{premise}

\ai{\mc{\lnot  }P}{\mc{\lnot  }P}{A}

\ai{\Gamma ,  \mc{\lnot  }Q}{P}{1}

\ai{\mc{\lnot  }P,  \mc{\lnot  }Q}{\mc{\lnot  }P}{2}

\ai{\Gamma ,  \mc{\lnot  }P}{\mc{\lnot  }\lnot  Q}{3,4,\negI}

\ai{\Gamma ,  \mc{\lnot  }P}{Q}{5,\negE}

\ai{\Gamma }{\lnot  P\mc{\limplies  }Q}{6,\condI}

\end{argumentN}

}
}
\end{enumerate}

\item Derive from \p{\seq{\Gamma}{P \limplies (Q\lor R)}} and \p{\seq{\Delta}
 {\lnot Q}} to \p{\seq{\Gamma, \Delta}{P\limplies R}}. (Hint: First derive 
 \p{\seq{\Gamma,P}{Q \lor R}}. Then adapt the derivation in the first problem
 of the previous set of exercises.)
\opts{

 \dotline
 \dotline
 \dotline
 \dotline
 \dotline
 \dotline
 \dotline
 \dotline
 \dotline
 \dotline
 \dotline
 \dotline
 \dotline
 \dotline
}{
 \answer{
%title: %\Gamma \vdash p\supset (q\vee r),  \Delta \vdash \neg q  to  \Gamma ,  
 %\Delta \vdash p\supset r
\begin{argumentN}[1]
%generated  by  gentzen

\ai{\Gamma }{P\mc{\limplies  }(Q\lor  R)}{premise}

\ai{\Delta }{\mc{\lnot  }Q}{premise}

\ai{P}{P}{A}

\ai{\Gamma ,  P}{Q\mc{\lor  }R}{1,3,\condE}

\ai{Q}{Q}{A}

\ai{\Delta ,  \mc{\lnot  }R}{\mc{\lnot  }Q}{2}

\ai{Q,  \mc{\lnot  }R}{Q}{5}

\ai{\Delta ,  Q}{\mc{\lnot  }\lnot  R}{6,7,\negI}

\ai{\Delta ,  Q}{R}{8,\negE}

\ai{R}{R}{A}

\ai{\Gamma ,  \Delta ,  P}{R}{4,9,10,\disjE}

\ai{\Gamma ,  \Delta }{P\mc{\limplies  }R}{11,\condI}

\end{argumentN}
}
}


\end{enumerate} 




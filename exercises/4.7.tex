\begin{enumerate}

 \item Show that Disjunction Elimination is a valid rule of inference.

\answer{

Here is one way.

We are trying to show if \p{\Lambda_1\lentails s_1 \lor s_2}, \p{\Lambda_2, 
s_1\lentails s_3}, and \p{\Lambda_3,s_2\lentails s_3}, then \p{\Lambda_1, 
\Lambda_2, \Lambda_3 \lentails s_3}.

Given \p{\Lambda_2, s_1\lentails s_3}, we get \p{\Lambda_2 \lentails 
s_1\limplies s_3}. And given \p{\Lambda_3, s_2\lentails s_3}, we get 
\p{\Lambda_3 \lentails s_2 \limplies s_3}. But this means that the truth of all 
of \p{\Lambda_1}, \p{\Lambda_2}, \p{\Lambda_3}  guarantees the truth of \p{s_3}. 

So \p{\Lambda_1, \Lambda_2, \Lambda_3 \lentails s_3}.

}

 \item Show that Negation Introduction is a valid rule of inference.

  \answer{ 

   We are trying to show that if \p{\seq{\Lambda_1,s_1}{s_2}} and 
   \p{\seq{\Lambda_2,s_1}{\lnot s_2}} are both correct, then so is 
   \p{\seq{\Lambda_1,\Lambda_2}{\lnot s_1}}.

   Consider the following derivation:

   \begin{argument*}

	\ai{\Lambda_1,s_1}{s_2}{}
   \ai{\Lambda_2,s_1}{\lnot s_2}{}
   \ai{\Lambda_1}{s_1\limplies s_2}{1,\condI}
   \ai{\Lambda_2}{s_1\limplies s_2}{2,\condI}
   \ai{\Lambda_1,\Lambda_2}{(s_1\limplies s_2)\land(s_1\limplies \lnot s_2)}{3,4,
   \conjI}
  \end{argument*}

  Since \condI{} and \conjI{} are both valid rules of inference, given the first 
  two sequents are correct, we know that so is the last:

  \p{\Lambda_1,\Lambda_2\lentails (s_1\limplies s_2)\land(s_1\limplies\lnot s_2)}

  We can use truth tables to show:
  \p{\lentails [(s_1\limplies s_2)\land(s_1\limplies\lnot s_2)]\limplies \lnot 
  s_1}
  
  Therefore,

  \p{\Lambda_1,\Lambda_2\lentails \lnot s_1}

   So \negI{} is valid.
 }




\end{enumerate}

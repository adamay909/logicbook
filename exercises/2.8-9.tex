\section*{Exercises for 2.8 and 2.9}
\newcommand{\lbhelper}[1]{\input{|"lbhelper #1"}}
\begin{enumerate}

 \item  Take the procedure for enumerating class-1 sentences. The procedure
lists sentences in groups of five (five sentences for each fraction). List
the first 3 groups of sentences that the procedure lists (that’s 15 sentences in 
total).
\answer{

	Let \p{S_1, S_2, S_3, ...} be the sentences of class-1. Then:

 The first three fractions enumerated on Figure 2.1 are 1/1, 2/1, 1/2. We plug 
 the numerator into \p{n} and the denominator into \p{m}.  So the first three 
 groups of sentences are:

 \p{S_1, \lnot S_1, S_1\land S_1, S_1\lor S_1, S_1\limplies S_1},
 \p{S_2, \lnot S_2, S_2\land S_1, S_2\lor S_1, S_2\limplies S_1},
 \p{S_1, \lnot S_1, S_1\land S_2, S_1\lor S_2, S_1\limplies S_2}
}

\item The procedure for enumerating class-1 sentences lists sentences in
 groups of five. One of them is \lbhelper{-ab -f KP_nP_m}. Why is it ok not to 
 list \lbhelper{-ab -f KP_mP_n}?

\answer{

 \lbhelper{-ab -f KP_nP_m} is listed when we encounter n/m, and we know 
 \lbhelper{-ab -f KP_mP_n} will also be listed eventually because  for every 
 fraction n/m, the fraction m/n will also be listed eventually.  

Quite a few of you answered that it is because \p{A_m\land A_n} is logically 
equivalent to \p{A_n\land A_m}. But that wouldn't explain why it is also ok not 
to list \p{A_m\limplies A_n}. 

}

\item  ``Even a formal language with infinitely many atomic sentences could
not express all the ways the world could be.'' Is this true or false? Explain.

\answer{

 Each interpretation represents a way the world could be. Given infinitely many 
 atomic sentences, there are more interpretations than there are natural 
 numbers.  But even with infinitely many atomic sentences, there are only as 
 many sentences as there are natural numbers. So the formal language cannot 
 express all the ways the world could be.

Any language whose sentences are composed of countably many characters has only 
as many sentences as there are natural numbers. In particular, if a language can 
be presented on a computer screen, it only has as many sentences as there are 
natural numbers---because a computer must represent a sentence by a natural 
number. Now consider this:

For any subset S of natural numbers, there is a fact whether or not the number 1 
is in that subset. Could we express for each subset S of natural numbers the 
claim that the number 1 is a member of S? The answer is No: there are more 
subsets of natural numbers than there are natural numbers, but only as many 
sentences as there are natural numbers. So there are uncountably many very 
simple mathematical facts that no language---formal or otherwise---of ours can 
even express. 

And if they can't be expressed, they can't be proven either. If you are 
interested in mathematics, this is a point worth letting sink in. Just to repeat 
the point:  no matter how powerful our mathematical language, there are 
uncountably many subsets of natural numbers of which we cannot even ask whether 
1 is a member of that set. And once you see the point, you will notice that 
there many more mathematical facts we cannot even express, let alone prove.

}

\end{enumerate}



\section*{Exercises for 4.1--4.4}

\begin{enumerate}
 \setlength{\itemsep}{2em}

 \item Fill in missing items:

\begin{argumentN}[1]
%generated by gentzen

\ai{Γ}{(P\lor Q)\mc{\limplies }R}{premise}

\ai{P}{P}{A}

\ai{\mask{P}}{\mask{P\mc{\lor }Q}}{2,\disjI}

\ai{\mask{Γ, P}}{\mask{R}}{1,3,\condE}

\ai{Γ}{\mask{P\mc{\limplies }R}}{4,\condI}

\ai{Q}{Q}{A}

\ai{\mask{Q}}{\mask{P\mc{\lor }Q}}{6,\disjI}

\ai{\mask{Γ, Q}}{\mask{R}}{1,7,\condE}

\ai{Γ}{Q\mc{\limplies }R}{8,\condI}

\ai{Γ}{\mask{(P\limplies R)\mc{\land }(Q\limplies R)}}{5,9,\conjI}

\end{argumentN}


\item Add missing annotations for the following proof of             
 \lbh{-s :CANPNQNKPQ}

\begin{argumentN}[1]
%generated by gentzen

\ai{\lnot P\mc{\lor }\lnot Q}{\lnot P\mc{\lor }\lnot Q}{A}

\ai{P\mc{\land }Q}{P\mc{\land }Q}{\mask{A}}

\ai{\mc{\lnot }P}{\mc{\lnot }P}{\mask{A}}

\ai{P\mc{\land }Q}{P}{\mask{2,\conjE}}

\ai{\mc{\lnot }P, P\mc{\land }Q}{\mc{\lnot }P}{\mask{3}}

\ai{\mc{\lnot }P}{\mc{\lnot }(P\land Q)}{\mask{4,5,\negI}}

\ai{\mc{\lnot }Q}{\mc{\lnot }Q}{\mask{A}}

\ai{P\mc{\land }Q}{Q}{\mask{2,\conjE}}

\ai{\mc{\lnot }Q, P\mc{\land }Q}{\mc{\lnot }Q}{\mask{7}}

\ai{\mc{\lnot }Q}{\mc{\lnot }(P\land Q)}{\mask{8,9,\negI}}

\ai{\lnot P\mc{\lor }\lnot Q}{\mc{\lnot }(P\land Q)}{\mask{1,6,10,\disjE}}

\ai{}{(\lnot P\mc{\lor }\lnot Q)\limplies \lnot(P\land Q)}{\mask{11,\condI}}

\end{argumentN}

\item Prove \lbh{-s :CNCPQNQ}. (Hint: assume \plsh{NCPQ} as well as  
 \plsh{Q}.)
\opts{
 \dotline

 \dotline
 
 \dotline

 \dotline

 \dotline

 \dotline

 \dotline

 \dotline

 \dotline

 \dotline

 \dotline

 \dotline
}
{
\begin{argumentN}[1]
%generated by gentzen

\ai{\mc{\lnot }(P\limplies Q)}{\mc{\lnot }(P\limplies Q)}{A}

\ai{Q}{Q}{A}

\ai{Q, P}{Q}{2}

\ai{Q}{P\mc{\limplies }Q}{3,\condI}

\ai{\mc{\lnot }(P\limplies Q), Q}{\mc{\lnot }(P\limplies Q)}{1}

\ai{\mc{\lnot }(P\limplies Q)}{\mc{\lnot }Q}{4,5,\negI}

\ai{}{\mc{\lnot }(P\limplies Q)\limplies \lnot Q}{6,\condI}
 
\end{argumentN}}


\item Here is part of a proof of \lbh{-s NAPQ:KNPNQ}. Complete the rest.

\begin{argumentN}[1]
%generated by gentzen

\ai{\lnot (P \lor Q)}{\mc{\lnot }(P\lor Q)}{A}

\ai{P}{P}{A}

\ai{P}{P\mc{\lor }Q}{2,\disjI}

\ai{\lnot (P \lor Q), P}{\mc{\lnot }(P\lor Q)}{1}

\ai{\lnot (P \lor Q)}{\mc{\lnot }P}{3,4,\negI}

\end{argumentN}
\opts{

 \dotline
 
 \dotline

 \dotline

 \dotline

 \dotline

 \dotline

 \dotline

 \dotline
 
}{
 \begin{argumentN}[6]
\ai{Q}{Q}{A}

\ai{Q}{P\mc{\lor }Q}{6,\disjI}

\ai{\lnot (P \lor Q), Q}{\mc{\lnot }(P\lor Q)}{1}

\ai{\lnot (P \lor Q)}{\mc{\lnot }Q}{7,8,\negI}

\ai{\lnot (P \lor Q)}{\lnot P\mc{\land }\lnot Q}{5,9,\conjI}

\end{argumentN}
}
\newpage

\item Add missing items.

\begin{argumentN}[1]
%generated by gentzen

\ai{\mc{\lnot }(P\limplies Q)}{\mc{\lnot }(P\limplies Q)}{A}

\ai{\mc{\lnot }P}{\mc{\lnot }P}{A}

\ai{\mask{\mc{\lnot }P, \mc{\lnot }Q}}{\mask{\mc{\lnot }P}}{2}

\ai{\mask{P}}{\mask{P}}{A}

\ai{\mask{P, \mc{\lnot }Q}}{\mask{P}}{4}

\ai{\mc{\lnot }P, P}{\mc{\lnot }\lnot Q}{3,5,\negI}

\ai{\mc{\lnot }P, P}{Q}{6,\negE}

\ai{\mask{\mc{\lnot }P}}{\mask{P\mc{\limplies }Q}}{\mask{7,\condI}}

\ai{\mc{\lnot }(P\limplies Q), \mask{\mc{\lnot }P}}{\mc{\lnot }(P\limplies Q)}
{1}

\ai{\mc{\lnot }(P\limplies Q)}{\mc{\lnot }\lnot P}{8,9,\negI}

\ai{\mc{\lnot }(P\limplies Q)}{P}{\mask{10,\negE}}

\ai{}{\mc{\lnot }(P\limplies Q)\limplies P}{\mask{11,\condI}}

\end{argumentN}

\item Prove \p{\seq{P\limplies Q, P \lor Q}{Q}}. Hint: Assume \p{P\limplies Q} 
 and assume \p{Q}. Use \disjE.


 \opts{
\dotline

\dotline

\dotline

\dotline

\dotline

\dotline

\dotline

\dotline

\dotline

\dotline
}
{

%title: %P\supset Q,P\vee Q\vdash Q
\begin{argumentN}[1]
%generated  by  gentzen

\ai{P\mc{\limplies  }Q}{P\mc{\limplies  }Q}{A}

\ai{P\mc{\lor  }Q}{P\mc{\lor  }Q}{A}

\ai{P}{P}{A}

\ai{P\mc{\limplies  }Q,  P}{Q}{1,3,\condE}

\ai{Q}{Q}{A}

\ai{P\mc{\limplies  }Q,  P\mc{\lor  }Q}{Q}{2,4,5,\disjE}

\end{argumentN}

}
\newpage

\item Recall the Prisoner's Dilemma (exercise for 3.1-4). Let \p{P} mean that 
 Jerry will confess, and let \p{Q} mean that Ben is better off confessing.  Turn  
 the reasoning in the standardized form provided in the answer key into a 
 derivation that utilizes EM (hint: the derivation will have two premises).

 \opts{
\dotline

\dotline

\dotline

\dotline

\dotline

\dotline

\dotline

\dotline

\dotline

}
{

\begin{argumentN}[1]
%generated by gentzen

\ai{Γ}{P\mc{\limplies }Q}{premise}

\ai{Δ}{\lnot P\mc{\limplies }Q}{premise}

\ai{}{P\mc{\lor }\lnot P}{EM}

\ai{P}{P}{A}

\ai{Γ, P}{Q}{1,4,\condE}

\ai{\mc{\lnot }P}{\mc{\lnot }P}{A}

\ai{Δ, \mc{\lnot }P}{Q}{2,6,\condE}

\ai{Γ, Δ}{Q}{3,5,7,\disjE}

\end{argumentN}

 }
 \setlength{\itemsep}{0em}
 


\end{enumerate}


\subsubsection*{1.}
\def\aitem{\item[\stepcounter{argnum}\arabic{argnum}.]}

Put the following arguments in standard form. You may find that the argument 
forms we have named so far do not suffice to capture all the steps in the 
argument. If that happens, flag it and make some suggestions for possible 
argument patterns that could be used to fill the gap.

\begin{enumerate}
 \renewcommand{\labelenumi}{\alph{enumi}.}
 \item If Sally has taken logic or has taken calculus, she has satisfied 
  Pomona's area 5 requirements. She has taken calculus. So she has satisfied the 
  area 5 requirements.
\answer{

 \begin{argument*}

  \aitem If Sally has taken logic or has taken calculus, then she has satisfied 
  Pomona's area 5 requirements. \texpl{premise}

  \aitem She has taken calculus. \texpl{premise}

  \aitem Sally has taken logic or has taken calculus. \texpl{from 2}

  \aitem Sally has satisfied Pomona's area 5 requirements. \texpl{1,3,MP}

 \end{argument*}

 Note: It is tempting to skip line 3 and go straight to the conclusion. Not 
 notice that premise does not match the antecedent of the conditional in premise 
 1.
}

 \item Only the butler or the gardener could have killed the old man. The 
  gardener did not do it. So the butler killed the old man.
\answer{

  \begin{argument*}

   \aitem The butler killed the old man or the gardener killed the old man.  
   \texpl{premise}

   \aitem The gardener did not kill the old man. \texpl{premise}

   \aitem The butler killed the old man. \texpl{1,2, argument by elimination}

  \end{argument*}

The following general inference pattern seems fine:

\begin{argument*}

 \aitem A or B. \texpl{premise}

 \aitem not A. \texpl{premise}

 \aitem B. \texpl{1,2, argument by elimination}

\end{argument*}

After all, if a disjunction is true and one of the disjuncts is false, the other 
disjunct must be true.

You might wonder what happened to the `only' in the plain prose version of the 
reasoning. The `only' signifies that the two options stated are exhaustive (no 
third suspect). You might worry whether this exhaustiveness has been fully 
captured. Here is an attempt at capturing it:

\begin{argument*}

 \aitem If someone killed the old man, then it was the butler or it was the 
 gardener. \texpl{premise}

 \aitem Someone killed the old man. \texpl{premise}

 \aitem It was the butler or it was the garderner. \texpl{1,2,MP}

 \aitem It was not the bulter. \texpl{premise}

 \aitem It was the gardener. \texpl{3,4,argument by elimination}

\end{argument*}

The `If A,  then B' construction in premise 1 can also be read as `A only if B' 
which enables us to capture the `only' in the plain prose version. But this 
rendering may feel less natural than the one above.

}


 \item Only the butler or the gardener could have killed the old man. So if the 
  butler didn't do it, the gardener must have done it.

\answer{ 

 Let's take this literally and take the conclusion to be the conditional \pp{if 
 the butler didn't kill the old man, then the gardener killed the old man}.

  \begin{argument*}

   \aitem The butler killed the old man or the gardener killed the old man.  
   \texpl{premise}

   \aitem The gardener did not kill the old man. \texpl{assumption}

   \aitem The butler killed the old man. \texpl{1,2, argument by elimination}

   \aitem If the gardener did not kill the old man, then the butler killed the 
   old man. \texpl{1,2,3, conditional weakening}

  \end{argument*}

If you stopped at 3 and accepted that the butler did it, you would accept 
something based on a mere assumption that the gardener is innocent. That is 
generally a bad idea. It makes sense, therefore, to weaken your position to just 
accepting the conditional in 4.

There is another fairly plausible way of reading the argument (one that I did 
not intend but I have to agree is acceptable), and that is to take it as a way 
of saying:

  \begin{argument*}

   \aitem The butler killed the old man or the gardener killed the old man.  
   \texpl{premise}

   \aitem The gardener did not kill the old man. \texpl{premise}

   \aitem The butler killed the old man. \texpl{1,2, argument by elimination}


  \end{argument*}

  Notice that 2 is now labeled a premise. So the idea is that it is taken as 
  somehow known (not just assumed) that the gardener didn't do it.

 }

\end{enumerate}

\subsubsection*{2.}

Put each of the following into a valid argument in standard form. Be creative 
(i.e., add unstated premises, extra steps, etc. as necessary).

\begin{enumerate}

 \renewcommand{\labelenumi}{\alph{enumi}.}
\item The new movie, which is directed by Christopher McQuarrie, runs for two 
 hours and forty-three minutes, and its full title is ``Mission: 
 Impossible---Dead Reckoning Part One,'' which takes about half an hour to say.  
 If Part Two, which is due to be released next June, is of similar dimensions, 
 we’ll be landed with a tale that is more than five hours in the telling.  
 Concision junkies will have to look elsewhere.

 \answer{

  The first thing to do to identify the conclusion. It seems pretty clear that 
  the point is that the movie is too long, so let's take that as the conclusion.  
  The considerations offered in favor of that assessment of the movie are: part 
  one runs for 2h43m and that with part two the whole story will more than five 
  hours long. So here is a first stab:

  \begin{argument*}

   \aitem Part 1 is 2h43m long.\texpl{premise}

   \aitem Part 2 is of similar length.\texpl{premise}

   \aitem The whole is more than five hours long. \texpl{1,2, math}

   \aitem If the whole is more than five hours long, then it is too long.  
   \texpl{premise}

   \aitem It is too long. \texpl{3,4,MP}

  \end{argument*}

  But we might wonder how the author gets to premise 2? What the author says is 
  that \pp{\emph{if} part 2 is similar dimensions as part 1, then the whole is 
  more than five hours long}. But why think part 2 will be of similar dimensions?  
  Here the remark about the lengthy title might help. How about the following to 
  get support premise 2 above:

  \begin{argument*}

   \aitem The title of part 1 is way too long. \texpl{premise}

   \aitem If the title of the movie is too long, then the director is not 
   interested in conciseness. \texpl{premise}

   \aitem The director is not interested in conciseness. \texpl{1,2,MP}

   \aitem If the director is not interested in conciseness, then part 2 will be 
   of similar length as part 1. \texpl{premise}

   \aitem Part 2 will be of similar length as part 1. \texpl{3,4,MP}

  \end{argument*}

  Putting the two arguments together, we get:

  \begin{argument*}

   \aitem Part 1 is 2h43m long.\texpl{premise}
   
   \aitem The title of part 1 is way too long. \texpl{premise}

   \aitem If the title of the movie is too long, then the director is not 
   interested in conciseness. \texpl{premise}

   \aitem The director is not interested in conciseness. \texpl{2,3,MP}

   \aitem If the director is not interested in conciseness, then part 2 will be 
   of similar length as part 1. \texpl{premise}

   \aitem Part 2 will be of similar length as part 1. \texpl{4,5,MP}

   \aitem The whole is more than five hours long. \texpl{1,6, math}

   \aitem If the whole is more than five hours long, then it is too long.  
   \texpl{premise}

   \aitem It is too long. \texpl{7,8,MP}

  \end{argument*}

 }


\item \emph{Background: In 1839,  Lin Zexu, a high official of China, published 
  an open letter to Queen Victoria of England calling for the cessation of the 
  sales of opium in China by British merchants. (The letter went nowhere, China 
  confiscated the opium stored by traders, and the British Empire responded by 
 going to war with China in the name of defending the right to free trade---the 
disgrace known as the Opium War.)}

  Put the reasoning in the following snippet into standard form:

  We have heard that in your own country opium is prohibited with the utmost
strictness and severity; this is a strong proof that you know full well how 
hurtful it is to mankind. Since then you do not permit it to injure your own 
country, you ought not to have the injurious drug transferred to another 
country.

\answer{
 
 The conclusion is \pp{you ought not to have the injurious drug transferred to 
 another country}. The explanation for that is \pp{you do not permit opium to 
 injure your own country}. Here is a reconstruction:

 \begin{argument*}

  \aitem Opium is prohibited with the utmost strictness and severity in your 
  country.  \texpl{premise}

  \aitem You prohibit opium with the utmost strictness and severity in your 
  country only if you know how hurtful opium is to mankind. \texpl{premise}

  \aitem You know how hurtful opium is to mankind. \texpl{1,2,MP}

  \aitem If opium is prohibited strictly and severely and you know how hurtful 
  opium is, then you do not permit opium to injure your own country.  
  \texpl{premise}

  \aitem You do not permit opium to injure your own country. \texpl{3,4,MP}

  \aitem If you do not permit opium to injure own own country, then you ought 
  not to permit it to injure another country. \texpl{premise}

  \aitem You ought not to permit opium to injure another country. \texpl{5,6,
  MP}

  \aitem If you ought not to permit opium to injure another country, then you 
  ought not to have opium transferred to another country. \texpl{premise}

  \aitem You ought not to have opium transferred to another country. \texpl{7,8,
  MP}

 \end{argument*}

}


\item \emph{The idea that prohibiting the sales of opium is a serious 
 infringement of liberty was a respectable idea back then. The following are the 
thoughts of John Stuart Mill, one of the most important and respected British 
philosophers of the 19th century, writing in 1859.}

 Put Mill's reasoning in standard form:

 These interferences [like the prohibition of the sales of opium] are 
 objectionable, not as infringements on the liberty of the producer or seller, 
 but on that of the buyer. ...  If poisons [like opium]  were never bought or 
 used for any purpose except the commission of murder, it would be right to 
 prohibit their manufacture and sale.  They may, however, be wanted not only for 
 innocent but for useful purposes, and restrictions cannot be imposed in the one 
 case without operating in the other.  ...  [W]hen there is not a certainty, but 
 only a danger of mischief, no one but the person himself can judge of the 
 sufficiency of the motive which may prompt him to incur the risk: in this case, 
 therefore, he ought, I conceive, to be only warned of the danger; not forcibly 
 prevented from exposing himself to it.

\answer{

The conclusion is that prohibition of the sales of opium infringes on the 
liberty of the buyer. 

\begin{argument*}

 \aitem If effects of opium are not certain and it is impossible to regulate 
 only the bad cases, then only the person himself is allowed to judge whether or 
 not to incur the risk of taking opium.  \texpl{premise}

 \aitem If only the person himself is allowed to judge whether or not to incur 
 the risk of taking opium, then the prohibition of the sales of opium infringes 
 on the liberty of the buyer. \texpl{premise}

 \aitem Opium is used for evil, innocent, and useful purposes. \texpl{premise}

 \aitem If opium is used for evil, innocent, and useful purposes, then effects 
 of opium are uncertain. \texpl{premise}

 \aitem The effects of opium are uncertain. \texpl{3,4,MP}

 \aitem It is impossible to regulate only the bad cases of opium use.  
 \texpl{premise}

 \aitem The effects of opium are uncertain and it is impossible to regulate only 
 the bad cases of opium use. \texpl{from 5,6}

 \aitem Only the person himself is allowed to judge whether or not to incur the 
 risk of taking opium. \texpl{1,7,MP}

 \aitem The prohibition of sales of opium infringes on the liberty of the buyer.  
 \texpl{2,6,MP}

\end{argument*}
}


\end{enumerate}

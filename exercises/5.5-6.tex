
\subsection*{Exercises for 5.5--5.6}
\begin{enumerate}
 \setlength{\itemsep}{2.5em}
\item Consider the following formula:

 \p{\lforall x\big[(Px\lor Sx)\limplies Cx\big]}
 
\begin{enumerate}
 \item Construct an interpretation that makes the above formula true.


 \item Construct an interpretation that makes the above formula false.
 
\end{enumerate}
  \answer{ 

   The following models are just examples.

   \begin{enumerate}
	\item An intuitive interpretation: Let \p{Px} mean that x is a Pomona 
	 student, \p{Sx} mean that x is a Scripps student, and \p{Cx} mean that x is 
	 a 5C student. On this interpretation, the above formula is true.

   A formal model: 

   \begin{itemize}
	\item Domain of discourse: Charlie, Logan, Rowan, Sidney.

	\item Referent of constants: \p{c} refers to Charlie, \p{l} refers to Logan, 
	 \p{r} refers to Rowan, \p{s} refers to Sidney.

	\item Extension of \p{P}: Charlie and Logan. Extension of \p{S}: Rowan and 
	 Sidney. Extension of \p{C}: Charlie, Logan, Rowan, and Sidney.

   \end{itemize}

  \item An intuitive interpretation: Let \p{Px} mean that x is a Pitzer student, 
   \p{Sx} mean that x is a Scripps student, and \p{Cx} mean that x is a CMC 
   student. On this interpretation, the above formula is false.

   A formal model:

   \begin{itemize}
	\item Domain of discourse: Charlie, Logan, Rowan, Sidney.

	\item Referent of constants: \p{c} refers to Charlie, \p{l} refers to Logan, 
	 \p{r} refers to Rowan, \p{s} refers to Sidney.

	\item Extension of \p{P}: Charlie and Logan. Extension of \p{S}: Rowan and 
	 Sidney. Extension of \p{C}: empty.

   \end{itemize}
 \end{enumerate}


  }

\item Consider the following formula:

 \p{Pa \limplies \forall x(Rx\limplies \lnot Pa)}

 \begin{enumerate}

  \item Construct an interpretation that makes the formula true.

  \item Construct an interpretation that makes the formula false.

 \end{enumerate}

\answer{

 \begin{enumerate}

  \item Notice that the whole formula is a conditional so that if \p{Pa} is 
   false, the whole is true. We therefore only need to make sure \p{Pa} is false 
   in our model.

   \begin{itemize}

	\item Domain of discourse: Ben, Jerry.
	\item Referents of constants: \p{a} refers to Ben, \p{b} refers to Jerry.
	\item Extension of \p{P}: empty. Extension of \p{R}: Ben and Jerry.
   \end{itemize}

  \item \p{Pa} must be true, and \p{\lforall x(Rx\limplies\lnot Pa)} has to be 
   false. The latter requires that \p{R\kappa\limplies\lnot Pa} be false for 
   some constant \p{\kappa}.

   \begin{itemize}

	\item Domain of discourse: Ben, Jerry.
	\item Referents of constants: \p{a} refers to Ben, \p{b} refers to Jerry.
	\item Extension of \p{P}: Ben. Extension of \p{R}: Ben and Jerry.
	
   \end{itemize}

 \end{enumerate}

}


\item Consider the following formula:

 \p{\lforall x\lforall y(Rxy \limplies Ryx)}

 \begin{enumerate}

  \item Construct an interpretation that makes the formula true.

  \item Construct an interpretation that makes the formula false.

 \end{enumerate}

\answer{

 \begin{enumerate}

  \item Intuitively, any interpretation of \p{R} as a symmetric relationship 
   makes the formula true.

   \begin{itemize}

	\item Domain of discourse: Jill, Joe, George, Laura.
	\item Referents of constants: \p{j_1} refers to Jill, \p{j_2} refers to Joe, 
	 \p{g} refers to George, \p{l} refers to Laura.
	\item Extension of \p{R}: the ordered pairs <Jill, Joe>, <Joe,Jill>, <George,
	 Laura>, <Laura, George>.

   \end{itemize}
	(Think of \p{Rxy} as meaning x is married to y.)

	\item Intuitively, any interpretation of \p{R} as an asymmetric relationship 
	 makes the formula false.

	 \begin{itemize}
	  \item Domain of discourse: Kamala, Mike.
	  \item Referents of constants: \p{k} refers to Kamala, \p{m} refers to 
	   Mike.
	  \item Extension of \p{R}: the ordered pair <Kamala, Mike>.
	 \end{itemize}
	(Think of \p{Rxy} as meaning x is the successor of y as vice president.)
  \end{enumerate}

 }

\item Consider the following formula:

 \p{\lthereis xFx \limplies \lnot\lforall x\lnot Fx}

 Explain why it is not possible to construct an interpretation that makes this formula 
 false.
\answer{

 Since the formula is a conditional, its being false require the truth of 
 \p{\lthereis xFx} and the falsity of \p{\lnot\lforall x\lnot Fx}. If 
 \p{\lthereis xFx} is true, then the extension of \p{F} has at least one member.  
 On the other hand, if \p{\lnot\lforall x\lnot Fx} is false, then \p{\lforall 
 x\lnot Fx} is true which means that the extension of \p{F} is empty. Since the 
 extension of \p{F} cannot be both empty and non-empty, there can be no model 
 that makes the above formula false.

}


\item Consider:

 \p{\lthereis xFx \land \lforall x\lnot Fx}

 Explain why it is not possible to construct an interpretation that makes this formula 
 true.
\answer{

 Since the sentence is a conjunction, its truth requires the truth of 
 \p{\lthereis xFx} and the truth of \p{\lforall x\lnot Fx}. But no model can 
 make both true (see above for a bit more detail).
}





\item Which of the following are logical truths? For those that are not,
  provide counterexamples. (A counterexample is a quick of providing an interpretation 
  that makes the sentence false. For instance, consider \lbh{-p -f 
  CUxKMxHxAUxMxUxHx}. Here is a counterexample: suppose everyone in class is a 
  CMC or a HMC student. It doesn't follow that everyone is a CMC student or 
  everyone is a HMC student---maybe there's a mix of the two groups).

 \begin{enumerate}



 \item \lbh{-p -f CUxCFxGxCUxFxUxGx}
   
  \item \lbh{-p -f CXxCFxGxCXxFxXxGx}
   
  \item \lbh{-p -f CKXxFxXxGxXxKFxGx}
   
  \item \lbh{-p -f CXxKFxGxKXxFxXxGx}
   
 \end{enumerate}
\answer{

 (a) and (d) are logical truths. 

 (b), (c) are not logical truths. You can see that by constructing 
 counterexamples.  E.g.:
 (b) It might that there is someone who gets grumpy when hungry, but that does 
 not mean that if there is someone hungry, there is someone grumpy (the hungry 
 person need not be one who gets grumpy when hungry).

 (c) There are vegans, and there are carnivores. It does not follow that there 
 are some who are both vegan and carnivore.
 

}
\end{enumerate}

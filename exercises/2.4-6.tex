\section*{Exercises for 2.4--2.6}

\begin{enumerate}

 \item Produce the syntax tree for each of the following:

  \begin{enumerate}

   \item \plsh{KKPQAPQ}

	\answer{{\plsts{KKPQAPQ}}}
   
   \item \plsh{AKSTNR}

	\answer{{\plsts{AKSTNR}}}

   \item \plsh{KPKQR}

	\answer{{\plsts{KPKQR}}}
   
   \item \plsh{KARST}

	\answer{{\plsts{KARST}}}

   \item \plsh{ARKST}

	\answer{{\plsts{ARKST}}}

   \item \plsh{ANAPRKQS}

	\answer{{\plsts{ANAPRKQS}}}

  \end{enumerate}

 \item Give the truth table for \plsh{KPKQR}.
  
  \answer{{\pltt{KPKQR}}.}



 \item Give the truth table for \plsh{KARST}.

  \answer{{\pltt{KARST}}}

 \item Give the truth table for \plsh{ARKST}.

  \answer{{\pltt{ARKST}}}


 \item Give the truth table for \plsh{NKPQ}.

  \answer{{\pltt{NKPQ}}}

 \item Give the truth table for \plsh{ANPNQ}.

  \answer{{\pltt{ANPNQ}}}

 \item Give the truth table for \plsh{NARS}.

  \answer{{\pltt{NARS}}}

 \item Give the truth table for \plsh{KNRNS}.

  \answer{{\pltt{KNRNS}}}

 \item Show, using a truth table, that \plsh{APNP} is a tautology.

  \answer{{\pltt{APNP}}}

 \item Show, using a truth table, that \plsh{KPNP} is a contradiction.

  \answer{{\pltt{KPNP}}}

 \item Explain why any sentence of the form \p{s \lor \lnot s} is a tautology 
  (even if \p{s} is not an atomic sentence).

  \answer{ 

   Given any interpretation, \p{s} is either true of false. If \p{s} is true in 
   that interpretation, \p{s\lor \lnot s} is also true in that interpretation.  
   If \p{s} is false, then \p{\lnot s} is true, so \p{s\lor \lnot s} is true in 
   that interpretation. So \p{s \lor \lnot s} is true in every interpretation,  
   which means that \p{s \lor \lnot s} is a tautology.

  }
  \newcommand{\ansF}[1]{\dotfill True/\anscircled{False}}
 \newcommand{\ansT}[1]{\dotfill \anscircled{True}/False}

 \item True or False?

  \begin{enumerate}

   \item If one disjunct of a disjunction is a tautology, then the whole 
	disjunction is a tautology.

	\ansT

   \item If a disjunction is a tautology, one of the disjuncts is a tautology.

	\ansF

   \item The negation of a contradiction is a tautology.

	\ansT

   \item  \p{s_1 \land s_2} is consistent iff. \p{s_1} and \p{s_2} are 
	consistent with each other.

	\ansT

   \item If \p{s_1} and \p{s_2} are inconsistent with each other, then \p{s_1 
	\lor s_2} is inconsistent.

	\ansF

  \end{enumerate}

\end{enumerate}

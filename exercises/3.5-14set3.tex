
\newcommand{\lbhelper}[1]{\input{|"lbhelper #1"}}
 

In \emph{Furman v. Georgia} (1972) the U.S. Supreme Court ruled that all death 
penalty schemes extant at that point were unconstitutional. This forced a 
moratorium on capital punishment in the United States.

In a concurring opinion, Thurgood Marshall---first African American 
justice---argued for an even stronger position: capital punishment in any shape 
or form violates the constitution; i.e., no amount of reform can make it 
constitutionally acceptable. 

The following is a heavily abridged version of Marshall's opinion which we will 
formalize as a derivation.

\begin{quote}
 \textbf{Marshall's concurring opinion in Furman v. Georgia}


It is immediately obvious ...that ... if [capital punishment]  violates the 
Constitution, it does so because it is excessive or unnecessary, or because it 
is abhorrent to currently existing moral values.

...
 
 In order to assess whether or not death is an excessive or unnecessary penalty, 
 it is necessary to consider the reasons why a legislature might select it as 
 punishment for one or more offenses, and examine whether less severe penalties 
 would satisfy the legitimate legislative wants as well as capital punishment.  
 If they would, then the death penalty is unnecessary cruelty, and, therefore, 
 unconstitutional.

 There are six purposes conceivably served by capital punishment: retribution, 
 deterrence, prevention of repetitive criminal acts, encouragement of guilty 
 pleas and confessions, eugenics, and economy. These are considered seriatim 
 below.

... 

 The history of the Eighth Amendment supports only the conclusion that 
 retribution for its own sake is improper.

...
 In light of the massive amount of evidence before us, I see no alternative but 
 to conclude that capital punishment cannot be justified on the basis of its 
 deterrent effect.

 ... In light of these facts, if capital punishment were justified purely on the 
 basis of preventing recidivism, it would have to be considered to be excessive 
 ...

Since life imprisonment is sufficient for bargaining purposes, the death penalty 
is excessive if used for the same purposes [i.e., encouraging guilty pleas and 
confessions].
...

 I can only conclude, as has virtually everyone else who has looked at the 
 problem, that capital punishment cannot be defended on the basis of any eugenic 
 purposes.
 ...

 When all is said and done, there can be no doubt that it costs more to execute 
 a man than to keep him in prison for life.

 There is but one conclusion that can be drawn from all of this—i.e., the death 
 penalty is an excessive and unnecessary punishment that violates the Eighth 
 Amendment. ... 


 In addition, even if capital punishment is not excessive, it nonetheless 
 violates the Eighth Amendment because it is morally unacceptable to the people 
 of the United States at this time in their history.

 ... the question with which we must deal is not whether a substantial 
 proportion of American citizens would today, if polled, opine that capital 
 punishment is barbarously cruel, but whether they would find it to be so in the 
 light of all information presently available.

... Assuming knowledge of all the facts presently available regarding capital 
punishment, the average citizen would, in my opinion, find it shocking to his 
conscience and sense of justice. For this reason alone capital punishment cannot 
stand.
 
 
 [You can get the whole verdict, which includes Marshall's opinion, online. Go 
 to the relevant Wikipedia page which will give you a number of links where you 
 can find the whole text.]

\end{quote}

Let's see how we could present this is a derivation. This has many premises, so 
you may want to use subscripts (e.g., \p{\Gamma_1, \Gamma_2}) to avoid running 
out of letters. All the lines must be in sequent form (we want to turn the 
argument into a formal derivation):

\begin{enumerate}[leftmargin=*]
 \setlength{\itemsep}{1.5em}
 \item Take the very first sentence of Marshall's argument. That's also the 
  first premise of his argument. It amounts to this: if capital punishment is 
  excessive /unnecessary or abhorrent to currently existing moral values, 
  capital punishment is unconstitutional. Use the keys below to turn this into 
  the first premise: \begin{lkey*}
  \setlength{\itemsep}{0.5em}
  \item[$E$] Capital punishment is excessive or unnecessary.
  \item[$A$] Capital punishment is abhorrent to currently existing moral values.
  \item[$U$] Capital punishment violates the Constitution.
  \end{lkey*}

 \item Let's move to the next block of text. His point: If there are punishments 
  less severe than capital punishment that serve the legislative purpose, then 
  capital punishment would be excessive or unnecessary.  This is the second 
  premise. Use the key below to turn this into the second premise:
\begin{lkey*}
\item[$L$] There are punishments less severe than capital punishment that serve 
the legislative purpose.  \end{lkey*}



\item Marshall now lists six claims such that if any of them is true, it would 
 show  that capital punishment is the only punishment that serves the 
 legislative purpose.  They are:

 \begin{lkey*}
  \setlength{\itemsep}{0.5em}
 \item[$J_1$] Capital punishment is justified for purely retributive reasons.

 \item[$J_2$] Capital punishment is justified because it deters future crime.

 \item[$J_3$] Capital punishment is justified because it prevents recidivism 
  (repeat offense).

 \item[$J_4$] Capital punishment is justified because it encourages guilty pleas 
  and confessions.

 \item[$J_5$] Capital punishment is justified for eugenic reasons (i.e., it 
  improves the quality of the gene pool).

 \item[$J_6$] Capital punishment is justified because it is cheaper than life 
  imprisonment.
 \end{lkey*}

 State, as your next premise  the relationship between the claims $J_1$ through 
 $J_6$ and $L$. Be careful to consider how the overall argument is supposed to 
 work.


\item He now goes through each of $J_1$ through $J_6$ and says that each of them 
 is false.  These are further premises.  Add them to your derivation.

 
\item Describe how you can get to a sequent that says that the premises support 
 \p{L}. Write the sequent you would reach.
 
 Don't try to write out the derivation in full as it will take too long. Rather, 
 when you think about you will see that the derivation will be very repetitive, 
 and just describe what needs to be repeated to reach what we want. You may 
 appeal to points and derivations we have seen in previous exercises and the 
 like.
 
\item Continue the derivation to the conclusion that the premises show that 
 capital punishment is unconstitutional.
 
\item He has now established that capital punishment is unconstitutional. But he 
 has another argument for the same conclusion. That is in the last part of his 
 argument. That argument establishes that capital punishment is unconstitutional 
 because it is morally unacceptable. Turn this part of the argument into a 
 derivation. In addition to any keys from above use the following keys:
 \begin{lkey*}
  \setlength{\itemsep}{0.5em}
 \item[$K$] American citizens are informed of all facts presently available 
  regarding capital punishment.

 \item[$S$] The average American citizen finds capital punishment shocking to 
  his sense of conscience and sense of justice.
 \end{lkey*}

 \end{enumerate}

\remarkbox{

Just for your edification, here are the concluding  paragraphs of Marshall's 
opinion:

\begin{quote}

At a time in our history when the streets of the Nation's cities inspire fear 
and despair, rather than pride and hope, it is difficult to maintain objectivity 
and concern for our fellow citizens. But the measure of a country's greatness is 
its ability to retain compassion in time of crisis. No nation in the recorded 
history of man has a greater tradition of revering justice and fair treatment 
for all its citizens in times of turmoil, confusion, and tension than ours. This 
is a country which stands tallest in troubled times, a country that clings to 
fundamental principles, cherishes its constitutional heritage, and rejects 
simple solutions that compromise the values that lie at the roots of our 
democratic system.

In striking down capital punishment, this Court does not malign our system of 
government. On the contrary, it pays homage to it. Only in a free society could 
right triumph in difficult times, and could civilization record its magnificent 
advancement. In recognizing the humanity of our fellow beings, we pay ourselves 
the highest tribute. We achieve a major milestone in the long road up from 
barbarism and join the approximately 70 other jurisdictions in the world which 
celebrate their regard for civilization and humanity by shunning capital 
punishment.

\end{quote}

The Supreme Court reinstated the death penalty in 1976 against Marshall's 
objections. The U.S. is still a regular in the top five of the annual world 
rankings of judicial executions. Meanwhile, the most recent count of countries 
that have abolished the death penalty is 144.  

}

\answer{

 Here are the premises:

 \begin{argument}

  \ai{\Gamma_1}{(E\lor A)\limplies U}{premise}

  \ai{\Gamma_2}{L\limplies E}{premise}

  \ai{\Gamma_3}{\lnot L \limplies (J_1 \lor (J_2 \lor (J_3 \lor (J_4 \lor (J_5 
  \lor J_6)))))}{premise}

  \ai{\Gamma_4}{\lnot J_1}{premise}

  \ai{\Gamma_5}{\lnot J_2}{premise}
 
  \ai{\Gamma_6}{\lnot J_3}{premise}
 
  \ai{\Gamma_7}{\lnot J_4}{premise}
 
  \ai{\Gamma_8}{\lnot J_5}{premise}
 
  \ai{\Gamma_9}{\lnot J_6}{premise}

 \end{argument}

 Notice that for line 3, it is tempting to say:

 \p{\seq{\Gamma_3}{(J_1 \lor (J_2 \lor (J_3 \lor (J_4 \lor (J_5 \lor J_6)))))
 \limplies \lnot L}}.

 But this will not for the argument since the denial of \p{J_1} through \p{J_6} 
 won't allow us to move forward. The adverb `conceivably' that Marshall uses is 
 clear indication he thinks that \p{J_1} through \p{J_6} are the \emph{only} 
 ways the death penalty could be justified. That can be captured by our  line 3.
 
How do we move forward from here? Notice that we have shown earlier that we can 
derive from \p{\seq{\Gamma}{P\limplies (Q\lor R)}} and \p{\seq{\Delta}{\lnot Q}} 
to \p{\seq{\Gamma,\Delta}{P\limplies R}} (last problem on the exercise for Oct.   
10). Line 3 is of the form \p{\seq{\Gamma}{P \limplies (Q\lor R)}}: plug 
\p{\lnot L} into \p{P}, \p{J_1} into \p{Q}, \p{J_2 \lor (J_3 \lor (J_4\lor (J_5 
\lor J_6)))} into \p{R}, \p{\Gamma_3} into \p{\Gamma}, and \p{\Gamma_4} into 
\p{\Delta}.  Together with line 4, we will get:

\p{\seq{\Gamma_3,\Gamma_4}{J_2\lor(J_3\lor (J_4 \lor (J_5 \lor J_6)))}}.

 This again has the form \p{\seq{\Gamma}{P\limplies (Q\lor R)}}. So we can 
 repeat this. In fact, we can repeat this five times to end with:

 \p{\seq{\Gamma_3, \ldots, \Gamma_8}{\lnot L \limplies J_6}}.

 We know from the readings that we can derive from \p{\seq{\Gamma}{P\limplies Q}} 
 and \p{\seq{\Delta}{\lnot Q}} to \p{\seq{\Gamma,\Delta}{\lnot P}}. So we can 
 derive from the above with line 9 to:

 \p{\seq{\Gamma_3, \ldots, \Gamma_9}{\lnot\lnot L}}.

Let's this be our line 10:

\begin{argument}

 \ai{\Gamma_3, \ldots, \Gamma_9}{\lnot\lnot L}{as just described}

\end{argument}

From here we can continue:

\begin{argument}

 \ai{\Gamma_3, \ldots, \Gamma_9}{L}{10,\negE}

 \ai{\Gamma_2, \ldots, \Gamma_9}{E}{2,11,\condE}

 \ai{\Gamma_2, \ldots, \Gamma_9}{E\lor A}{12,\disjI}

 \ai{\Gamma_1, \ldots, \Gamma_9}{U}{1,13,\condE}

\end{argument}


Now to the second argument:

\begin{argument*}

 \ai{\Gamma_1}{(E \lor A)\limplies U}{premise}

 \ai{\Gamma_2}{(K\limplies S)\limplies A}{premise}

 \ai{\Gamma_3}{K \limplies S}{premise}

 \ai{\Gamma_2, \Gamma_3}{A}{2,3,\condE}

 \ai{\Gamma_2, \Gamma_3}{E \lor A}{4,\disjI}

 \ai{\Gamma_1,\Gamma_2, \Gamma_3}{U}{1,5,\condE}

\end{argument*}

Notice that it is important to make sure that there is no \p{K} in the datum of 
the concluding line. In fact, there must be no non-Greek letter in the datum of 
the conclusion, or else we would be attributing a fallacious argument to 
Marshall.  So even though Marshall sounds like he is assuming \p{K}, we need to 
make sure we interpret him in such a way that that is not what he is doing. Line 
3 does that.

}








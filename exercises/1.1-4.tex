\section*{Exercises for 1.1 through 1.4}

\begin{enumerate}

 \item Which of the following are sentences in the sense used in this course?

  \begin{enumerate}

   \item Who wants to live forever?

   \item I wonder whether anyone wants to live forever.

   \item Life might have existed on Mars.

   \item If God is dead, what is the point of morality?

   \item If it is true that God is dead, what is the point of morality?

   \item Can you close the window for me?

   \item I would like you to close the window.

   \item Please close the window.

  \end{enumerate}
\answer{

 b, c, g
}

 \item Characterize each of the following as a compound of two atomic sentences 
  (there might be multiple ways of doing it).

  \begin{itemize}

   \item Some countries are resource rich and some other countries are resource 
	poor.

   \item Venus shares many characteristics with Earth but Venus is 
	uninhabitable.

   \item If you have taken PHIL60, then you have fulfilled the Area 5 
	requirement.

   \item The Taj Mahal is a much admired architectural complex but some people 
	consider it kitschy and some politicians claim that it does not represent 
	Indian culture.

  \end{itemize}

\answer{
  
 \begin{itemize}

  \item \pp{Some countries are resource rich} and \pp{some other countries are 
   resource poor}.

  \item \pp{Venus shares many characteristics with Earth} but \pp{Venus is 
   uninhabitable}.

  \item \pp{If you have taken PHIL60}, then \pp{you have fulfilled the Area 5 
   requirement}.

   \item
	
	\begin{enumerate} 
	 
	 \item \pp{The Taj Mahal is a much admired architectural complex} but 
	  \pp{some people consider it kitschy and some politicians claim that it 
	  does not represent Indian culture}.   \vskip 0.75em

	  OR:

   \item \pp{The Taj Mahal is a much admired architectural complex but some 
	people consider it kitschy} and \pp{some politicians claim that it does not 
	represent Indian culture}.
  \end{enumerate}
  \end{itemize}

 }

 \item Sometimes a sentence that does not look like a compound sentence can be 
  reformulated as a compound sentence. E.g., ``Marie Curie was a Polish 
  scientist'' can be reformulated as ``Marie Curie was Polish and Marie Curie 
  was a scientist''. Turn each of the following into a compound sentence with 
  two atomic sentences.

  \begin{enumerate}

   \item Tokyo in the summer is hot and humid.

   \item Cuneiform was used to write Sumerian as well as Akkadian.

   \item The Holy Roman Empire was neither holy nor Roman.

   \item The culprit was the butler or the cook.

   \item You can satisfy the Area 5 requirement by taking logic or by taking 
	calculus.

  \end{enumerate}

  \answer{

   \begin{enumerate}

	\item \pp{Tokyo in the summer is hot} and \pp{Tokyo in the summer is humid}.  

	\item \pp{Cuneiform was used to write Sumerian} and \pp{cuneiform was used 
	 to write Akkadian}.

	\item \pp{The Holy Roman Empire was not holy} and \pp{the Holy Roman Empire 
	 was not Roman}.

	\item \pp{The culprit was the butler} or \pp{the culprit was the cook}.

	\item \pp{You can satisfy the Area 5 requirement by taking logic} and 
	 \pp{you can satisfy the Area 5 requirement by taking calculus}.


	 \textbf{ Note on (e):}



Original: You can satisfy the Area 5 requirement by taking logic or by taking 
calculus.

A: You can satisfy the Area 5 requirement by taking logic

B: you can satisfy the Area 5 requirement by taking calculus
 
\begin{minipage}{0.3\txw}
 \begin{tabular}{cc||c}
\p{A} & \p{B} & A or B\\
\hline
\emph{T} & \emph{T}  & \emph{T}\\
\emph{F} & \emph{T}  & \emph{T}\\
\emph{T} & \emph{F}  & \emph{T}\\
\emph{F} & \emph{F}  & \emph{F}\\
\end{tabular}
\end{minipage}
\begin{minipage}{0.3\txw}
 \begin{tabular}{cc||c}
\p{A} & \p{B} & A and B\\
\hline
\emph{T} & \emph{T}  & \emph{T}\\
\emph{F} & \emph{T}  & \emph{F}\\
\emph{T} & \emph{F}  & \emph{F}\\
\emph{F} & \emph{F}  & \emph{F}\\
\end{tabular}
\end{minipage}
\begin{minipage}{0.3\txw}
 \begin{tabular}{cc||c}
\p{A} & \p{B} & Orig.\\
\hline
\emph{T} & \emph{T}  & \emph{T}\\
\emph{F} & \emph{T}  & \emph{F}\\
\emph{T} & \emph{F}  & \emph{F}\\
\emph{F} & \emph{F}  & \emph{F}\\
\end{tabular}

\end{minipage}
The third table gives the natural understanding of the original and it is the 
same as \pp{A and B}, not \pp{A or B}.
\end{enumerate}

  }


	\item State the truth conditions for \pp{neither A nor B} in table form.

	 \answer{
\begin{tabular}{cc||c}
\p{A} & \p{B} & neither A nor B\\
\hline
\emph{T} & \emph{T}  & \emph{F}\\
\emph{F} & \emph{T}  & \emph{F}\\
\emph{T} & \emph{F}  & \emph{F}\\
\emph{F} & \emph{F}  & \emph{T}\\
\end{tabular}

}
 \item State the truth conditions for \pp{A but not B} in table form.
\answer{
 \begin{tabular}{cc||c}
\p{A} & \p{B} & A but not B\\
\hline
\emph{T} & \emph{T}  & \emph{F}\\
\emph{F} & \emph{T}  & \emph{F}\\
\emph{T} & \emph{F}  & \emph{T}\\
\emph{F} & \emph{F}  & \emph{F}\\
\end{tabular}
}

\end{enumerate}



\begin{enumerate}
 \setlength{\leftmarginii}{1em}
 \setlength{\leftmarginiii}{0.1em}

 \item For each of the following, indicate whether or not the inference is 
  allowed.  Yes means allowed. \p{a}, \p{b} are constants, and they do not 
  appear in \p{\Gamma} and \p{\Delta}.

  \begin{enumerate}

   \item ~ 

\begin{argumentN}[1]
%generated by gentzen

\ai{Fa}{Fa}{A}

\ai{Fa}{Gb\mc{\lor }Fa}{1,\disjI}

\ai{Fa}{\lthereis x Gx\mc{\lor }Fa}{2,\exI}

\end{argumentN}

\dotfill Yes/\anscircled{No}

\item ~


\begin{argumentN}[1]
%generated by gentzen

\ai{Γ}{\mc{\lthereis x }(Gx\land Fx)}{premise}

\ai{Ga\mc{\land }Fa}{\mc{\lthereis x }Gx}{premise}

\ai{Γ}{\mc{\lthereis x }Gx}{1,2,\exE}
\end{argumentN}

\dotfill \anscircled{Yes}/No

\item ~
 \begin{argumentN}[1]

\ai{Γ}{\mc{\lthereis x }(Gx\land Fx)}{premise}

\ai{Γ, Ga\mc{\land }Fa}{\lthereis x Gx\mc{\land }Fa}{premise}

\ai{Γ}{\lthereis x Gx\mc{\land }Fa}{1,2,\exE}

\end{argumentN}

\dotfill Yes/\anscircled{No}

\item ~

\begin{argumentN}[1]
%generated by gentzen

\ai{Fa}{Fa}{A}

\ai{Fa}{Gb\mc{\lor }Fa}{1,\disjI}

\ai{Fa}{\mc{\lthereis x }(Gx\lor Fa)}{2,\exI}

\end{argumentN}
\dotfill \anscircled{Yes}/No

\item ~

\begin{argumentN}[1]
%generated by gentzen

\ai{Fa}{Fa}{A}

\ai{Fa}{Gb\mc{\lor }Fa}{2,\disjI}

\ai{Fa}{\mc{\lthereis x }Gx\lor Fa}{2,\exI}

\end{argumentN}

\dotfill Yes/\anscircled{No}


\item ~

\begin{argumentN}[1]
%generated by gentzen

\ai{\mc{\lforall x }(Fx\limplies Ga)}{\mc{\lforall x }(Fx\limplies Ga)}{A}

\ai{\mc{\lforall x }(Fx\limplies Ga)}{Fa\mc{\limplies }Ga}{1,\uniE}

\ai{\mc{\lforall x }(Fx\limplies Ga)}{\mc{\lforall x }(Fx\limplies Gx)}{2,\uniI}

\end{argumentN}

\dotfill Yes/\anscircled{No}


\newpage
\item ~

\begin{argumentN}[1]
%generated by gentzen

\ai{Γ}{\mc{\lforall x }Fx}{premise}

\ai{Δ}{\mc{\lforall y }Gy}{premise}

\ai{Γ, Δ}{\mc{\lforall x }\lforall y (Fx\land Gy)}{1,2,\conjI}

\end{argumentN}

\dotfill Yes/\anscircled{No}


\item ~

\begin{argumentN}[1]
%generated by gentzen

\ai{Γ}{\lforall x Fx\mc{\land }Ga}{premise}

\ai{Γ}{\mc{\lforall x }\lforall y (Fx\land Gy)}{1,\uniI}

\end{argumentN}
\dotfill Yes/\anscircled{No}

\item ~
  

\begin{argumentN}[1]
%generated by gentzen

\ai{Γ}{\lforall x( Fx\mc{\land }Ga)}{premise}

\ai{Γ}{\mc{\lforall y }\lforall x (Fx\land Gy)}{1,\uniI}

\end{argumentN}

\dotfill \anscircled{Yes}/No

\item  ~


\begin{argumentN}[1]
%generated by gentzen

\ai{Γ, \mc{\lthereis x }\lnot Fx}{\mc{\lthereis x }Mx}{premise}

\ai{Δ, \mc{\lthereis x }\lnot Fx}{\mc{\lnot }\lthereis x Mx}{premise}

\ai{Γ, Δ}{\mc{\lforall x }Fx}{1,2,\negI}

\end{argumentN}

\dotfill Yes/\anscircled{No}



\newpage



\end{enumerate}


 \item Let's prove the theorems known as Quantifier Exchange. Here are the first 
  four.

  For these four, do not appeal to Quantifier Exchange anywhere. Notice we stop 
  one step short of the actual theorem.  The theorems in questions are 
  conditionals of the form \p{s_1\limplies s_2}.  Instead of proving that, we 
  stop at \p{s_1\lproves s_2} because the we can get to the conditional 
  trivially by \condI. But the actual theorem is still the conditional!



  \begin{enumerate}
 \setlength{\itemsep}{1em}
   \item \lbh{-p -s XxFx:NUxNFx}. The proof of this is in the readings.  
	See if you can reproduce it.

	\opts{
\dotline
\dotline
\dotline
\dotline
\dotline
\dotline
\dotline
\dotline
\dotline

}
{
\begin{argumentN}[1]
%generated by gentzen

\ai{\mc{\lthereis x }Fx}{\mc{\lthereis x }Fx}{A}

\ai{\mc{\lforall x }\lnot Fx}{\mc{\lforall x }\lnot Fx}{A}

\ai{Fa}{Fa}{A}

\ai{\mc{\lforall x }\lnot Fx}{\mc{\lnot }Fa}{2,\uniE}

\ai{Fa, \mc{\lforall x }\lnot Fx}{Fa}{3}

\ai{Fa}{\mc{\lnot }\lforall x \lnot Fx}{4,5,\negI}

\ai{\mc{\lthereis x }Fx}{\mc{\lnot }\lforall x \lnot Fx}{1,6,\exE}

\end{argumentN}
}


\item \lbh{-p -s NUxNFx:XxFx}. Add the missing annotations (this is a little 
 more efficient than the one in an earlier  exercise). 

\begin{argumentN}[1]
%generated by gentzen

\ai{\mc{\lnot }\lforall x \lnot Fx}{\mc{\lnot }\lforall x \lnot Fx}{A}

\ai{\mc{\lnot }\lthereis x Fx}{\mc{\lnot }\lthereis x Fx}{A}

\ai{Fa}{Fa}{A}

\ai{\mask{Fa}}{\mask{\mc{\lthereis x }Fx}}{\mask{3,\exI}}

\ai{}{Fa\mc{\limplies }\lthereis x Fx}{4,\condI}

\ai{}{\mask{(Fa\limplies \lthereis x Fx)\mc{\limplies }(\lnot \lthereis x 
Fx\limplies \lnot Fa)}}{\mask{CP}}

\ai{}{\mask{\lnot \lthereis x Fx\mc{\limplies }\lnot Fa}}{5,6,\condE}

\ai{\mc{\lnot }\lthereis x Fx}{\mc{\lnot }Fa}{2,7,\condE}

\ai{\mc{\lnot }\lthereis x Fx}{\mc{\lforall x }\lnot Fx}{\mask{8,\uniI}}

\ai{\mask{\mc{\lnot }\lforall x \lnot Fx, \mc{\lnot }\lthereis x Fx}}
{\mask{\mc{\lnot }
\lforall x \lnot Fx}}{1}

\ai{\mask{\mc{\lnot }\lforall x \lnot Fx}}{\mask{\mc{\lnot }\lnot \lthereis x 
Fx}}
{9,
 10,
\negI}

\ai{\mc{\lnot }\lforall x \lnot Fx}{\mc{\lthereis x }Fx}{\mask{11,\negE}}

\end{argumentN}

\newpage

\item \lbh{-p -s UxFx:NXxNFx}. Prove this sequent. Hint: assume the 
 sentence in the datum, and assume the denial of the succedent of this sequent.

 \opts{
\dotline
\dotline
\dotline
\dotline
\dotline
\dotline
\dotline
\dotline
\dotline
\dotline
\dotline
\dotline
\dotline
\dotline

 }
 {
\begin{argumentN}[1]
%generated by gentzen

\ai{\mc{\lforall x }Fx}{\mc{\lforall x }Fx}{A}

\ai{\mc{\lthereis x }\lnot Fx}{\mc{\lthereis x }\lnot Fx}{A}

\ai{\mc{\lnot }Fa}{\mc{\lnot }Fa}{A}

\ai{\mc{\lforall x }Fx}{Fa}{1,\uniE}

\ai{\mc{\lnot }Fa, \mc{\lthereis x }\lnot Fx}{\mc{\lnot }Fa}{3}

\ai{\mc{\lforall x }Fx, \mc{\lthereis x }\lnot Fx}{Fa}{4}

\ai{\mc{\lforall x }Fx, \mc{\lnot }Fa}{\mc{\lnot }\lthereis x \lnot Fx}{5,6,\negI}

\ai{\mc{\lforall x }Fx, \mc{\lthereis x }\lnot Fx}{\mc{\lnot }\lthereis x \lnot Fx}{2,7,\exE}

\ai{\mc{\lforall x }Fx}{\mc{\lnot }\lthereis x \lnot Fx}{2,8,\negI}

\end{argumentN}
}
   \item \lbh{-p -s NXxNFx:UxFx}. Prove this sequent. This was in an earlier 
	exercise. See if you can reproduce it.


	\opts{
\dotline
\dotline
\dotline
\dotline
\dotline
\dotline
\dotline
\dotline
\dotline
\dotline
\dotline

}
{
\begin{argumentN}[1]
%generated by gentzen

\ai{\mc{\lnot }\lthereis x \lnot Fx}{\mc{\lnot }\lthereis x \lnot Fx}{A}

\ai{\mc{\lnot }Fa}{\mc{\lnot }Fa}{A}

\ai{\mc{\lnot }Fa}{\mc{\lthereis x }\lnot Fx}{2,\exI}

\ai{\mc{\lnot }\lthereis x \lnot Fx, \mc{\lnot }Fa}{\mc{\lnot }\lthereis x \lnot Fx}{1}

\ai{\mc{\lnot }\lthereis x \lnot Fx}{\mc{\lnot }\lnot Fa}{3,4,\negI}

\ai{\mc{\lnot }\lthereis x \lnot Fx}{Fa}{5,\negE}

\ai{\mc{\lnot }\lthereis x \lnot Fx}{\mc{\lforall x }Fx}{6,\uniI}

\end{argumentN}
}
\end{enumerate}

\newpage
\item Let's prove the remaining theorems known as Quantifier Exchange. You may 
 appeal to the theorems proven in the previous question. Hint: for the latter 
 three, the first one below gives you something like a template.

\begin{enumerate}
 \setlength{\itemsep}{1em}

   \item \lbh{-p -s XxNFx:NUxFx}. Fill in the missing items.

\begin{argumentN}[1]
%generated by gentzen

 \ai{\mc{\lthereis x }\lnot Fx}{\mc{\lthereis x }\lnot Fx}{\mask{A}}

 \ai{\mc{\lforall x }Fx}{\mc{\lforall x }Fx}{\mask{A}}

 \ai{}{\lforall x Fx\mc{\limplies }\lnot \lthereis x \lnot Fx}{\mask{QE}}

 \ai{\mc{\lforall x }Fx}{\mc{\lnot }\lthereis x \lnot Fx}{\mask{2,3,\condE}}

\ai{\mc{\lthereis x }\lnot Fx, \mc{\lforall x }Fx}{\mc{\lthereis x }\lnot Fx}
{\mask{1}}

\ai{\mc{\lthereis x }\lnot Fx}{\mc{\lnot }\lforall x Fx}{\mask{4,5,\negI}}

\end{argumentN}

   \item \lbh{-p -s NUxFx:XxNFx}. Prove this sequent.

	\opts{
 \dotline
 \dotline
 \dotline
 \dotline
 \dotline
 \dotline
 \dotline
 \dotline
 \dotline
}
{
\begin{argumentN}[1]
%generated by gentzen

\ai{\mc{\lnot }\lforall x Fx}{\mc{\lnot }\lforall x Fx}{A}

\ai{\mc{\lnot }\lthereis x \lnot Fx}{\mc{\lnot }\lthereis x \lnot Fx}{A}

\ai{}{\lnot \lthereis x \lnot Fx\mc{\limplies }\lforall x Fx}{QE}

\ai{\mc{\lnot }\lthereis x \lnot Fx}{\mc{\lforall x }Fx}{2,3,\condE}

\ai{\mc{\lnot }\lforall x Fx, \mc{\lnot }\lthereis x \lnot Fx}{\mc{\lnot }\lforall x Fx}{1}

\ai{\mc{\lnot }\lforall x Fx}{\mc{\lnot }\lnot \lthereis x \lnot Fx}{4,5,\negI}

\ai{\mc{\lnot }\lforall x Fx}{\mc{\lthereis x }\lnot Fx}{6,\negE}

\end{argumentN}
}
\item \lbh{-p -s UxNFx:NXxFx}. Prove this.

	\opts{
 \dotline
 \dotline
 \dotline
 \dotline
 \dotline
 \dotline
 \dotline
 \dotline
 \dotline
}
{
\begin{argumentN}[1]
%generated by gentzen

\ai{\mc{\lforall x }\lnot Fx}{\mc{\lforall x }\lnot Fx}{A}

\ai{\mc{\lthereis x }Fx}{\mc{\lthereis x }Fx}{A}

\ai{}{\lthereis x Fx\mc{\limplies }\lnot \lforall x \lnot Fx}{QE}

\ai{\mc{\lthereis x }Fx}{\mc{\lnot }\lforall x \lnot Fx}{2,3,\condE}

\ai{\mc{\lforall x }\lnot Fx, \mc{\lthereis x }Fx}{\mc{\lforall x }\lnot Fx}{1}

\ai{\mc{\lforall x }\lnot Fx}{\mc{\lnot }\lthereis x Fx}{4,5,\negI}

\end{argumentN}
}

\newpage
\item \lbh{-p -s NXxFx:UxNFx}. Prove this.

	\opts{
 \dotline
 \dotline
 \dotline
 \dotline
 \dotline
 \dotline
 \dotline
 \dotline
 \dotline
}
{
\begin{argumentN}[1]
%generated by gentzen

\ai{\mc{\lnot }\lthereis x Fx}{\mc{\lnot }\lthereis x Fx}{A}

\ai{\mc{\lnot }\lforall x \lnot Fx}{\mc{\lnot }\lforall x \lnot Fx}{A}

\ai{}{\lnot \lforall x \lnot Fx\mc{\limplies }\lthereis x Fx}{QE}

\ai{\mc{\lnot }\lforall x \lnot Fx}{\mc{\lthereis x }Fx}{2,3,\condE}

\ai{\mc{\lnot }\lthereis x Fx, \mc{\lnot }\lforall x \lnot Fx}{\mc{\lnot }\lthereis x Fx}{1}

\ai{\mc{\lnot }\lthereis x Fx}{\mc{\lnot }\lnot \lforall x \lnot Fx}{4,5,\negI}

\ai{\mc{\lnot }\lthereis x Fx}{\mc{\lforall x }\lnot Fx}{6,\negE}

\end{argumentN}
}
  \end{enumerate}

  \newpage
 \item Let's see some cases of interactions between the conditional and 
  quantifiers. (Hint for constructing proofs in this section: The proofs have a 
  lot in common).
  \begin{enumerate}
   \setlength{\itemsep}{1em}

   \item \lbh{-p -s CUxFxUxGx:AXxNFxUxGx}. Add the missing annotations:


\begin{argumentN}[1]
%generated by gentzen

\ai{\lforall x Fx\mc{\limplies }\lforall x Gx}{\lforall x Fx\mc{\limplies }
\lforall x Gx}{\mask{A}}

\ai{}{(\lforall x Fx\limplies \lforall x Gx)\mc{\limplies }(\lnot \lforall x 
Fx\lor \lforall x Gx)}{\mask{IM}}

\ai{\lforall x Fx\mc{\limplies }\lforall x Gx}{\lnot \lforall x Fx\mc{\lor }
\lforall x Gx}{\mask{1,2,\condE}}

\ai{\mc{\lnot }\lforall x Fx}{\mc{\lnot }\lforall x Fx}{\mask{A}}

\ai{}{\lnot \lforall x Fx\mc{\limplies }\lthereis x \lnot Fx}{\mask{QE}}

\ai{\mc{\lnot }\lforall x Fx}{\mc{\lthereis x }\lnot Fx}{\mask{4,5,\condE}}

\ai{\mc{\lnot }\lforall x Fx}{\lthereis x \lnot Fx\mc{\lor }\lforall x Gx}
{\mask{6,
\disjI}}

\ai{\mc{\lforall x }Gx}{\mc{\lforall x }Gx}{\mask{A}}

\ai{\mc{\lforall x }Gx}{\lthereis x \lnot Fx\mc{\lor }\lforall x Gx}{\mask{8,
\disjI}}

\ai{\lforall x Fx\mc{\limplies }\lforall x Gx}{\lthereis x \lnot Fx\mc{\lor }
\lforall x Gx}{\mask{3,7,9,\disjE}}

\end{argumentN}

   \item \lbh{-p -s CUxFxXxGx:AXxNFxXxGx}. Prove this.
	
	\opts{
	 \dotline
	 \dotline
	 \dotline
	 \dotline
	 \dotline
	 \dotline
	 \dotline
	 \dotline
	 \dotline
	 \dotline
	 \dotline
	 \dotline
	 \dotline
	}{


\begin{argumentN}[1]
%generated by gentzen

\ai{\lforall x Fx\mc{\limplies }\lthereis x Gx}{\lforall x Fx\mc{\limplies }\lthereis x Gx}{A}

\ai{}{(\lforall x Fx\limplies \lthereis x Gx)\mc{\limplies }(\lnot \lforall x Fx\lor \lthereis x Gx)}{IM}

\ai{\lforall x Fx\mc{\limplies }\lthereis x Gx}{\lnot \lforall x Fx\mc{\lor }\lthereis x Gx}{1,2,\condE}

\ai{\mc{\lnot }\lforall x Fx}{\mc{\lnot }\lforall x Fx}{A}

\ai{}{\lnot \lforall x Fx\mc{\limplies }\lthereis x \lnot Fx}{QE}

\ai{\mc{\lnot }\lforall x Fx}{\mc{\lthereis x }\lnot Fx}{4,5,\condE}

\ai{\mc{\lnot }\lforall x Fx}{\lthereis x \lnot Fx\mc{\lor }\lthereis x Gx}{6,\disjI}

\ai{\mc{\lthereis x }Gx}{\mc{\lthereis x }Gx}{A}

\ai{\mc{\lthereis x }Gx}{\lthereis x \lnot Fx\mc{\lor }\lthereis x Gx}{8,\disjI}

\ai{\lthereis x Fx\mc{\limplies }\lthereis x Gx}{\lthereis x \lnot Fx\mc{\lor }\lthereis x Gx}{3,7,9,\disjE}

\end{argumentN}

}

\newpage
\item \lbh{-p -s CXxFxUxGx:AUxNFxUxGx}. Prove this.

 \opts{
	 \dotline
	 \dotline
	 \dotline
	 \dotline
	 \dotline
	 \dotline
	 \dotline
	 \dotline
	 \dotline
	 \dotline
	 \dotline
	 \dotline
	 \dotline
	}{


\begin{argumentN}[1]
%generated by gentzen

\ai{\lthereis x Fx\mc{\limplies }\lforall x Gx}{\lthereis x Fx\mc{\limplies }\lforall x Gx}{A}

\ai{}{(\lthereis x Fx\limplies \lforall x Gx)\mc{\limplies }(\lnot \lthereis x Fx\lor \lforall x Gx)}{IM}

\ai{\lthereis x Fx\mc{\limplies }\lforall x Gx}{\lnot \lthereis x Fx\mc{\lor }\lforall x Gx}{1,2,\condE}

\ai{\mc{\lnot }\lthereis x Fx}{\mc{\lnot }\lthereis x Fx}{A}

\ai{}{\lnot \lthereis x Fx\mc{\limplies }\lforall x \lnot Fx}{QE}

\ai{\mc{\lnot }\lthereis x Fx}{\mc{\lforall x }\lnot Fx}{4,5,\condE}

\ai{\mc{\lnot }\lthereis x Fx}{\lforall x \lnot Fx\mc{\lor }\lforall x Gx}{6,\disjI}

\ai{\mc{\lforall x }Gx}{\mc{\lforall x }Gx}{A}

\ai{\mc{\lforall x }Gx}{\lforall x \lnot Fx\mc{\lor }\lforall x Gx}{8,\disjI}

\ai{\lthereis x Fx\mc{\limplies }\lforall x Gx}{\lforall x \lnot Fx\mc{\lor }\lforall x Gx}{3,7,9,\disjE}

\end{argumentN}

}

\item \lbh{-p -s CXxFxXxGx:AUxNFxXxGx}. Prove this.

\opts{
\dotline
\dotline
\dotline
\dotline
\dotline
\dotline
\dotline
\dotline
\dotline
\dotline
\dotline
\dotline
}
{

\begin{argumentN}[1]
%generated by gentzen

\ai{\lthereis x Fx\mc{\limplies }\lthereis x Gx}{\lthereis x Fx\mc{\limplies }\lthereis x Gx}{A}

\ai{}{(\lthereis x Fx\limplies \lthereis x Gx)\mc{\limplies }(\lnot \lthereis x Fx\lor \lthereis x Gx)}{IM}

\ai{\lthereis x Fx\mc{\limplies }\lthereis x Gx}{\lnot \lthereis x Fx\mc{\lor }\lthereis x Gx}{1,2,\condE}

\ai{\mc{\lnot }\lthereis x Fx}{\mc{\lnot }\lthereis x Fx}{A}

\ai{}{\lnot \lthereis x Fx\mc{\limplies }\lforall x \lnot Fx}{QE}

\ai{\mc{\lnot }\lthereis x Fx}{\mc{\lforall x }\lnot Fx}{4,5,\condE}

\ai{\mc{\lnot }\lthereis x Fx}{\lforall x \lnot Fx\mc{\lor }\lthereis x Gx}{6,\disjI}

\ai{\mc{\lthereis x }Gx}{\mc{\lthereis x }Gx}{A}

\ai{\mc{\lthereis x }Gx}{\lforall x \lnot Fx\mc{\lor }\lthereis x Gx}{8,\disjI}

\ai{\lthereis x Fx\mc{\limplies }\lthereis x Gx}{\lforall x \lnot Fx\mc{\lor }\lthereis x Gx}{3,7,9,\disjE}

\end{argumentN}

}

\end{enumerate}


\newpage

\item Brackets matter. 

 \begin{enumerate}

  \item   Prove \lbh{-p -s XxCFxUyGy:AXxNFxUxGx}.  (Just so you see the 
   importance of brackets, compare this to 3c.)

 \opts{
\dotline
\dotline
\dotline
\dotline
\dotline
\dotline
\dotline
\dotline
\dotline
\dotline
\dotline
\dotline
\dotline
\dotline
\dotline
\dotline
}{

\begin{argumentN}[1]
%generated by gentzen

\ai{\mc{\lthereis x }(Fx\limplies \lforall y Gy)}{\mc{\lthereis x }(Fx\limplies \lforall y Gy)}{A}

\ai{Fa\mc{\limplies }\lforall y Gy}{Fa\mc{\limplies }\lforall y Gy}{A}

\ai{}{(Fa\limplies \lforall y Gy)\mc{\limplies }(\lnot Fa\lor \lforall y Gy)}{IM}

\ai{Fa\mc{\limplies }\lforall y Gy}{\lnot Fa\mc{\lor }\lforall y Gy}{2,3,\condE}

\ai{\mc{\lnot }Fa}{\mc{\lnot }Fa}{A}

\ai{\mc{\lnot }Fa}{\mc{\lthereis x }\lnot Fx}{5,\exI}

\ai{\mc{\lnot }Fa}{\lthereis x \lnot Fx\mc{\lor }\lforall x Gx}{6,\disjI}

\ai{\mc{\lforall y }Gy}{\mc{\lforall y }Gy}{A}

\ai{\mc{\lforall y }Gy}{Ga}{8,\uniE}

\ai{\mc{\lforall y }Gy}{\mc{\lforall x }Gx}{9,\uniI}

\ai{\mc{\lforall y }Gy}{\lthereis x \lnot Fx\mc{\lor }\lforall x Gx}{10,\disjI}

\ai{Fa\mc{\limplies }\lforall y Gy}{\lthereis x \lnot Fx\mc{\lor }\lforall x Gx}{4,7,11,\disjE}

\ai{\mc{\lthereis x }(Fx\limplies \lforall y Gy)}{\lthereis x \lnot Fx\mc{\lor }\lforall x Gx}{1,12,\exE}

\end{argumentN}
}

\newpage


\end{enumerate}

\newcommand{\ps}[1]{\lbh{-p -f #1}}
\item Construct models that fit the given specifications:

 \begin{enumerate}
 \cover{\setlength{\itemsep}{1.5em}}
  \item \ps{XxFx} true, \ps{UxGx} true.

   \opts{
	\dotline
	\dotline
	\dotline
	\dotline
	\dotline
	\dotline
   }{

	Domain of discourse: Ash, Beck.

	Referent of constants: \p{a} refers to Ash, \p{b} refers to Beck.

	Extension of \p{F}: \p{a}. 

	Extension of \p{G}: \p{a} and \p{b}.

   }
 
  \item \ps{XxFx} true, \ps{UxGx} false.

   \opts{
	\dotline
	\dotline
	\dotline
	\dotline
	\dotline
	\dotline
   }{

	Domain of discourse: Ash, Beck.

	Referent of constants: \p{a} refers to Ash, \p{b} refers to Beck.

	Extension of \p{F}: \p{a}. 

	Extension of \p{G}: \p{b}.
   }

  \item \ps{XxFx} false, \ps{UxGx} true.

   \opts{
	\dotline
	\dotline
	\dotline
	\dotline
	\dotline
	\dotline
   }{

	Domain of discourse: Ash, Beck.

	Referent of constants: \p{a} refers to Ash, \p{b} refers to Beck.

	Extension of \p{F}: empty. 

	Extension of \p{G}: \p{a} and \p{b}.
   }

  \item \ps{CXxFxUxGx} false.

   \opts{
	\dotline
	\dotline
	\dotline
	\dotline
	\dotline
	\dotline
   }{

same as (b)
   }
 \end{enumerate}

\newpage

\item Explain why it is not possible to construct a model that fits the given 
 specifications?
 \begin{enumerate}
 \cover{\setlength{\itemsep}{2em}}

  \item  \ps{UxCFxGx} true, \ps{UxANFxGx} false.

   \opts{
	\dotline
	\dotline
	\dotline
	\dotline
	\dotline
	\dotline
   }
   {
	\p{s_1\limplies s_2} and \p{\lnot s_1
	\lor s_2} are logically equivalent. So for any constant \p{\kappa}, 
	\p{F\kappa\limplies G\kappa} is true iff. \p{\lnot F\kappa \lor G\kappa}.  
	So we cannot make \ps{UxCFxGx} true without making \ps{UxANFxGx} also true.
   }




  \item \ps{UxCFxGx} true, \ps{XxFx} true, \ps{XxGx} false.

   \opts{
	\dotline
	\dotline
	\dotline
	\dotline
	\dotline
	\dotline
   }
   {
	If the first sentence is true, then anything that is \p{F} is also \p{G}. So 
	if there is something that is \p{F} as required by the truth of the second 
	sentence, then the third sentence, \ps{XxGx}, must also be true.
   }
 
  \item \ps{XxFx} false, \ps{XxGx} false, \ps{XxAFxGx} true.

   \opts{
	\dotline
	\dotline
	\dotline
	\dotline
	\dotline
	\dotline
   }{
	If anything is such that it is \p{F} or \p{G} as demanded by the third 
	sentence, it must be \p{F} or \p{G}. So at least one of \ps{XxFx} and 
	\ps{XxGx} must be true.

   }
 \end{enumerate}
 
 


\end{enumerate}  

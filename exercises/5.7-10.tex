\section*{Exercises for 5.7 to 5.10}

\begin{enumerate}

\item Consider: derivation: everyone gets grumpy when they are hungry; Erika is 
 hungry; so someone is grumpy.  Add the missing annotations in the formalization 
 of this (\p{e} is a constant referring to Erika; \p{Hx} means `x is hungry', 
 \p{Gx} that `x is grumpy'; \p{e} does not occur in \p{\Gamma,\Delta}):


\begin{argumentN}[1]
%generated by gentzen

\ai{Γ}{\mc{\lforall x }(Hx\limplies Gx)}{premise}

\ai{Δ}{He}{premise}

\ai{Γ}{He\mc{\limplies }Ge}{\mask{1,\uniE}}

\ai{Γ, Δ}{Ge}{\mask{2,3,\condE}}

\ai{Γ, Δ}{\mc{\lthereis x }Gx}{\mask{4,\exI}}

\end{argumentN}


\item Consider: All philosophy majors take logic. So if everyone majors in 
 philosophy, everyone takes logic. Below is a formalization this.  Add the 
 missing annotations:
 (\p{Px} means `x majors in philosophy', \p{Lx} means `x takes logic'): 


\begin{argumentN}[1]
%generated by gentzen

\ai{Γ}{\mc{\lforall x }(Px\limplies Lx)}{premise}

\ai{\mc{\lforall x }Px}{\mc{\lforall x }Px}{\mask{A}}

\ai{\mc{\lforall x }Px}{Pa}{\mask{2,\uniE}}

\ai{Γ}{Pa\mc{\limplies }La}{\mask{1,\uniE}}

\ai{Γ, \mc{\lforall x }Px}{La}{\mask{3,4,\condE}}

\ai{Γ, \mc{\lforall x }Px}{\mc{\lforall x }Lx}{\mask{5,\uniI}}

\ai{Γ}{\lforall x Px\mc{\limplies }\lforall x Lx}{\mask{6,\condI}}

\end{argumentN}

\item Add missing annotations:

\begin{argumentN}[1]
%generated by gentzen

\ai{Γ}{\mc{\lthereis x }(Fx\land Gx)}{premise}

\ai{Fa\mc{\land }Ga}{Fa\mc{\land }Ga}{\mask{A}}

\ai{Fa\mc{\land }Ga}{Fa}{\mask{2,\conjE}}

\ai{Fa\mc{\land }Ga}{\mc{\lthereis x }Fx}{\mask{3,\exI}}

\ai{Fa\mc{\land }Ga}{Ga}{\mask{2,\conjE}}

\ai{Fa\mc{\land }Ga}{\mc{\lthereis x }Gx}{\mask{5,\exI}}

\ai{Fa\mc{\land }Ga}{\lthereis x Fx\mc{\land }\lthereis x Gx}{\mask{4,6,\conjI}}

\ai{Γ}{\lthereis x Fx\mc{\land }\lthereis x Gx}{\mask{1,7,\exE}}

\end{argumentN}



\item What is wrong with the following attempted derivation?

 \begin{argument*}
  \ai{\Gamma}{\lforall x(Fx\lor Gx)}{premise}
  \ai{\Gamma}{Fc \lor Gc}{1,\uniE}
  \ai{Fc}{Fc}{A}
	\ai{Fc}{\lforall xFx}{3,\uniI}
	\ai{Fc}{\lforall xFx \lor \lforall xGx}{4,\disjI}
	\ai{Gc}{Gc}{A}
	\ai{Gc}{\lforall xGx}{6,\uniI}
	\ai{Gc}{\lforall xFx \lor\lforall xGx}{7,\disjI}
	\ai{\Gamma}{\lforall xFx\lor\lforall xGx}{2,5,8,\disjE}
 \end{argument*}

 \answer{

  On lines 4 and 7, \uniI{} is misapplied as the rule requires that \p{c} not 
  appear in any of the sentences on the datum side of lines 3 and 6 
  respectively.

 }

\item Add missing annotations:

 
\begin{argumentN}[1]
%generated by gentzen

 \ai{\mc{\lthereis x }(Fx\lor Gx)}{\mc{\lthereis x }(Fx\lor Gx)}{\mask{A}}

 \ai{Fa\mc{\lor }Ga}{Fa\mc{\lor }Ga}{\mask{A}}

 \ai{Fa}{Fa}{\mask{A}}

 \ai{Fa}{\mc{\lthereis x }Fx}{\mask{3,\exI}}

 \ai{Fa}{\lthereis x Fx\mc{\lor }\lthereis x Gx}{\mask{4,\disjI}}

 \ai{Ga}{Ga}{\mask{A}}

 \ai{Ga}{\mc{\lthereis x }Gx}{\mask{6,\exI}}

 \ai{Ga}{\lthereis x Fx\mc{\lor }\lthereis x Gx}{\mask{7,\disjI}}

 \ai{Fa\mc{\lor }Ga}{\lthereis x Fx\mc{\lor }\lthereis x Gx}{\mask{2,5,8,\disjE}}

\ai{\mc{\lthereis x }(Fx\lor Gx)}{\lthereis x Fx\mc{\lor }\lthereis x Gx}
{\mask{1,
 9,
\exE}}

\end{argumentN}

\newpage

\item Add missing items.

\begin{argumentN}[1]
%generated by gentzen

\ai[0.33]{\mc{\lforall x }(Fx\limplies Gx)}{\mc{\lforall x }(Fx\limplies Gx)}{A}

\ai[0.33]{\mc{\lforall x }Fx}{\mc{\lforall x }Fx}{A}

\ai[0.33]{\mc{\lforall x }Fx}{Fa}{\mask{2,\uniE}}

\ai[0.33]{\mask{\mc{\lforall x }(Fx\limplies Gx)}}{\mask{Fa\mc{\limplies }Ga}}{1,\uniE}

\ai[0.33]{\mc{\lforall x }(Fx\limplies Gx), \mc{\lforall x }Fx}{Ga}{3,4,\condE}

\ai[0.33]{\mask{\mc{\lforall x }(Fx\limplies Gx), \mc{\lforall x }Fx}}
{\mask{\mc{\lforall x }
Gx}}
{5,
\uniI}

\ai[0.33]{\mask{\mc{\lforall x }(Fx\limplies Gx)}}{\mask{\lforall x 
  Fx\mc{\limplies }
\lforall x Gx}}
{6,\condI}

\ai[0.25]{}{\lforall x (Fx\limplies Gx)\mc{\limplies }(\lforall x Fx\limplies 
\lforall x Gx)}{7,\condI}

\end{argumentN}


\item What's wrong with the following derivation (\p{a} is a constant not 
 occurring in \p{\Gamma}):


%title: %
\begin{argumentN}[1]
%generated  by  gentzen

\ai{\Gamma }{\lthereis  x  Fx\mc{\limplies  }\lthereis  x  Gx}{premise}

\ai{\mc{\lthereis  x  }Fx}{\mc{\lthereis  x  }Fx}{A}

\ai{\Gamma ,  \mc{\lthereis  x  }Fx}{\mc{\lthereis  x  }Gx}{1,2,\condE}

\ai{Ga}{Ga}{A}

\ai{Fa,  Ga}{Ga}{4}

\ai{Ga}{Fa\mc{\limplies  }Ga}{5,\condI}

\ai{Ga}{\mc{\lthereis  x  }(Fx\limplies  Gx)}{6,\exI}

\ai{\Gamma ,  Ga}{\mc{\lthereis  x  }(Fx\limplies  Gx)}{7}

\ai{\Gamma }{\mc{\lthereis  x  }(Fx\limplies  Gx)}{3,8,\exE}

\end{argumentN}


\answer{

 The datum of the last line is wrong. It needs to be:
 \begin{argumentN}[9]
	\ai{\Gamma,\lthereis xFx}{\lthereis x(Fx\limplies Gx)}{3,8,\exE}
   \end{argumentN}
  }


  \item Give an example illustrating why line 9 in the previous problem does not 
   follow from line 1.

\opts{

 \dotline
 \dotline
 \dotline
 \dotline
 \dotline
 \dotline
}
{

	Here is one: If there are winners, then there are losers. It does not follow 
   from that that there are some who when they win they lose.
  }

\item Here is something you might be tempted to do: from \p{\seq{\Gamma}
 {\lforall x(Fx\lor Gx)}} derive    \p{\seq{\Gamma}{\lforall xFx\lor \lforall 
xGx}}. Explain using an example why this would be a fallacious inference.
\opts{

 \dotline
 \dotline
 \dotline
 \dotline
 \dotline
 \dotline
}
{

 Let \p{Fx} mean that x is a Pomona student, and let \p{Gx} mean that x is a HMC 
 student. It does not follow from that that everyone is a Pomona student or 
 everyone is an HMC student.
}

\end{enumerate}

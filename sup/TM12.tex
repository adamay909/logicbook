
\documentclass[border=5pt]{standalone}

\usepackage{logic_flowchart}

\begin{document}

\begin{tikzpicture}

 \tikzset{node distance = 1cm and 1cm}
 
 \node(start) [terminal] {START};
 
 \node(1) [decision, below=of start] {Is 0?};

 \node(2) [process, right=of 1] {write 0};

 \node(3) [process, below=of 2] {move left};

 \node (4) [decision, below=of 3] {Is 0?};

 \node(5) [process, below=of 4] {write 0};

 \node(6) [process, below=of 5] {move right};

 \node(7) [process, right=of 4] {write 0};

 \node(8) [process, below=of 7] {move right};

 \node(100) [process, left=of 1] {write 0};

 \node(101) [process, below=of 100] {move left};

 \node (102) [decision, below=of 101] {Is 0?};

 \node(103) [process, below=of 102] {write 0};

 \node(106) [process, below=of 103] {move right};

 \node(107) [process, right=of 102] {write 0};

 \node(108) [process, below=of 107] {move right};

 \node(110) [terminal, below=of 108] {STOP};
\end{tikzpicture}

\end{document}



